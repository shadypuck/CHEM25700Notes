\documentclass[../psets.tex]{subfiles}

\pagestyle{main}
\renewcommand{\leftmark}{Problem Set \thesection}
\setcounter{section}{6}

\begin{document}




\section{Broader Classes of Groups}
\begin{enumerate}
    \item \marginnote{11/28:}Suppose that $\Z/m\Z$ is a subgroup of $S_n$ for some $n,m>2$. Prove that $D_{2m}$ is also a subgroup of $S_n$.
    \item Let $G=\text{SL}_2(\F_3)$. Prove that the subgroup
    \begin{equation*}
        H = \gen{
            \begin{pmatrix}
                0 & -1\\
                1 & 0\\
            \end{pmatrix},
            \begin{pmatrix}
                1 & 1\\
                1 & -1\\
            \end{pmatrix},
            \begin{pmatrix}
                1 & -1\\
                -1 & -1\\
            \end{pmatrix}
        }
    \end{equation*}
    is isomorphic to the quaternion group $Q$ (where $i,j,k$ map to the given matrices). Deduce that $\text{SL}_2(\F_3)$ and $S_4$ are not isomorphic.
    \item Let $G$ be a group, and let $N\subset G$ be the subgroup generated by the elements $xyx^{-1}y^{-1}$ for all pairs $x,y\in G$. Prove that $N$ is a normal subgroup, and that $G/N$ is abelian.
    \item Compute the order of the following groups as well as a set of generators.
    \begin{enumerate}
        \item The centralizer of $(12345)$ in $A_7$.
        \item The centralizer of $((12),(123))$ in $S_5\times S_5$.
        \item The normalizer of $H=\gen{(12),(34),(56),(78)}$ in $S_8$.
    \end{enumerate}
    \item \textbf{Projective Linear Groups Over Finite Fields.} Let $p$ be prime, and let $\F_p=\Z/p\Z$. Note that one can add and multiply elements of $\F_p$. Let $\text{GL}_2(\F_p)$ be the group of $2\times 2$ invertible matrices over $\F_p$, and let $\text{SL}_2(\F_p)\subset\text{GL}_2(\F_p)$ denote the subgroup of matrices of determinant one.
    \begin{enumerate}
        \item There are $p^2-1$ non-zero vectors $v\in\F_p^2$. Let a "line" $\ell=[v]\subset\F_p^2$ denote the scalar multiples $\lambda v$ of a non-zero vector $v$. Prove that the set $X$ of lines has cardinality $|X|=p+1$.
        \item Prove that $\text{SL}_2(\F_p)$ and $\text{GL}_2(\F_p)$ act naturally on $X$ by $g\cdot[v]=[g\cdot v]$.
        \item Prove that this action is transitive for both $\text{GL}_2(\F_p)$ and $\text{SL}_2(\F_p)$.
        \item Prove that the kernel of the action consists precisely of the scalar matrices $\lambda I$ in either $\text{SL}_2(\F_p)$ or $\text{GL}_2(\F_p)$.
        \item Let $\text{PGL}_2(\F_p)$ and $\text{PSL}_2(\F_p)$ denote the quotient of $G$ and $H$ by the subgroup of scalar matrices. Prove that $|\text{PGL}_2(\F_p)|=(p^2-1)p$ and $|\text{PSL}_2(\F_p)|=6$ if $p=2$ and $\frac{1}{2}(p^2-1)p$ otherwise.
        \item Prove that $\text{PGL}_2(\F_2)=\text{PSL}_2(\F_2)=S_3$.
        \item Prove that $\text{PGL}_2(\F_3)=S_4$ and $\text{PSL}_2(\F_3)=A_4$. (Compare with Question 2.)
        \item Prove that $\text{PSL}_2(\F_5)=A_5$ and $\text{PGL}_2(\F_5)=S_5$. (Hint: Using that $A_6$ is simple, prove that any index 6 subgroup of $A_6$ or $S_6$ is $A_5$ or $S_5$, respectively.)
    \end{enumerate}
\end{enumerate}




\end{document}