\documentclass[../psets.tex]{subfiles}

\pagestyle{main}
\renewcommand{\leftmark}{Problem Set \thesection}
\setcounter{section}{5}

\begin{document}




\section{Theory of Group Actions}
\marginnote{11/14:}You should think about and try to solve the starred questions, but several of them are quite messy and some are difficult, so only submit the ones without stars.
\begin{enumerate}
    \item Exercises 4.1.7-4.1.8 of \textcite{bib:DummitFoote}.
    \begin{enumerate}[label={\textbf{\arabic*.}}]
        \setcounter{enumii}{6}
        \item Let $G$ be a transitive permutation group on the finite set $A$. A \textbf{block} is a nonempty subset $B$ of $A$ such that for all $\sigma\in G$, either $\sigma(B)=B$ or $\sigma(B)\cap B=\emptyset$ (here $\sigma(B)$ is the set $\{\sigma(b)\mid b\in B\}$).
        \begin{enumerate}[label={\textbf{(\alph*)}}]
            \item Prove that if $B$ is a block containing the element $a\in A$, then the set $G_B$ defined by $G_B=\{\sigma\in G\mid\sigma(B)=B\}$ is a subgroup of $G$ containing $G_a$.
            \item Show that if $B$ is a block and $\sigma_1(B),\sigma_2(B),\dots,\sigma_n(B)$ are all the distinct images of $B$ under the elements of $G$, then these form a partition of $A$.
            \item A (transitive) group $G$ on a set $A$ is said to be \textbf{primitive} if the only blocks in $A$ are the trivial ones: The sets of size 1 and $A$ itself. Show that $S_4$ is primitive on $A=\{1,2,3,4\}$. Show that $D_8$ is not primitive as a permutation group on the four vertices of a square.
            \item Prove that the transitive group $G$ is primitive on $A$ if and only if for each $a\in A$, the only subgroups of $G$ containing $G_a$ are $G_a$ and $G$ (i.e., $G_a$ is a \textbf{maximal} subgroup of $G$). \emph{Hint.} See Exercise 2.4.16. Use part (a).
        \end{enumerate}
        \item A transitive permutation group $G$ on a set $A$ is called \textbf{doubly transitive} if for any (hence all) $a\in A$, the subgroup $G_a$ is transitive on the set $A\setminus\{a\}$.
        \begin{enumerate}[label={\textbf{(\alph*)}}]
            \item Prove that $S_n$ is doubly transitive on $\{1,2,\dots,n\}$ for all $n\geq 2$.
            \item Prove that a doubly transitive group is primitive. Deduce that $D_8$ is not doubly transitive in its action on the four vertices of a square.
        \end{enumerate}
    \end{enumerate}
    \item Exercise 4.2.9 of \textcite{bib:DummitFoote}.
    \begin{enumerate}[label={\textbf{\arabic*.}}]
        \setcounter{enumii}{8}
        \item Prove that if $p$ is a prime and $G$ is a group of order $p^\alpha$ for some $\alpha\in\Z^+$, then every subgroup of index $p$ is normal in $G$. Deduce that every group of order $p^2$ has a normal subgroup of order $p$.
    \end{enumerate}
    \item Suppose that $G$ acts transitively and faithfully on a finite set $X$, and that $G$ is abelian. Prove that $|G|=|X|$. Show that the equality need not hold if $G$ is not abelian.
    \item Let $G$ be a finite group and let $H$ be any subgroup.
    \begin{enumerate}
        \item Prove that the left action of $G$ on the coset space $G/H$ has kernel $N=\bigcap_{g\in G}gHg^{-1}$.
        \item Prove that $N=\bigcap_{g\in G}gHg^{-1}$ is the largest normal subgroup of $G$ contained in $H$.
    \end{enumerate}
    \item \textbf{The Quaternions.} Let $\HH=\R\oplus\R i\oplus\R j\oplus\R k$ be a 4-dimensional vector space over $\R$. Define a non-commutative associative multiplication structure on $\HH$ by the formulae
    \begin{align*}
        ij &= -ji = k&
        jk &= -kj = i&
        ki &= -ik = j&
        i^2 &= j^2 = k^2 = -1
    \end{align*}
    \begin{enumerate}
        \item ($\star$) Show that there is a map $\phi:\HH\to M_2(\C)$, where $M_2(\C)$ is the vector space of $2\times 2$ matrices over $\C$, defined by sending
        \begin{align*}
            i &\mapsto
            \begin{pmatrix}
                \sqrt{-1} & 0\\
                0 & -\sqrt{-1}\\
            \end{pmatrix}&
            j &\mapsto
            \begin{pmatrix}
                0 & 1\\
                -1 & 0\\
            \end{pmatrix}&
            k &\mapsto
            \begin{pmatrix}
                0 & \sqrt{-1}\\
                \sqrt{-1} & 0\\
            \end{pmatrix}
        \end{align*}
        for which
        \begin{enumerate}
            \item $\phi$ is injective as a map of vector spaces over $\R$.
            \item $\phi$ respects multiplication; if $q_1,q_2$ are two quaternions, then $\phi(q_1q_2)=\phi(q_1)\phi(q_2)$. This should reduce easily enough to the case where $q_i,q_j$ are elements of the set $\phi(1),\phi(i),\phi(j),\phi(k)$. The map $\phi$ is not a group homomorphism since 0 is not an invertible quaternion, but we shall see below in part (c) that non-zero quaternions form a group, so $\phi$ restricted to $\HH^\times$ is actually a homomorphism from $\HH^\times$ to $\text{GL}_2(\C)$.
        \end{enumerate}
        \item Define the conjugate of a quaternion $q=a+bi+cj+dk$ by $\bar{q}:=a-bi-cj-dk$. Prove that $N(q):=q\bar{q}=a^2+b^2+c^2+d^2$.
        \item Prove that non-zero quaternions $\HH^\times$ form a group under multiplication.
        \item Let $Q=\gen{i,j}$ be the subgroup of $\HH^\times$ generated by $i,j$. Prove that $Q$ is a group of order 8. ($Q$ is known as the "quaternion group.")
        \item Prove that every subgroup of $Q$ is normal.
        \item Let $N=\pm 1\subset Q$. Prove that $Q/N\cong(\Z/2\Z)^2$ and that $Q/N$ is not isomorphic to a subgroup of $Q$.
        \item ($\star$) Let $\Gamma$ be the subgroup of $\HH^\times$ generated by the elements of $Q$ together with $\frac{1}{2}(1+i+j+k)$. Prove that $\Gamma$ is a group of order 24.
        \item Prove that $\Gamma$ is \emph{not} isomorphic to $S_4$, and $Q$ is \emph{not} isomorphic to $D_8$. In fact, $\Gamma=\text{SL}_2(\F_3)$.
        \item ($\star$) Construct a surjective homomorphism from $\Gamma$ to $A_4$.
        \item Prove that the subgroup $\HH^1$ of quaternions $q$ with $N(q)=1$ is a subgroup of $\HH^\times$. Deduce that the 3-sphere $S^3\subset\R^4$ defined by $a^2+b^2+c^2+d^2=1$ has a natural structure of a group. Note that $S^1$ also has a natural group structure given by rotations in $\text{SO}(2)$. It turns out that $S^n$ has a natural (i.e., continuous) group structure only for $n=1$ and $n=3$.
        \item ($\star$) Say that a quaternion is \textbf{pure} if it is of the form $bi+cj+dk$, i.e., $a=0$. We may identify pure quaternions with $\R^3$. Show that if $u$ is a pure quaternion, then $quq^{-1}$ is still a pure quaternion for any $q\in\HH^\times$.
        \item ($\star$) Prove that the action of $q$ on $\R^3$ by $q\cdot u=quq^{-1}$ is via elements of $\text{SO}(3)$, and deduce that there is a homomorphism $\HH^\times\to\text{SO}(3)$.
        \item ($\star$) Prove that the restriction of this homomorphism to $\HH^1\to\text{SO}(3)$ is surjective and has kernel of order 2.
    \end{enumerate}
\end{enumerate}




\end{document}