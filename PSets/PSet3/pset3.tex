\documentclass[../psets.tex]{subfiles}

\pagestyle{main}
\renewcommand{\leftmark}{Problem Set \thesection}
\setcounter{section}{2}

\begin{document}




\section{Subgroups and Group Functions}
\begin{enumerate}
    \item \marginnote{10/17:}Let $\sigma\in S_n$ be an $n$-cycle, and let $\tau\in S_n$ be a 2-cycle. Show by constructing a counterexample that there exists a choice $\sigma,\tau,n$ such that $\gen{\sigma,\tau}\neq S_n$. Bonus Question: Determine for which $n$ such an example exists.
    \begin{proof}
        % We know that every $S_n=\gen{\sigma,\tau}$ where $\sigma=(1,\dots,n)$ and $\tau=(n,n+1)$, because this allows us to generate all elementary transpositions, which allows us to generate all transpositions, which allows us to generate all permutations.
        % What if we space out the $n$-cycle, i.e., have it increment by 2's? And our transposition likewise.
        % Find all subgroups of $S_n$; find a subgroup that is generated by an $n$-cycle and a 2-cycle.

        % We need an $n$-cycle $g$ such that no power of $g$ is equal to the shift-by-one $n$-cycle. An $n$-cycle necessarily has order $n$.

        % Pick $n=4$, $\sigma=(1,3,2,4)$, and $\tau=(1,2)$. We have that $\sigma^2=(1,2)(3,4)$. $\sigma\tau=(1,4)(2,3)$. $\tau\sigma=(1,3)(2,4)$. $\tau\sigma^2=\sigma^2\tau=(3,4)$. Then it is impossible to construct $(2,3)$.


        % Pick $n=4$, $\sigma=(1,2,3,4)$, and $\tau=(2,4)$. Both $\sigma,\tau$ satisfy the rule "$f(a_i)+f[a_{(i+2)\mod\len(f)}]$ is an even number," as can readily be confirmed by casework. However, there are elements of $S_4$ that do not satisfy this rule, e.g., $(1,2,3)$ --- here, we have that
        % \begin{equation*}
        %     f(a_1)+f[a_{(1+2)\mod 3}] = f(a_1)+f(a_3)
        %     = f(1)+f(3)
        %     = 2+1
        %     = 3
        % \end{equation*}
        % which is odd.
        % Adi's rule: $\sigma(i)+\sigma((i+2)\mod 4)$ is even. Works for $\gen{(1,3),(1,2,3,4)}$.


        % It's about the Dihedral group being a subgroup. $\sigma$ is a rotation; $\tau$ is a reflection. Since $\sigma,\tau$ are rigid, transformations but not every $f\in S_n$ is, there's our differentiation.

        % For all $n\geq 4$; we can always choose a dihedral subgroup $D_{2n}\subset S_n$ for these $n$ which will be generated by an $n$-cycle (rotation) and 2-cycle (reflection). For $n=5$, we can choose $\tau=(2,5)(3,4)$. It's always gonna be $(2,n)(3,n-1)\cdots$.

        % We should be able to prove that $\gen{\sigma,\tau}\cong D_{2n}$. Since $\sigma$ is an $n$-cycle, we know that $|\sigma|=n$. Since $\tau$ is a 2-cycle, we know that $|\tau|=2$. Suppose for the sake of contradiction that there exists $0\leq i\leq n-1$ such that $\tau=\sigma^i$. Then $\sigma^{2i}=\tau^2=e$, so $i=n/2$. If $n$ is odd, we have arrived at a contradiction. If $n$ is even, then consider that $\tau(1)=1$ but $\sigma^i(1)=1+n/2$, a contradiction if we are assuming that $\tau=\sigma^i$. Suppose for the sake of contradiction that $\tau\sigma^i=\tau\sigma^j$ for some $i\neq j$ satisfying $0\leq i,j\leq n-1$. It follows that $\sigma^i=\sigma^j$. But this contradicts claim 1.
        % It follows that $\{e,\sigma,\sigma^2,\dots,\sigma^{n-1},\tau,\tau\sigma,\tau\sigma^2,\dots,\tau\sigma^{n-1}\}\subset\gen{\sigma,\tau}$.
        % Now wish to show that $\sigma\tau=\tau\sigma^{-1}$. By direct computation, we can show that
        % \begin{equation*}
        %     \sigma\tau = (2,1)(3,n)(4,n-1)(5,n-2)\cdots
        %     = \tau\sigma^{-1}
        % \end{equation*}
        % It follows by induction that $\sigma^i\tau=\tau\sigma^{-i}$: The previous step was the base case, and the inductive step is
        % \begin{equation*}
        %     \sigma^{i+1}\tau = \sigma(\sigma^i\tau)
        %     = \sigma(\tau\sigma^{-i})
        %     = (\sigma\tau)\sigma^{-i}
        %     = (\tau\sigma^{-1})\sigma^{-i}
        %     = \tau\sigma^{-(i+1)}
        % \end{equation*}
        % Now let $x\in\gen{\sigma,\tau}$ be arbitrary. Then $x=\sigma^i\tau^j\sigma^k\tau^\ell\cdots$. Any power $\tau$ is raised to that is not $1\mod 2$ will make that term equal to the identity. Thus, we may rewrite $x=\sigma^i\tau\sigma^j\tau\cdots$. Invoking the above rule, we can combine $\tau$'s further:
        % \begin{equation*}
        %     x = \sigma^i\tau\tau\sigma^{-j}\cdots
        %     = \sigma^{i-j}\cdots
        % \end{equation*}
        % Thus, $\tau$ appears in the condensed decomposition of $x$ iff $\tau$ appears an odd number of times in the expanded decomposition. In other words, $\tau$ appears at most once. Moreover, the other term will be composed entirely of $\sigma$ raised to some power, which we can take mod $n$. Thus, $x=\tau^k\sigma^i$ for some $k=0,1$ and $0\leq i\leq n-1$. Therefore, $x\in\{e,\sigma,\sigma^2,\dots,\sigma^{n-1},\tau,\tau\sigma,\tau\sigma^2,\dots,\tau\sigma^{n-1}\}$, so
        % \begin{equation*}
        %     \gen{\sigma,\tau} = \{e,\sigma,\sigma^2,\dots,\sigma^{n-1},\tau,\tau\sigma,\tau\sigma^2,\dots,\tau\sigma^{n-1}\}
        % \end{equation*}
        % Thus, $|\gen{\sigma,\tau}|=2n$. Note that the idea behind this argument is that $\gen{\sigma,\tau}\cong D_{2n}$.
        % Moreover, we know that $\gen{\sigma,\tau}\neq S_n$ whenever $2n=|\gen{\sigma,\tau}|\neq|S_n=n!$. But since $\gen{\sigma,\tau}\subset S_n$, $2n\leq n!$. Thus, a necessary condition for $\gen{\sigma,\tau}\neq S_n$ is $2n<n!$, which is true for $n\geq 4$.


        As a particular counterexample, we may pick
        \begin{empheq}[box=\fbox]{align*}
            n &= 4&
            \sigma &= (1,2,3,4)&
            \tau &= (2,4)
        \end{empheq}
        Notice that $\gen{\sigma,\tau}\cong D_4$ with $\sigma\sim r$ and $\tau\sim s$; this observation will motivate the remainder of our proof. We withhold a proof that $\gen{(1,2,3,4),(2,4)}\neq S_4$ in favor of proving the more general fact that for any
        \begin{equation*}
            \boxed{n \geq 4}
        \end{equation*}
        we may use the $n$-cycle $\sigma=(1,2,\dots,n)$ and the 2-cycle $\tau=(2,n)$ to generate a subgroup of $S_n$ of order $2n$. This fact will imply the desired result, as explained below. Let's begin.\par
        We will first show that $\sigma^i\tau=\tau\sigma^{-i}$ for $i=1,\dots,n-1$. For the base case $n=1$, we have by direct computation that
        \begin{equation*}
            \sigma\tau = (1,2)(3,4,\dots,n)
            = \tau\sigma^{-1}
        \end{equation*}
        Now suppose inductively that we have proven the claim for $i$. Then
        \begin{equation*}
            \sigma^{i+1}\tau = \sigma(\sigma^i\tau)
            = \sigma(\tau\sigma^{-i})
            = (\sigma\tau)\sigma^{-i}
            = (\tau\sigma^{-1})\sigma^{-i}
            = \tau\sigma^{-(i+1)}
        \end{equation*}
        as desired.\par
        We will now prove that
        \begin{equation*}
            \gen{\sigma,\tau} = \{e,\sigma,\sigma^2,\dots,\sigma^{n-1},\tau,\tau\sigma,\tau\sigma^2,\dots,\tau\sigma^{n-1}\}
        \end{equation*}
        via a bidirectional inclusion proof. This will imply our desired result by inspection. The right-to-left case follows directly from the definition of generators. For the left-to-right case, let $x\in\gen{\sigma,\tau}$ be arbitrary. Then $x$ is equal to a finite product of $\sigma$'s and $\tau$'s, i.e., $x=\tau^i\sigma^j\tau^k\sigma^\ell\cdots$. With respect to $i,k$, and other exponents of the $\tau$'s: If these numbers are not congruent to $1\mod 2$, then that term (e.g., $\tau^i,\tau^k,\dots$) is equal to the identity (because $|\tau|=2$). Thus, we may rewrite $x=\tau\sigma^i\tau\sigma^j\cdots$. Invoking the above rule, we can combine that $\tau$'s further:
        \begin{equation*}
            x = \tau\tau\sigma^{-i}\sigma^j\cdots
            = \sigma^{j-i}\cdots
        \end{equation*}
        It should not be hard to see that $\tau$ only appears in the fully condensed decomposition of $x$ iff $\tau$ appears an odd number of times in the expanded decomposition. In other words, $\tau$ appears at most once (and when it does show up, we can make it appear on the leftmost side of the equation). Moreover, the other term will be composed entirely of $\sigma$ raised to some power, which we can take mod $n$ since $|\sigma|=n$. Thus, $x=\tau^k\sigma^i$ for some $k=0,1$ and $0\leq i\leq n-1$. Therefore,
        \begin{equation*}
            x \in \{e,\sigma,\sigma^2,\dots,\sigma^{n-1},\tau,\tau\sigma,\tau\sigma^2,\dots,\tau\sigma^{n-1}\}
        \end{equation*}
        so we have the desired set equality. As stated above, it follows by inspection that $|\gen{\sigma,\tau}|=2n$, as desired.\par
        To have $\gen{\sigma,\tau}\neq S_n$, we want $2n<n!$ (recall that $|S_n|=n!$). This inequality is satisfied for $n\geq 4$, proving our result. Note that we can confirm by casework that there are no two elements $\sigma,\tau\in S_n$ for $n=1,2,3$ satisfying the desired conditions:\par
        $S_1$: We cannot pick a 2-cycle in $S_1$.\par
        $S_2$: The only 2-cycle in $S_2$ generates the entire set.\par
        $S_3$: $S_3$ is generated by $\gen{(1,2),(2,3)}$. Any 2- and 3-cycles we pick will generate these two transpositions.
    \end{proof}
    \item Shuffling Redux. Let $G$ be the subgroup generated by the union of the following elements.
    \begin{itemize}
        \item $(n,53-n)$ for all $n$;
        \item The element $(1,2,\dots,26)(52,51,\dots,27)$ of order 26;
        \item The element $(1,2)(51,52)$.
    \end{itemize}
    With this definition in mind, respond to the following.
    \begin{enumerate}
        \item Let $H=\gen{(n,53-n)\mid n\in[52]}$. Prove that $H\cong(\Z/2\Z)^{26}$ inside $S_{52}$.
        \begin{proof}
            % For every element of $H$, $n,53-n$ are either in the right order, or in the flipped order.

            % Define $\psi:\Z/2\Z\to 
            
            % as follows: Let $a$ be a 26-tuple in $(\Z/2\Z)^{26}$. If the $i^\text{th}$ entry of $a$ is 1, include $(i,53-i)$ in $\phi(a)$.


            Let $a=(a_1,\dots,a_{26})$ be a 26-tuple, every entry of which is either 1 or 0. Define $\psi:(\Z/2\Z)^{26}\to H$ by
            \begin{equation*}
                \psi(a) = \bigcirc_{i=1}^{26}(i,53-i)^{a_i}
            \end{equation*}
            To prove that $\psi$ is a homomorphism, it will suffice to show that $\psi(a+_2b)=\psi(a)\psi(b)$. But we have that
            \begin{align*}
                \psi(a+_2b) &= \bigcirc_{i=1}^{26}(i,53-i)^{a_i+_2b_i}\\
                &= \bigcirc_{i=1}^{26}(i,53-i)^{a_i}\circ(i,53-i)^{b_i}\\
                &= \left[ \bigcirc_{i=1}^{26}(i,53-i)^{a_i} \right]\circ\left[ \bigcirc_{i=1}^{26}(i,53-i)^{b_i} \right]\\
                &= \psi(a)\psi(b)
            \end{align*}
            where we get from the first to the second line via: If $a_i+b_i\leq 1$, regular exponent rules hold; if $a_i,b_i=1$, then $a_i+_2b_i=0$ and $(i,53-i)^{a_i+_2b_i}=e$ just the same as $(i,53-i)^1\circ(i,53-i)^1=(i,53-i)^2=e$. We get from the second to the third line since disjoint cycles commute.\par
            We verify bijectivity by noting that since the generators of $H$ are disjoint 2-cycles, every element of $H$ can be written in the form
            \begin{equation*}
                \bigcirc_{i=1}^{26}(i,53-i)^{a_i}
            \end{equation*}
            with every $a_i\in\{0,1\}$. Thus, $\psi^{-1}$ can be defined by sending each $a_i$ to the $i^\text{th}$ slot in the 26-tuple $a$. It will naturally follow that $\psi\circ\psi^{-1}=I=\psi^{-1}\circ\psi$, proving bijectivity.
        \end{proof}
        \item Show that there is a homomorphism $\phi:G\to S_{26}$ such that\dots
        \begin{enumerate}
            \item $\phi$ is surjective;
            \item $\ker\phi=H$.
        \end{enumerate}
        (It follows from this that $G$ has order $2^{26}\cdot 26!=27064431817106664380040216576000000$.)
        \begin{proof}
            % Let $i\in[26]$. Suppose $\tau(i)\in[26]$. Then $f(\tau)=\tau$, and we have the desired result. If $\tau(i)\in[27:52]$, then $f(\tau(i))=53-\tau(i)$. Then $\sigma(53-\tau(i))=53-\sigma(\tau(i))$ by the below rule. Then $f(53-\sigma(\tau(i)))=f(\sigma(\tau(i)))$ by the second below rule.

            % $\sigma,\tau$ both obey $F(53-n)=53-F(n)$
            % Prove that $f(i)=f(53-i)$.


            % \begin{itemize}
            %     \item WTS: $\phi(\sigma\tau)=\phi(\sigma)\phi(\tau)$ for all $\sigma,\tau\in G$.
            %     \begin{itemize}
            %         \item Let $\sigma,\tau\in G$ be arbitrary.
            %         \item All generators of $G$ obey $f(n)+f(53-n)=53$ plus HW1, Q1: $\sigma,\tau$ obey $f(n)+f(53-n)=53$.
            %         \item $h(i)=h(53-i)$:
            %         \begin{itemize}
            %             \item $i\in[26]$: $53-i\in[27:52]$, so $h(i)=i=53-(53-i)=h(53-i)$.
            %             \item $i\in[27:52]$: $53-i\in[26]$, so $h(i)=53-i=h(53-i)$.
            %         \end{itemize}
            %         \item $\phi(\sigma\tau)=h\circ(\sigma\tau)|_{[26]}=h(\sigma(\tau))$.
            %         \item $\phi(\sigma)\phi(\tau)=(h\circ\sigma|_{[26]})\circ(h\circ\tau|_{[26]})=h(\sigma(h(\tau)))$.
            %         \item Let $i\in[26]$. WTS: $h(\sigma(\tau(i)))=h(\sigma(h(\tau(i))))$. Divide into two cases.
            %         \item $\tau(i)\in[26]$:
            %         \begin{itemize}
            %             \item $h(\tau(i))=\tau(i)$, as desired.
            %         \end{itemize}
            %         \item $\tau(i)\in[27:52]$:
            %         \begin{align*}
            %             h(\sigma(h(\tau(i)))) &= h(\sigma(53-\tau(i)))\tag*{Definition of $h$}\\
            %             &= h(53-\sigma(\tau(i)))\tag*{Lemma 1}\\
            %             &= h(\sigma(\tau(i)))\tag*{Lemma 2}
            %         \end{align*}
            %     \end{itemize}
            % \end{itemize}


            Define $w:[52]\to[26]$ by
            \begin{equation*}
                w(i) =
                \begin{cases}
                    i & i\in[26]\\
                    53-i & i\in[27:52]
                \end{cases}
            \end{equation*}
            Define $\phi:G\to S_{26}$ by
            \begin{equation*}
                \phi(g) = w\circ g|_{[26]}
            \end{equation*}
            We now prove two lemmas.\par\smallskip
            Lemma 1: Any $f\in G$ obeys the functional rule $f(n)+f(53-n)=53$. This follows from the facts that all generators of $G$ obey said functional rule, $f$ is a composition of the generators of $G$, and compositions of functions that obey said functional rule obey said function rule (as per HW1, Q1).\par
            Lemma 2: $w(i)=w(53-i)$. We divide into two cases ($i\in[26]$ and $i\in[27:52]$). If $i\in[26]$, then $53-i\in[27:52]$, so $w(i)=i=53-(53-i)=w(53-i)$. If $i\in[27:52]$, then $53-i\in[26]$, so $w(i)=53-i=w(53-i)$.\par\smallskip
            To prove that $\phi$ actually maps elements of $G$ to $S_{26}$ as defined, it will suffice to show that for any $g\in G$, $\phi(g):[26]\to[26]$ is a bijection.\par
            Let $g\in G$ and $i\in[26]$ be arbitrary. We divide into two cases ($g(i)\in[26]$ and $g(i)\in[27:52]$). If $g(i)\in[26]$, then $w(g(i))=g(i)\in[26]$. If $g(i)\in[27:52]$, then $w(g(i))=53-g(i)\in[26]$. Therefore, $\phi(g):[26]\to[26]$.\par
            Now suppose $w(g(i))=w(g(j))$. If either $g(i),g(j)\in[27:52]$, invoke Lemmas 1-2 to rewrite $w(g(x))=w(53-g(x))=w(g(53-x))$. Since $g$, itself, has mirror symmetry, what we are essentially doing here is guaranteeing that both $i,j\in[26]$ or $i,j\in[27:52]$; there may be distinct $i,j\in[52]$ such that $w(g(i))=w(g(j))$ (namely, $i,53-i$), but we are going to show that there is only one $i,j\in[26]$ such that $w(g(i))=w(g(j))$. Continuing, based on our rewrite, we may assume that $g(i),g(j)\in[26]$. Now let
            \begin{equation*}
                w(i) =
                \begin{cases}
                    w_1(i) & i\in[26]\\
                    w_2(i) & i\in[27:52]
                \end{cases}
            \end{equation*}
            where $w_1,w_2$ are naturally bijections. Since $g(i),g(j)\in[26]$, we have
            \begin{align*}
                w(g(i)) &= w(g(j))\\
                w_1(g(i)) &= w_1(g(j))\\
                g(i) &= g(j)\\
                i &= j
            \end{align*}
            where the last line follows since $g\in S_{52}$ is a bijection by definition. Note that if $i,j\in[27:52]$, we may take $53-i=53-j$ to be the unique desired element of $[26]$.\par
            Lastly, let $j\in[26]$. It follows from the above that either $g^{-1}(w_1^{-1}(j))$ or $g^{-1}(w_2^{-1}(j))$ is an element of $[26]$, as desired.\par\smallskip
            To prove that $\phi$ is a homomorphism, it will suffice to show that $\phi(\sigma\tau)=\phi(\sigma)\phi(\tau)$ for all $\sigma,\tau\in G$. Let $\sigma,\tau\in G$ be arbitrary. Now notice that
            \begin{align*}
                \phi(\sigma\tau) &= w\circ(\sigma\tau)|_{[26]}
                = w(\sigma(\tau))&
                \phi(\sigma)\phi(\tau) &= (w\circ\sigma|_{[26]})\circ(w\circ\tau|_{[26]})
                = w(\sigma(w(\tau)))
            \end{align*}
            Thus, if we let $i\in[26]$ be arbitrary, it will suffice to show that $w(\sigma(\tau(i)))=w(\sigma(w(\tau(i))))$ to prove that $\phi$ is a homomorphism. We divide into two cases ($\tau(i)\in[26]$ and $\tau(i)\in[27:52]$). If $\tau(i)\in[26]$, then $w(\tau(i))=\tau(i)$, implying the desired result. If $\tau(i)\in[27:52]$, then
            \begin{align*}
                w(\sigma(w(\tau(i)))) &= w(\sigma(53-\tau(i)))\tag*{Definition of $w$}\\
                &= w(53-\sigma(\tau(i)))\tag*{Lemma 1}\\
                &= w(\sigma(\tau(i)))\tag*{Lemma 2}
            \end{align*}
            as desired.\par\smallskip
            To prove that $\phi$ is surjective, it will suffice to show that for all $\sigma\in S_{26}$, there exists $g\in G$ such that $\phi(g)=\sigma$. Note that this argument will be distinct (but closely related to) our earlier argument that $\phi(g)$ is surjective. Take
            \begin{align*}
                g(i) &=
                \begin{cases}
                    w_1^{-1}(\sigma(w(i))) & i\in[26]\\
                    w_2^{-1}(\sigma(w(i))) & i\in[27:52]
                \end{cases}\\
                &=
                \begin{cases}
                    \sigma(i) & i\in[26]\\
                    53-\sigma(53-i) & i\in[27:52]
                \end{cases}
            \end{align*}
            Now we must prove that $g\in G$. Recall from class that $S_{26}=\gen{(1,2),(1,2,\dots,26)}$, and note that $(1,2)(52,51)$ and $(1,2,\dots,26)(52,51,\dots,27)$ are generators of $G$. It follows that $(1,2)(52,51)$ and $(1,2,\dots,26)(52,51,\dots,27)$ generate all elements of $G$ that permute the elements of $[26]$, and do the same permutation symmetrically to $[27:52]$. This combined with the observations that $g:[26]\to[26]$, $g:[27:52]\to[27:52]$, and $g$ has obeys the mirror symmetry equation $f(n)+f(53-n)=53$ (as is evident from its definition) proves that $g$ is generated by these two generators, and is thus an element of $G$.\par
            To prove that $\ker\phi=H$, it will suffice to show that $\phi(h)=e\in S_{26}$ for all $h\in H$. Let $h\in H$ be arbitrary. Then since $h$ is the product of disjoint 2-cycles which are all mirror symmetric, we know that for every $i\in[26]$, $h$ either sends $i\mapsto i$ or $i\mapsto 53-i$. If $h:i\mapsto i$, then $w\circ h:i\mapsto i$. If $h:i\mapsto 53-i$, then $w(h(i))=w(53-i)=53-(53-i)=i$. Either way, $w\circ h$ is the identity on $[26]$, so $\phi(h)=e\in S_{26}$, as desired.
        \end{proof}
        \item Prove that the group generated by the two riffle shuffles is a subgroup of $G$. (In fact, they are equal.)
        \begin{proof}
            % This is there is some solution with the dihedral group.

            % The only property of $A,B$ we'll use is that they preserve symmetry, as discussed in HW1.

            % (a),(b) strongly hint that $\gen{A,B}$ is just $G$, and we directly prove that in (c).

            % The most efficient solution follows up from part (b); see the board.
    
            % Note that both $G$ and $\gen{A,B}$ preserve the mirror symmetry. You have a set of 26 mirrored pairs and $A$ permutes them.
    
            % Let $S$ be the 26 symmetric shuffles and $\text{sym}(S)$ be the set of all permutations of $S$. Since it's the set of all permutations of $S$, it's clearly isomorphic to $S_{26}$, but that's a bit unnatural. The important implication of (b) is that $\phi:G\to\text{sym}(S)$. $\text{sym}(S)$ is basically $A$. Show that $A,B$ are generated by the generators of $G$.


            To prove this, it will suffice to show that $A,B\in G$ because then, all products of them are naturally a subset of all products of the generators of $G$. Both $A,B$ obey mirror symmetry; thus, $\phi(A),\phi(B)\in S_{26}$ because of the way $\phi$ is defined in part (b). It follows since $\phi$ is surjective that we can find, using the algorithm in part (b), elements $A',B'\in G$ such that $\phi(X')=\phi(X)$ and $X'|_{[26]}\in S_{26}$. Moreover, since $A,B$ obey mirror symmetry, we can find $h,h'\in H$ such that $hA|_{[26]},h'B_{[26]}\in S_{26}$. But this implies that $hA=A'$ and $hB=B'$, i.e., that $A=h^{-1}A'\in G$ and likewise for $B$, as desired.
        \end{proof}
    \end{enumerate}
    \item Let $G$ be a finite group, and let $g,h\in G$ both have order 2. Determine the possible orders of $gh$.
    \begin{proof}
        % $|g|=2$ implies $|g^{-1}|=2$; Take $h=g^{-1}$ and then $|gh|=1$.
        % $G$ is abelian and $h\neq g^{-1}$ implies $ghgh=g^2h^2=ee=e$, so $|gh|=2$.
        % In $(\Z/2\Z)^2$, we have $(1,0),(1,1),(0,1)$ for elements of order 2. Any sum of these not of the form $g+g$ is one of the other ones of these.

        % Fix $n=|G|$ and $m<n$. Do there exist $g,h$ of order 2 such that $|gh|=m$? Don't fix the order of the group.
        % I give you $n\in\N$: Does there exist $(G,g,h)$ such that $|g|=|h|=2$ and $|gh|=n$?

        % The answer is all $n$; within any $S_m$, you can construct an element of order $n$.

        % Use the dihedral group. Take $g=s$ and $h=rs$; $gh=s^2r=r$.


        We will prove that
        \begin{equation*}
            \boxed{|gh|\text{ can be any natural number.}}
        \end{equation*}
        We divide into three cases ($|gh|=1$, $|gh|=2$, and $|gh|>2$).\par\smallskip
        Suppose we want $|gh|=1$. Consider $S_2$. Let $g=(1,2)$ and $h=g^{-1}=(1,2)$. Then clearly $|g|=|h|=2$, but $|gh|=|e|=1$.\par
        Suppose we want $|gh|=2$. Let $G$ be some abelian group containing distinct elements of order 2 (for example, take $G=(\Z/2\Z)^2$). Let $g$ be one such element and $h$ another. Then $(gh)^2=ghgh=g^2h^2=ee=e$, so $|gh|=2$, as desired.\par
        Suppose we want $|gh|=n$ for some $n>2$. Consider the dihedral group $D_{2n}$. Let $g=rs$ and $h=s$. Then $g^2=rsrs=rssr^{-1}=e$ and $h^2=s^2=e$, so $|g|=2$ and $|h|=2$. Moreover, $gh=rss=r$, so $|gh|=n$, as desired.
    \end{proof}
    \item Suppose that the map $\phi:G\to G$ given by $\phi(x)=x^2$ is a homomorphism. Prove that $G$ is abelian.
    \begin{proof}
        % We know: $\phi(x)=x^2$ and $x^2y^2=(xy)^2=\phi(xy)=\phi(x)\phi(y)$. We want to show $xy=yx$ for $xy$ arbitrary.

        % $xxyy=x^2y^2=\phi(x)\phi(y)=\phi(xy)=(xy)^2=xyxy$, apply cancellation lemma to both sides!


        To prove that $G$ is abelian, it will suffice to show that $xy=yx$ for all $x,y\in G$. Let $x,y\in G$ be arbitrary. Then
        \begin{equation*}
            xxyy = x^2y^2
            = \phi(x)\phi(y)
            = \phi(xy)
            = (xy)^2
            = xyxy
        \end{equation*}
        so by consecutive applications of the cancellation lemma, we have the desired result.
    \end{proof}
    \item Call a subgroup $H\subset G$ \textbf{cyclic} if $H=\gen{g}=\gen{g,g^{-1}}$ for some $g\in G$.
    \begin{enumerate}
        \item Prove that any cyclic subgroup $H\subset G$ is abelian.
        \begin{proof}
            Let $H$ be cyclic. Then $H=\gen{h}$. Let $x,y\in H$ be arbitrary. Then $x=h^i$ and $y=h^j$. It follows that
            \begin{equation*}
                xy = h^ih^j
                = h^{i+j}
                = h^{j+i}
                = h^jh^i
                = yx
            \end{equation*}
            as desired.
        \end{proof}
        \item Prove that any cyclic subgroup $H\subset G$ is either isomorphic to $\Z$ or to $\Z/n\Z$, and that the latter happens exactly when $h$ has finite order $n$.
        \begin{proof}
            We divide into two cases ($G$ is infinite and $G$ is finite).\par
            Let $G=\gen{g}$ be infinite. Then
            \begin{equation*}
                G = \{\dots,g^{-2},g^{-1},e,g,g^2,g^3,\dots\}
            \end{equation*}
            Now suppose for the sake of contradiction that $g^a=g^b$ for some distinct $a,b\in\Z$. Then $g^{a-b}=e$, so $|G|\leq a-b$, a contradiction. Therefore, $G=\{G^\Z\}$. In particular, we may define $\phi:\Z\to G$ by $k\mapsto g^k$. This map has the property that $a+b\mapsto g^ag^b$, i.e., $\phi(a)\phi(b)=\phi(ab)$.\par
            Let $G=\gen{g}$ be finite. Then
            \begin{equation*}
                G = \{e,g,g^2,\dots,g^{n-1}\}
            \end{equation*}
            Now suppose for the sake of contradiction that $g^a=g^b$ for some distinct $0\leq a,b<n$ with $a>b$ WLOG. Then $g^{a-b}=e$, so $|G|\leq a-b<n$, a contradiction. Therefore, we may once again define $\phi:\Z/n\Z\to G$ as above. Note that $a+b\mapsto g^{(a+b)\mod n}$. This is still a homomorphism, though.
        \end{proof}
        \item Let $G$ be any group. Prove that there is a bijection between the set of homomorphisms $\{\phi:\Z\to G\}$ and $G$ given by
        \begin{equation*}
            \phi \mapsto \phi(1)
        \end{equation*}
        (Exercise 2.3.19 of \textcite{bib:DummitFoote}.)
        \begin{proof}
            To prove that the given map is bijective, it will suffice to show that it is injective and surjective.\par
            Suppose $\phi(1)=\psi(1)$. Then if $n\in\Z$ is arbitrary,
            \begin{equation*}
                \phi(n) = \phi(\underbrace{1+\cdots+1}_{n\text{ times}})
                = \underbrace{\phi(1)\cdots\phi(1)}_{n\text{ times}}
                = \underbrace{\psi(1)\cdots\psi(1)}_{n\text{ times}}
                = \psi(\underbrace{1+\cdots+1}_{n\text{ times}})
                = \psi(n)
            \end{equation*}
            so $\phi=\psi$, as desired.\par
            Now let $g\in G$ be arbitrary. Define $\phi:Z\to G$ by
            \begin{equation*}
                \phi(n) = g^n
            \end{equation*}
            Then $\phi(1)=g$, as desired, and $\phi$ is a homomorphism since
            \begin{equation*}
                \phi(n+m) = g^{n+m}
                = g^ng^m
                = \phi(n)\phi(m)
            \end{equation*}
            as desired.
        \end{proof}
        \item Exhibit a proper subgroup of $\Q$ which is not cyclic. (Exercise 2.4.15 of \textcite{bib:DummitFoote}.)
        \begin{proof}
            Consider
            \begin{equation*}
                \boxed{H = \gen{1,\frac{1}{2},\frac{1}{4},\frac{1}{8},\frac{1}{16},\dots}}
            \end{equation*}
            with addition as the group operation. $H$ is a proper subgroup since every element of $H$ will necessarily have $2^k$ in the denominator for some $k\in\N_0$. Moreover, $H$ is not cyclic: Suppose for the sake of contradiction that $H=\gen{g}$. Then $g=n/2^k$ for some $n,k$. But then $1/2^{k+1}$, for instance, is unaccounted for.
        \end{proof}
        \item Let $G$ be a finite group. Prove that $G$ is equal to the union of its proper subgroups if and only if it is not cyclic.
        \begin{proof}
            Suppose first that $G$ is equal to the union of its proper subgroups. Each proper subgroup is generated by some proper subset of the generators of $G$. For there to be a nontrivial proper subset of the set of generators, the set of generators must have cardinality greater than or equal to 2. In particular, if the cardinality of this set is not 1, then $G$ cannot be cyclic, as desired.\par
            Now suppose that $G$ is not cyclic. Then $\gen{g}$ is a proper subgroup of $G$ for all $g\in G$; clearly, $G$ is equal to the union of all of these subgroups.
        \end{proof}
    \end{enumerate}
    \item Let $p$ be prime, and let $G=\text{GL}_2(\F_p)$ be the group of invertible $2\times 2$ matrices modulo $p$. Prove that $|G|=(p^2-1)(p^2-p)$. (See \S 1.4 of \textcite{bib:DummitFoote}.)
    \begin{proof}
        % A low-brow way is to bash out the formula with casework ($b=0$, $b\neq 0$).

        % The more principled solution is to use the determinant to detect full rankness. Count how many ways you can pick two linearly independent columns.

        % We can still assume the determinant is multiplicative. We may have to prove that nonzero determinant implies invertible.
        
        % For every choice of $a,c$, $b$ may take on $p$ values and $d$ can take on $p-1$ values (there is at most one choice of $d$ such that $b,d$ is linearly dependent on $ac$; and in fact, modulo $p$, there is exactly one choice)
        
        
        First off, note that in general, we can assume the facts of the determinant that we know over $\R^n,\C^n$ hold true independent of field.\par
        Let
        \begin{equation*}
            A =
            \begin{pmatrix}
                a & b\\
                c & d\\
            \end{pmatrix}
        \end{equation*}
        be a $2\times 2$ matrix modulo $p$. Then $a,b,c,d\in\{0,1,\dots,p-1\}$. We know that $A$ is invertible iff
        \begin{equation*}
            \det(A) = ad-bc \neq 0
        \end{equation*}
        and iff the two columns $(a,c)^T,(b,d)^T$ are linearly independent.\par
        Let's begin by counting the possible values of $(a,c)^T$. $a$ can take on $p$ values and $c$ can take on $p$ values, but in the specific case that $a=0$, we do not want to choose $c=0$ as well (because then $\det(A)=0$). Thus, there are $p^2-1$ choices of $(a,c)^T$.\par
        Now let's count the possible values of $(b,d)^T$ corresponding to each $(a,c)^T$. Let $(a,c)^T$ be arbitrary. WLOG assume that $a\neq 0$. We want $(b,d)^T$ to be linearly independent, but since linear independence is a requirement of both variables, we can let $b$ be any of the $p$ values and fix our constraint on $d$. We will do this. Suppose we have chosen $b\in\{0,\dots,p-1\}$. Then $bc\in\Z/p\Z$. Moreover, since $p$ is prime, every nonzero element of $\Z/p\Z$ is a generator of the group of order $p$. Thus, there exists exactly one $d\in\{0,\dots,p-1\}$ such that $ad=bc$, i.e., $ad-bc=0$. Therefore, for any choice of $b$, there are $p-1$ choice for $d$ that preserve linear independence.\par
        It follows that the total order of the group is
        \begin{equation*}
            |G| = (p^2-1)p(p-1)
            = (p^2-1)(p^2-p)
        \end{equation*}
        as desired.
    \end{proof}
\end{enumerate}




\end{document}