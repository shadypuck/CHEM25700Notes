\documentclass[../psets.tex]{subfiles}

\pagestyle{main}
\renewcommand{\leftmark}{Final}
\setcounter{section}{8}
\setenumerate[1]{label={\textbf{\arabic*.}}}
\setenumerate[2]{label={\arabic*.}}

\begin{document}




\section{Final Exam}
\begin{enumerate}
    \item \marginnote{12/8:}($40=20+20$ points) \textbf{For this question: no working required}\par
    Compute the orders of the following sets:
    \begin{enumerate}
        \item The centralizer of $(123)(456)(789)\in S_9$.
        \begin{proof}
            Let $g=(123)(456)(789)$, and let $\{g\}$ denote the conjugacy class of $g$ in $S_n$. We build up to applying the orbit-stabilizer theorem.\par
            Since the conjugacy class of $g$ is the set of all elements with the same cycle shape, and we have a formula for calculating the number of elements in the symmetric group of order $n$ given a certain cycle shape, we apply the formula to learn that
            \begin{equation*}
                |\{g\}| = \frac{n!}{\prod_{i=1}^kp_i^{c_i}\cdot c_i!}
                = \frac{9!}{3^3\cdot 3!}
                = 2240
            \end{equation*}
            Let $S_9\acts S_9$ by conjugation. Then $\Orb(g)=\{g\}$ and $\Stab(g)=C_{S_9}(g)$, so we have by the orbit-stabilizer theorem and the above that
            \begin{align*}
                |\Orb(g)|\cdot|\Stab(g)| &= |S_9|\\
                |C_{S_9}(g)| &= \frac{9!}{2240}\\
                \Aboxed{|C_{S_9}(g)| &= 162}
            \end{align*}
        \end{proof}
        \item The normalizer of $H=\gen{(12345)}\subset S_6$.
        \begin{proof}
            Let $X$ be the set of subgroups of $S_6$ of order 5. Every subgroup in $X$ is generated by a 5-cycle. In particular, there are
            \begin{equation*}
                \binom{6}{5}\cdot(5-1)! = 144
            \end{equation*}
            5-cycles in $S_6$. Additionally, each such subgroup contains 4 distinct 5-cycles and $e$, so
            \begin{equation*}
                |X| = \frac{144}{4} = 36
            \end{equation*}
            Let $S_6\acts X$ by conjugation. Since all 5-cycles are conjugate in $S_6$, the action is transitive. By the definition of the stabilizer and normalizer, we have that
            \begin{equation*}
                \Stab(H) = \{\sigma\in S_6\mid\sigma\cdot H=H\}
                = \{\sigma\in S_6\mid\sigma H\sigma^{-1}=H\}
                = N_{S_6}(H)
            \end{equation*}
            It follows by the orbit-stabilizer theorem that
            \begin{align*}
                |\Orb(H)|\cdot|\Stab(H)| &= |S_6|\\
                |N_{S_6}(H)| &= \frac{6!}{36}\\
                \Aboxed{|N_{S_6}(H)| &= 20}
            \end{align*}
        \end{proof}
    \end{enumerate}
    \item ($40=20+20$ points) For each of the following groups $G$, find the smallest $n$ such that $G$ is isomorphic to a subgroup of $S_n$. Justify your answers.
    \begin{enumerate}
        \item The group $G=S_5\times\Z/4\Z$.
        \begin{proof}
            We know that
            \begin{equation*}
                C_{S_n}((1,\dots,k)) \cong S_{n-k}\times\Z/k\Z
            \end{equation*}
            Thus, solving $k=4$ and $n-k=5$, we find
            \begin{equation*}
                \boxed{n = 9}
            \end{equation*}
        \end{proof}
        \item The dihedral group $D_{72}$ of order 72.
        \begin{proof}
            We know that $D_{72}$ contains an element of order $72/2=36$. The order of an element of $S_n$ is equal to the least common multiple of its constituent cycle lengths. Thus, we know that the product of a disjoint 4-cycle and 9-cycle in $S_{13}$ satisfies
            \begin{equation*}
                |(1,2,3,4)(5,6,7,8,9,10,11,12,13)| = 36
            \end{equation*}
            and that $S_{13}$ is the smallest symmetric group to contain such an element. Thus, we can map
            \begin{equation*}
                r \mapsto (1,2,3,4)(5,6,7,8,9,10,11,12,13)
            \end{equation*}
            Additionally, we can geometrically picture separate rotations of a 4-gon and a 9-gon to motivate choosing the following as a reflection element.
            \begin{equation*}
                s \mapsto (2,4)(6,13)(7,12)(8,11)(9,10)
            \end{equation*}
            Together, these elements satisfy the relations
            \begin{align*}
                r^{36} &= s^2 = e&
                rs &= sr^{-1}
            \end{align*}
            which characterize $D_{72}$. Therefore, we take
            \begin{equation*}
                \boxed{n = 13}
            \end{equation*}
        \end{proof}
    \end{enumerate}
    \item Let $A$ be a finite abelian group. Suppose that $A$ acts transitively and faithfully on a set $X$. Prove that $|A|=|X|$.
    \begin{proof}
        We approach this proof from the perspective of the Orbit-Stabilizer Theorem. According to it,
        \begin{equation*}
            |A| = |\Orb(x)|\cdot|\Stab(x)|
        \end{equation*}
        for all $x\in X$. Since $A\acts X$ is transitive, $\Orb(x)=X$, and we can further refine the above to
        \begin{equation*}
            |A| = |X|\cdot|\Stab(x)|
        \end{equation*}
        Thus, to prove that $|A|=|X|$, it will suffice to show that $|\Stab(x)|=1$ for all $x\in X$. To do so, we will show that $\Stab(x)=\Stab(y)$ for all $x,y\in X$, from which it will follow by the faithfulness of the action that
        \begin{equation*}
            \Stab(x) = \bigcap_{y\in X}\Stab(y)
            = \ker
            = \{e\}
        \end{equation*}
        for all $x\in X$, as desired. Let $x,y\in X$ be arbitrary. Since $A$ acts transitively, there exists $g\in A$ such that $g\cdot x=y$. Now suppose $h\in\Stab(y)$. Then since $A$ is abelian,
        \begin{align*}
            g\cdot x &= y\\
            &= h\cdot y\\
            &= h\cdot(g\cdot x)\\
            &= hg\cdot x\\
            &= gh\cdot x\\
            &= g\cdot(h\cdot x)
        \end{align*}
        It follows by the faithfulness of the action that $h\cdot x=x$, i.e., $h\in\Stab(x)$. Having shown that an arbitrary element of one stabilizer is necessarily in another, we know that all stabilizers are equal, and thus have the desired result.
    \end{proof}
    \item ($45=15+15+15$ points) Determine whether the following statements are true or false. Justify your answer in each case.
    \begin{enumerate}
        \item There are finitely many groups up to isomorphism which act faithfully on 5 points.
        \begin{proof}
            Let $X$ be a set with $|X|=5$, and let $G$ be a group such that $G\acts X$ faithfully. This group action induces a homomorphism $\phi:G\to S_5$ which, due to the faithfulness of the action, must have $\ker\phi=\{e\}$. In other words, $\phi$ is an injection. Therefore, by the first isomorphism theorem, there exists a bijection
            \begin{equation*}
                \tilde{\phi}:G\to\im\phi \leq S_5
            \end{equation*}
            i.e., we must have that $G$ is isomorphic to a subgroup of $S_5$. But since there are only finitely many subgroups of $S_5$ up to isomorphism, the statement is
            \begin{equation*}
                \boxed{\text{True.}}
            \end{equation*}
        \end{proof}
        \item Any finite group is a subgroup of $A_n$ for some integer $n$.
        \begin{proof}
            Let $G$ be an arbitrary finite group, and let $|G|=m$. By Cayley's theorem, we know that $G\leq S_m$. Thus, if we can prove that $S_m\leq A_n$ for some $n$, we are done. Let $n=m+2$, and define $\phi:S_m\to A_n$ by
            \begin{equation*}
                \sigma \mapsto
                \begin{cases}
                    \sigma & \sigma=\tau_1\cdots\tau_{2k}\\
                    \sigma(n+1,n+2) & \sigma=\tau_1\cdots\tau_{2k+1}
                \end{cases}
            \end{equation*}
            Obviously, $\phi$ is well-defined and outputs only even permutations. To prove that $\phi$ is a homomorphism, it will suffice to show that $\phi(\sigma\sigma')=\phi(\sigma)\phi(\sigma')$ for all $\sigma,\sigma'\in S_m$. We divide into four cases ($\sigma,\sigma'$ even, $\sigma$ even \& $\sigma'$ odd, $\sigma$ odd \& $\sigma'$ even, and $\sigma,\sigma'$ odd). For case 1, we have that $\sigma\sigma'$ is even as well, and hence
            \begin{equation*}
                \phi(\sigma\sigma') = \sigma\sigma'
                = \phi(\sigma)\phi(\sigma')
            \end{equation*}
            For case 2, we have that $\sigma\sigma'$ is odd, and hence
            \begin{equation*}
                \phi(\sigma\sigma') = \sigma\sigma'(n+1,n+2)
                = \phi(\sigma)\phi(\sigma')
            \end{equation*}
            For case 3, we have that $\sigma\sigma'$ is odd, and hence
            \begin{equation*}
                \phi(\sigma\sigma') = \sigma\sigma'(n+1,n+2)
                = \sigma(n+1,n+2)\sigma'
                = \phi(\sigma)\phi(\sigma')
            \end{equation*}
            where we have used the fact that disjoint cycles commute in the next to last step (this is why it's important to go up by $+2$). Lastly, for case 4, we have that $\sigma\sigma'$ is even, and hence
            \begin{equation*}
                \phi(\sigma\sigma') = \sigma\sigma'
                = \sigma\sigma'(n+1,n+2)^2
                = \sigma(n+1,n+2)\sigma'(n+1,n+2)
                = \phi(\sigma)\phi(\sigma')
            \end{equation*}
            Therefore, $\phi$ is a homomorphism.\par
            We can also see that $\phi$ is injective since distinct $\sigma,\sigma'$ will map to distinct outputs $\sigma,\sigma'$ (and the presence or absence of an appended disjoint cycle will not change the distinctness of the original permutations). Thus, $S_m\leq A_{m+2}$, as desired.\par
            Therefore, $G\leq A_{|G|+2}$, and we have proven that the statement is
            \begin{equation*}
                \boxed{\text{True.}}
            \end{equation*}
        \end{proof}
        \item A group of order $2688=2^7\cdot 3\cdot 7$ has a transitive action on a set $X$ with $|X|=21$.
        \begin{proof}
            % Let $G$ be a group of order 2688. By Sylow I, $G$ has a 3-Sylow subgroup and a 7-Sylow subgroup. Because of the prime factorization, these $p$-Sylows will both have order $p$ and thus be isomorphic to $\Z/p\Z$.\par
            % In particular, let $\Z/3\Z\leq G$ have generator $x$ and let $\Z/7\Z\leq G$ have generator $y$. Then $\gen{x,y}$ is a subgroup of $G$ of order 21.


            Leg $G$ be a group of order $|G|=2688$. By Sylow I, $G$ has a 2-Sylow subgroup $P$. By Lagrange's theorem,
            \begin{equation*}
                |G/P| = |G|/|P| = 21
            \end{equation*}
            We know that $G\acts G/P$ transitively for $P$ a subgroup, so take $X=G/P$. Therefore, we have proven that the statement is
            \begin{equation*}
                \boxed{\text{True.}}
            \end{equation*}
        \end{proof}
    \end{enumerate}
    \item (30 points) Let $G$ be a simple group of order 168. Determine the number of elements of $G$ of order 7.
    \begin{proof}
        Since $7$ is a prime number (and one that only appears once in the prime factorization of 168), we know that the number of elements in $G$ of order 7 will be equal to the number $n_7$ of 7-Sylows. By Sylow III, we know that $n_7\equiv 1\bmod 7$. Additionally, we have by Lagrange's theorem that $n_7\mid 168$; in particular, since $168=24\cdot 7$ and $n_7\not\equiv 0\bmod 7$, we know that $n_7\nmid 7$ so it must be that $n_7\mid 24$.\par
        The only two natural numbers that are both congruent to 1 mod 7 and divide 24 are $n_7\in\{1,8\}$. But if $n_7=1$, then by Sylow II, the sole 7-Sylow is normal in $G$, contradicting its simplicity. Therefore, we must have that
        \begin{equation*}
            \boxed{n_7 = 8}
        \end{equation*}
    \end{proof}
    \item ($20=10+10$ points) Let $n>1$ be an integer and let $G$ be a group of order $n$. The left action of $G$ on itself induces an injective map $\psi:G\to S_n$.
    \begin{enumerate}
        \item Prove that if $g\in G$ has order 2, then $n$ is even and the cycle decomposition of $\psi(g)$ consists of $n/2$ disjoint 2-cycles.
        \begin{proof}
            Since $g$ has order 2, Lagrange's theorem implies that $2\mid|G|$, so $n$ must be an even number, as desired.\par
            Let $G\acts G$ be the described left action. Let $h\in G$ be arbitrary. Since $|g|=2$ and hence $g\neq e$, we have by the Sudoku lemma that $gh\neq h$. It follows by the group action axioms that
            \begin{align*}
                g\cdot h &= gh&
                g\cdot gh &= h
            \end{align*}
            Thus, to every $h\in G$, there corresponds a unique matching element $gh$. By the faithfulness of the group action, there is no overlap between the pairs $h,gh$ and hence $G$ can be partitioned into pairs of elements $h,gh$. Moreover, we have by the above that for any $h\in G$,
            \begin{align*}
                \psi(g)(h) &= g\cdot h = gh&
                \psi(g)(gh) &= g\cdot gh = h
            \end{align*}
            so every pair $h,gh$ presents as a 2-cycle in the cycle decomposition of $\psi(g)\in S_G\cong S_n$, as desired.
        \end{proof}
        \item Prove that if $n\equiv 2\bmod 4$ and $n>2$, then $G$ is not a simple group.
        \begin{proof}
            % $G\leq S_{|G|}$.
            % Subgroup of index 2?

            % $G\acts G$ by left multiplication induces $G\hookrightarrow S_{|G|}$.
            % Lemma 11: $G\hookrightarrow A_{|G|}$ or $|G|=2$.
            % $G$ contains a $p$-Sylow for some $p\neq 2$. Therefore, $G$ contains an element of order $p$.
            
            % $G$ has a 2-Sylow of order 2. Thus, it's cycle decomposition consists of $n/2$ disjoint two cycles. But this number is strictly odd. Contradiction.


            Suppose for the sake of contradiction that $G$ is a group of order $n$. An equivalent formulation to $n\equiv 2\bmod 4$ is stating that $n=2m$ where $m$ is an odd number greater than or equal to 3. Since $|G|=2\cdot m$ where $m\nmid 2$, Sylow I implies that $G$ contains a 2-Sylow of order 2. In particular, there exists $g\in G$ of order 2 (take the nontrivial element of the 2-Sylow).\par
            Applying part (1), we learn that $\psi(g)\in S_n$ consists of $n/2=m$ (an odd number) of 2-cycles. Thus, $\psi(g)\notin A_n$.\par
            However, as a simple group with a transitive action on a set of $n\geq 2$ points, Lemma 11 asserts that either $G\hookrightarrow A_n$ or $|G|=2$. By the above, we know that $G\not\hookrightarrow A_n$, but by hypothesis, $n\neq 2$ either, a contradiction.
        \end{proof}
    \end{enumerate}
    \item Let $G$ be a subgroup of $A_8$ which is simple of order 504. Prove that the action of $G$ is 2-transitive, that is, for any pairs $\{a,b\}$ and $\{c,d\}$ of two distinct elements of $\{1,2,3,\dots,8\}$, there is an element $g$ such that
    \begin{align*}
        g(a) &= c&
        g(b) &= d
    \end{align*}
    \begin{proof}
        $504=2^3\cdot 3^2\cdot 7$.
    \end{proof}
\end{enumerate}




\end{document}