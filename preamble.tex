\usepackage[margin=1in]{geometry}
\usepackage{csquotes}
\usepackage{fancyhdr}
\usepackage{marginnote}
\usepackage{enumitem}
\usepackage{scrextend}
\usepackage[bottom]{footmisc}
\usepackage[style=apa]{biblatex}
\usepackage{xr}
\usepackage{siunitx}
\usepackage{tabu}
\usepackage{tikz,graphicx}
\usepackage{float,subcaption}
\usepackage{multirow}
\usepackage{amsmath,amssymb,amsthm,amsbsy}
\usepackage{bm,physics,nicematrix,empheq}
\usepackage[hidelinks]{hyperref}

\MakeOuterQuote{"}

\fancypagestyle{main}{
    \fancyhf{}
    \fancyhead[L]{\leftmark}
    \fancyhead[R]{MATH 25700}
    \fancyfoot[R]{Labalme\ \thepage}
}
\fancypagestyle{plain}{
    \fancyhead{}
    \renewcommand{\headrulewidth}{0pt}
}

\reversemarginpar

\setitemize[3]{label={\scriptsize$\blacksquare$}}
\setitemize[4]{label={\tikz[scale=0.06,baseline={(0,-0.14)}]{
    \draw [line width=0.3pt] (0,1) -- (1.2,0) -- (0,-1) -- (3.5,0) -- cycle;
    \fill (1.2,0) -- (0,-1) -- (3.5,0);
}}}

\addbibresource{../main.bib}
\DefineBibliographyStrings{english}{bibliography={References}}

\deffootnotemark{\textsuperscript{\textup{[}\thefootnotemark\textup{]}}}
\deffootnote[1.8em]{0em}{0em}{\textsuperscript{\thefootnote}}

\usetikzlibrary{matrix,positioning,arrows,angles}
\colorlet{rex}{red!80!black!90!orange!80}
\colorlet{rey}{red!80!black!90!orange!40}

\DeclareMathOperator{\spn}{span}
\DeclareMathOperator{\lcm}{lcm}
\DeclareMathOperator{\im}{im}
\DeclareMathOperator{\Aut}{Aut}
\DeclareMathOperator{\Inn}{Inn}
\DeclareMathOperator{\Out}{Out}
\DeclareMathOperator{\Orb}{Orb}
\DeclareMathOperator{\Stab}{Stab}
\DeclareMathOperator{\Aff}{Aff}
\DeclareMathOperator{\Fixed}{Fixed}
\setcounter{MaxMatrixCols}{20}

\theoremstyle{definition}
\newtheorem{proposition}{Proposition}
\newtheorem{theorem}[proposition]{Theorem}

\NiceMatrixOptions{cell-space-limits=1pt}

\DeclareFontFamily{U}{mathb}{}
\DeclareFontShape{U}{mathb}{m}{n}{
       <-5.5> mathb5
    <5.5-6.5> mathb6
    <6.5-7.5> mathb7
    <7.5-8.5> mathb8
    <8.5-9.5> mathb9
    <9.5-11>  mathb10
    <11->     mathb12
}{}
\DeclareSymbolFont{mathb}{U}{mathb}{m}{n}
\DeclareMathSymbol{\righttoleftarrow}{3}{mathb}{"FD}
% 
\makeatletter
\DeclareRobustCommand{\looparrow}[1]{%
  \mathrel{\mathpalette\looparrow@{#1}}%
}
\newcommand{\looparrow@}[2]{\reflectbox{$\m@th#1#2$}}
\makeatother
% 
\newcommand{\acts}{\looparrow{\righttoleftarrow}}

\newcommand{\N}{\mathbb{N}}
\newcommand{\Z}{\mathbb{Z}}
\newcommand{\Q}{\mathbb{Q}}
\newcommand{\R}{\mathbb{R}}
\newcommand{\C}{\mathbb{C}}
\newcommand{\F}{\mathbb{F}}
\newcommand{\HH}{\mathbb{H}}

\newcommand{\inp}[2]{\left\langle{#1},{#2}\right\rangle}
\newcommand{\gen}[1]{\left\langle{#1}\right\rangle}
\newcommand{\e}[1][]{\text{e}^{#1}}

\usepackage{subfiles}