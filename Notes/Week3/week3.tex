\documentclass[../notes.tex]{subfiles}

\pagestyle{main}
\renewcommand{\chaptermark}[1]{\markboth{\chaptername\ \thechapter\ (#1)}{}}
\setcounter{chapter}{2}

\begin{document}




\chapter{???}
\section{Subgroups and Generators}
\begin{itemize}
    \item \marginnote{10/10:}Defining \textbf{subgroups}.
    \begin{itemize}
        \item Let $G=(G,*)$ be a group, and let $H\subseteq G$ be a subset.
        \item What properties do we want $H$ to satisfy to consider it a "subgroup?"
        \begin{itemize}
            \item $H$ should inherit the binary operation from $G$.
            \item $H$ should be closed under multiplication using said binary operation.
            \item $H$ should be nonempty.
            \item $H$ should contain the inverses of every element --- this is automatic if $G$ is finite since the inverse of an element $g$ of order $n$ is $g^{n-1}$ and $g^{n-1}\in H$ by closure under multiplication.
            \item $H$ should also be associative; we also inherit this for free from $G$.
        \end{itemize}
    \end{itemize}
    \item Easy way to construct a subgroup.
    \begin{itemize}
        \item Let $G$ be a group, and let $x_1,x_2,\dots\in G$. We can let $H=\gen{x_1,x_2,\dots}$, i.e., $H$ is the group \textbf{generated} by $x_1,x_2,\dots$. In other words, $H$ is the set of all finite products $x_1,x_1^{-1},x_2,x_2^{-1},\dots$.
        \item This construction does give you all possible subgroups, but when you write it down, it's very hard to say what group you get.
    \end{itemize}
    \item Example: If you have $H\subset G$ a subgroup, then $H=\gen{h|_{h\in H}}$.
    \item \textbf{Cyclic} (group): A group $G$ for which there exists $g\in G$ such that $G=\gen{g}$.
    \item Examples:
    \begin{itemize}
        \item If $1<n<\infty$, then $\Z/n\Z=\gen{1}$.
        \item However, the generator isn't always unique --- $\Z/7\Z=\gen{3}$.
        \item If $G$ is generated by an element, it's also generated by its inverse. For example, $\Z=\gen{1}=\gen{-1}$.
    \end{itemize}
    \item Proposition: Let $G$ be a cyclic group. It follows that
    \begin{enumerate}
        \item If $|G|=\infty$, then $G$ is isomorphic to $\Z$;
        \item If $|G|=n<\infty$, then $G$ is isomorphic to $\Z/n\Z$.
    \end{enumerate}
    \begin{proof}
        Assertion 1: Let $G=\gen{g}$. Then
        \begin{equation*}
            G = \{\dots,g^{-2},g^{-1},e,g,g^2,g^3,\dots\}
        \end{equation*}
        Now suppose for the sake of contradiction that $g^a=g^b$ for some $a,b\in\Z$. Then $g^{a-b}=e$, so $|G|\leq a-b$, a contradiction. Therefore, $G=\{G^\Z\}$. In particular, we may define $\phi:\Z\to G$ by $k\mapsto g^k$. This map has the property that $a+b\mapsto g^ag^b$, i.e., $\phi(a)\phi(b)=\phi(ab)$\footnote{We all know that this is a \textbf{homomorphism}; Calegari just doesn't want to call it that yet.}.\par
        Assertion 2: Let $G=\gen{g}$. Then
        \begin{equation*}
            G = \{e,g,g^2,\dots,g^{n-1}\}
        \end{equation*}
        Now suppose for the sake of contradiction that $g^a=g^b$. Then $g^{a-b}=e$, so $|G|\leq a-b<n$, a contradiction. Therefore, we may once again define $\phi:\Z/n\Z\to G$ as above. Note that $a+b\mapsto g^{(a+b)\mod n}$. This is still a homomorphism, though.
    \end{proof}
    \item Claim: Any subgroup of a cyclic group is also cyclic.
    \item Example: $G=\Z$, $H=\gen{2002,686}$.
    \begin{itemize}
        \item $H=\{2002x+686y\mid x,y\in\Z\}$.
        \item To say that $H$ is cyclic is to say that it is equal to the integer multiples of some $d\in\Z$, i.e., there exists $d$ such that $G=\{zd\mid z\in\Z\}$.
        \item We can take $d=\gcd(2002,686)$.
        \item (Nonconstructive) proof: Let $d$ be the smallest positive integer in $H$. Suppose for the sake of contradiction that $md+k$ is in the group for some $1\leq k<d$. Then adding $-d$ $m$ times, we get that $k\in H$, a contradiction since we assumed $d$ was the smallest positive integer in $H$.
    \end{itemize}
    \item Let $G=\gen{x,y}$ be a group that is generated by two elements. Find a subgroup $H\subset G$ such that $H$ \emph{must} be generated by more than 2 elements.
    \begin{itemize}
        \item Let's work with $S_n=\gen{(1,2,\dots,n),(1,2)}$.
        \item The subgroup $H=\gen{(1,2),(3,4),(5,6)}$ will work.
        \begin{itemize}
            \item $H=\Z/2\Z\times\Z/2\Z\times\Z/2\Z$.
            \item Suppose $H=\gen{a,b}$. We can get $e,a,b,ab$. But because everything commutes, we can rearrange any product to $a^ib^j$ and cancel.
        \end{itemize}
    \end{itemize}
    \item When you want to answer questions like, "Is $\Z/180180\Z$ a subgroup of $S_n$ for some $n$," you need some more information on the structure of $S_n$.
    \item Group \textbf{presentations} allow us to write and describe a group really easily.
    \begin{itemize}
        \item Seems useful at first, but isn't really that useful once you see it more.
    \end{itemize}
\end{itemize}



\section{Homomorphisms}
\begin{itemize}
    \item \marginnote{10/12:}We've studied groups a lot at this point. But as with vector spaces, we don't have a complete theory of groups until we consider maps between them.
    \item Today: Homomorphisms.
    \item Let $H,G$ be groups.
    \item What qualities do we want a map of groups to have?
    \begin{itemize}
        \item Maps between vector spaces preserve linearity, so maps between groups should probably preserve the group operation.
        \item Bijection? As with linear maps, the bijective case is interesting, but we don't want to be this restrictive.
        \item In fact, that first quality is the only one we want.
    \end{itemize}
    \item \textbf{Homomorphism}: A map $\phi:H\to G$ of sets such that $\phi(x*_Hy)=\phi(x)*_G\phi(y)$.
    \item Lemma: Let $\phi:H\to G$ be a homomorphism. Then\dots
    \begin{enumerate}
        \item $\phi(e_H)=e_G$.
        \item $\phi(x^{-1})=\phi(x)^{-1}$.
    \end{enumerate}
    \begin{proof}
        Claim 1:
        \begin{align*}
            e_G\phi(x) &= \phi(x) = \phi(xe_H) = \phi(x)\phi(e_H)\\
            e_G &= \phi(e_H)
        \end{align*}
        Claim 2:
        \begin{equation*}
            e_G = \phi(e_H) = \phi(xx^{-1}) = \phi(x)\phi(x^{-1})
        \end{equation*}
    \end{proof}
    \item \textbf{Image} (of $\phi$): The subset of $G$ such that for all $h\in H$, $\phi(h)=g$. \emph{Denoted by} $\bm{\im\phi}$.
    \item \textbf{Kernel} (of $\phi$): The subset of $H$ containing all $h\in H$ such that $\phi(h)=e_G$. \emph{Denoted by} $\bm{\ker\phi}$.
    \item Lemma:
    \begin{enumerate}
        \item $\im\phi\subset G$ is a subgroup.
        \item $\ker\phi\subset H$ is a subgroup.
    \end{enumerate}
    \begin{proof}
        % Prove closed under multiplication, inverses, and nonempty. Let $g_1,g_2\in\im\phi$.\par
        % Same idea. Prove...


        Claim 1: We know that $\phi(e_H)=e_G$, so
        \begin{equation*}
            \im\phi \neq \emptyset
        \end{equation*}
        as desired. Next, let $g_1,g_2\in\im\phi$. Suppose $g_1=\phi(h_1)$ and $g_2=\phi(h_2)$. Then since $H$ is closed under multiplication as a subgroup, $h_1h_2\in H$. It follows that
        \begin{equation*}
            g_1g_2 = \phi(h_1)\phi(h_2)
            = \phi(h_1h_2)
            \in \im\phi
        \end{equation*}
        as desired. Lastly, let $g\in\im\phi$. Suppose $g=\phi(h)$. Then since $H$ is closed under inverses as a subgroup, $h^{-1}\in H$. It follows that
        \begin{equation*}
            g^{-1} = \phi(h)^{-1}
            = \phi(h^{-1})
            \in \im\phi
        \end{equation*}
        as desired.\par
        Claim 2: We know that $\phi(e_H)=e_G$, so
        \begin{equation*}
            \ker\phi \neq \emptyset
        \end{equation*}
        as desired. Next, let $g_1,g_2\in\ker\phi$. Then
        \begin{equation*}
            e_G = e_Ge_G
            = \phi(g_1)\phi(g_2)
            = \phi(g_1g_2)
        \end{equation*}
        so $g_1g_2\in\ker\phi$, as desired. Lastly, let $g\in\ker\phi$. Then
        \begin{equation*}
            e_G = \phi(e_H)
            = \phi(gg^{-1})
            = \phi(g)\phi(g^{-1})
            = e_G\phi(g^{-1})
            = \phi(g^{-1})
        \end{equation*}
    \end{proof}
    \item Examples:
    \begin{table}[h!]
        \centering
        \small
        \renewcommand{\arraystretch}{1.2}
        \begin{tabular}{c|c|c|c|c}
            $H$ & $G$ & $\phi$ & $\im\phi$ & $\ker\phi$\\
            \hline
            $H$ & $G$ & $\phi(h)=e$ & $\{e\}$ & $H$\\
            $H\leq G$ & $G$ & inclusion & $H$ & $\{e\}$\\
            $\Z$ & $\Z/n\Z$ & $k\mapsto k\mod n$ & $\Z/n\Z$ & $n\Z$\\
            $\text{O}(n)$ & $\R^*$ & $\det$ & $\{\pm 1\}$ & $\text{SO}(n)$\\
            $\text{GL}_n\R$ & $\R^*$ & $\det$ & $\R^*$ & $\text{SL}_n\R$\\
        \end{tabular}
        \caption{Examples of images and kernels.}
        \label{tab:ImKer}
    \end{table}
    \begin{itemize}
        \item The first example shows that there is always at least one homomorphism between two groups.
        \item $\R^*$ is the group of nonzero real numbers with multiplication as the group operation.
        \item The $\text{O}(n)$ example expresses the fact that $\det(AB)=\det(A)\det(B)$, i.e., that the determinant is a homomorphism.
        \begin{itemize}
            \item The kernel is $\text{SO}(n)$ since 1 is the multiplicative identity of $\R^*$ and all matrices in $\text{SO}(n)\subset\text{O}(n)$ get mapped to 1 by the determinant.
        \end{itemize}
        \item $\text{GL}_n\R$ is the set of all $n\times n$ invertible matrices over the field $\R$.
    \end{itemize}
    \item \textbf{Isomorphism}: A bijective homomorphism from $H\to G$.
    \begin{itemize}
        \item If an isomorphism exists between $H$ and $G$, we say, "$H$ is isomorphic to $G$."
    \end{itemize}
    \item Lemma: $H$ is isomorphic to $G$ implies $G$ is isomorphic to $H$.
    \begin{proof}
        $\phi:H\to G$ a bijection implies the existence of $\phi^{-1}:G\to H$. Claim: This is an isomorphism. We can formalize the notion, or just think of $\phi$ as relabeling elements of $H$ and $\phi^{-1}$ as unrelabeling them.
    \end{proof}
    \item Lemma: A homomorphism $\phi:H\to G$ is \textbf{injective} iff $\ker\phi=\{e_H\}$.
    \begin{proof}
        % We will prove that $\ker\phi\neq\{e_H\}$ implies that $\phi$ is not injective, and that $\phi$ is not injective implies $\ker\phi\neq\{e_H\}$.
        % Note that the kernel certainly contains $e_H$ by a previous lemma. Suppose $x\in\ker\phi$ and $x\neq e$. Then $\phi(x)=\phi(e_H)$ so $\phi$ is not injective. Proof by contrapositive here?\par
        % Now assume $\phi$ is not injective. Then $\phi(x)=\phi(y)$ for some $x\neq y$. Let $g=xy^{-1}$. If $g=e$, then $x=y$, a contradiction. Otherwise,
        % \begin{equation*}
        %     \phi(g) = \phi(xy^{-1}) = \phi(x)\phi(y^{-1})
        %     = \phi(x)\phi(y)^{-1}
        %     = \phi(x)\phi(x)^{-1}
        %     = e
        % \end{equation*}
        % Note that we get from the fourth to the fifth entry above using the fact that $\phi(x)=\phi(y)$.

        Suppose $\phi$ is injective. We know that $\phi(e_H)=e_G$ from a previous lemma; this implies that $e_H\in\ker\phi$. Now let $x\in\ker\phi$ be arbitrary. Then $\phi(x)=e_G=\phi(e_H)$. But since $\phi$ is injective, we have that $x=e_H$. Thus, we have proven that $e_H\in\ker\phi$, and any $x\in\ker\phi$ is equal to $e_H$; hence, we know that $\ker\phi=\{e_H\}$, as desired.\par
        Now suppose that $\ker\phi=\{e_H\}$. Let $\phi(x)=\phi(y)$. It follows that
        \begin{equation*}
            \phi(xy^{-1}) = \phi(x)\phi(y^{-1})
            = \phi(x)\phi(y)^{-1}
            = \phi(x)\phi(x)^{-1}
            = e_G
        \end{equation*}
        But this implies that
        \begin{align*}
            xy^{-1} &= e_H\\
            x &= y
        \end{align*}
        as desired.
    \end{proof}
    \item Problem: Is there a surjective homomorphism $\phi:S_5\to S_4$?
    \begin{itemize}
        \item Proposal 1: Send 5-cycles to the identity and everything else to itself.
        \item Proposal 2: "Drop 5" $(1,2)(3,4,5)\mapsto(1,2)(3,4)$.
        \begin{itemize}
            \item Counterexample: $(1,2,3,4,5)\mapsto(1,2,3,4)$.
        \end{itemize}
        \item Proposal 3: If it doesn't do something to everything, send it to $e$.
    \end{itemize}
    \item Lemma: Let $\phi:H\mapsto G$ be a homomorphism. If $|h|=n$, then $|\phi(h)|$ divides $n$, i.e., $n$ is a multiple of $|
    \phi(h)|$.
    \begin{proof}
        If $h^n=e$, then $\phi(h^n)=e=\phi(h)^n$.
    \end{proof}
    \item Equipped with this lemma, let's return to the previous problem.
    \begin{itemize}
        \item Suppose for the sake of contradiction that such a surjective homomorphism $\phi$ exists.
        \item Consider a 5-cycle $h\in S_5$; obviously, $|h|=5$.
        \item It follows by the lemma that $\phi(h)\in S_4$ has order which divides 5. But since the maximum order of an element in $S_4$ is 4, this means that $|\phi(h)|=1$, so $\phi(h)=e$.
    \end{itemize}
    \item If one 5-cycle maps to the identity, then all of their products must, too.
    \item What can map to an order 3 element in $S_4$?
    \item If $\psi(g)=(1,2,3)$, then $|g|$ is divisible by 3.
    \item In fact, no surjective map exists!
    \item In order for homomorphisms to exist, there must be some reason. If there aren't any (nontrivial ones), proving this can be easy.
    \item Now consider $S_4\mapsto S_3$.
    \begin{itemize}
        \item 4-cycles to $e$ or 2-cycles.
        \item 3-cycles to 3-cycles.
    \end{itemize}
    \item Idea: $S_4\cong\text{Cu}\cong S_3$.
    \begin{itemize}
        \item 3 pairs of opposite faces and 4 diagonals.
    \end{itemize}
\end{itemize}




\end{document}