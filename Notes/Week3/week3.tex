\documentclass[../notes.tex]{subfiles}

\pagestyle{main}
\renewcommand{\chaptermark}[1]{\markboth{\chaptername\ \thechapter\ (#1)}{}}
\setcounter{chapter}{2}

\begin{document}




\chapter{???}
\section{Introduction to Subgroups and Generators}
\begin{itemize}
    \item \marginnote{10/10:}Defining \textbf{subgroups}.
    \begin{itemize}
        \item Let $G=(G,*)$ be a group, and let $H\subseteq G$ be a subset.
        \item What properties do we want $H$ to satisfy to consider it a "subgroup?"
        \begin{itemize}
            \item $H$ should inherit the binary operation from $G$.
            \item $H$ should be closed under multiplication using said binary operation.
            \item $H$ should be nonempty.
            \item $H$ should contain the inverses of every element --- this is automatic if $G$ is finite since the inverse of an element $g$ of order $n$ is $g^{n-1}$ and $g^{n-1}\in H$ by closure under multiplication.
            \item $H$ should also be associative; we also inherit this for free from $G$.
        \end{itemize}
    \end{itemize}
    \item Easy way to construct a subgroup.
    \begin{itemize}
        \item Let $G$ be a group, and let $x_1,x_2,\dots\in G$. We can let $H=\gen{x_1,x_2,\dots}$, i.e., $H$ is the group \textbf{generated} by $x_1,x_2,\dots$. In other words, $H$ is the set of all finite products $x_1,x_1^{-1},x_2,x_2^{-1},\dots$.
        \item This construction does give you all possible subgroups, but when you write it down, it's very hard to say what group you get.
    \end{itemize}
    \item Example: If you have $H\subset G$ a subgroup, then $H=\gen{h|_{h\in H}}$.
    \item \textbf{Cyclic} (group): A group $G$ for which there exists $g\in G$ such that $G=\gen{g}$.
    \item Examples:
    \begin{itemize}
        \item If $1<n<\infty$, then $\Z/n\Z=\gen{1}$.
        \item However, the generator isn't always unique --- $\Z/7\Z=\gen{3}$.
        \item If $G$ is generated by an element, it's also generated by its inverse. For example, $\Z=\gen{1}=\gen{-1}$.
    \end{itemize}
    \item Proposition: Let $G$ be a cyclic group. It follows that
    \begin{enumerate}
        \item If $|G|=\infty$, then $G$ is isomorphic to $\Z$;
        \item If $|G|=n<\infty$, then $G$ is isomorphic to $\Z/n\Z$.
    \end{enumerate}
    \begin{proof}
        Assertion 1: Let $G=\gen{g}$. Then
        \begin{equation*}
            G = \{\dots,g^{-2},g^{-1},e,g,g^2,g^3,\dots\}
        \end{equation*}
        Now suppose for the sake of contradiction that $g^a=g^b$ for some $a,b\in\Z$. Then $g^{a-b}=e$, so $|G|\leq a-b$, a contradiction. Therefore, $G=\{G^\Z\}$. In particular, we may define $\phi:\Z\to G$ by $k\mapsto g^k$. This map has the property that $a+b\mapsto g^ag^b$, i.e., $\phi(a)\phi(b)=\phi(ab)$\footnote{We all know that this is a \textbf{homomorphism}; Calegari just doesn't want to call it that yet.}.\par
        Assertion 2: Let $G=\gen{g}$. Then
        \begin{equation*}
            G = \{e,g,g^2,\dots,g^{n-1}\}
        \end{equation*}
        Now suppose for the sake of contradiction that $g^a=g^b$. Then $g^{a-b}=e$, so $|G|\leq a-b<n$, a contradiction. Therefore, we may once again define $\phi:\Z/n\Z\to G$ as above. Note that $a+b\mapsto g^{(a+b)\mod n}$. This is still a homomorphism, though.
    \end{proof}
    \item Claim: Any subgroup of a cyclic group is also cyclic.
    \item Example: $G=\Z$, $H=\gen{2002,686}$.
    \begin{itemize}
        \item $H=\{2002x+686y\mid x,y\in\Z\}$.
        \item To say that $H$ is cyclic is to say that it is equal to the integer multiples of some $d\in\Z$, i.e., there exists $d$ such that $G=\{zd\mid z\in\Z\}$.
        \item We can take $d=\gcd(2002,686)$.
        \item (Nonconstructive) proof: Let $d$ be the smallest positive integer in $H$. Suppose for the sake of contradiction that $md+k$ is in the group for some $1\leq k<d$. Then adding $-d$ $m$ times, we get that $k\in H$, a contradiction since we assumed $d$ was the smallest positive integer in $H$.
    \end{itemize}
    \item Let $G=\gen{x,y}$ be a group that is generated by two elements. Find a subgroup $H\subset G$ such that $H$ \emph{must} be generated by more than 2 elements.
    \begin{itemize}
        \item Let's work with $S_n=\gen{(1,2,\dots,n),(1,2)}$.
        \item The subgroup $H=\gen{(1,2),(3,4),(5,6)}$ will work.
        \begin{itemize}
            \item $H=\Z/2\Z\times\Z/2\Z\times\Z/2\Z$.
            \item Suppose $H=\gen{a,b}$. We can get $e,a,b,ab$. But because everything commutes, we can rearrange any product to $a^ib^j$ and cancel.
        \end{itemize}
    \end{itemize}
    \item When you want to answer questions like, "Is $\Z/180180\Z$ a subgroup of $S_n$ for some $n$," you need some more information on the structure of $S_n$.
    \item Group \textbf{presentations} allow us to write and describe a group really easily.
    \begin{itemize}
        \item Seems useful at first, but isn't really that useful once you see it more.
    \end{itemize}
\end{itemize}




\end{document}