\documentclass[../notes.tex]{subfiles}

\pagestyle{main}
\renewcommand{\chaptermark}[1]{\markboth{\chaptername\ \thechapter\ (#1)}{}}
\setcounter{chapter}{8}

\begin{document}




\chapter{Simple Groups}
\section{Simple Groups I}
\begin{itemize}
    \item \marginnote{11/28:}\textbf{Simple} (group): A group $G$ for which the only normal subgroups of $G$ are $G$ and itself, i.e., $H\triangleleft G$ implies $H=G$ or $H=\{e\}$.
    \begin{itemize}
        \item Simple does not mean "easy" but means "cannot be broken up into pieces."
        \item By analogy, think of atoms as indivisible.
        \item If you have $G$ and $H\triangleleft G$, you get $H$ and $G/H$, and you can think of $G$ as being made up of $H,G/H$. Together, these two groups convey quite a bit of information about $G$.
        \item Warning: $H,G/H$ do \emph{not} determine $G$; just a lot of information about it.
        \begin{itemize}
            \item Example: Let $H=\Z/2\Z$ and $G/H=\Z/2\Z$. Then we could have $G=(\Z/2\Z)^2$ or $G=\Z/4\Z$.
        \end{itemize}
    \end{itemize}
    \item Idea: If you want to classify all finite groups, you might start with all finite simple groups, knowing that finite nonsimple groups can in some way be described by its simple quotients and subgroups.
    \item Problem I (Classification): Classify all finite simple groups.
    \begin{itemize}
        \item A bit like understanding all prime numbers first in order to understand all composite numbers.
    \end{itemize}
    \item Problem II (Extension problem): Given $A,B$, understand all $G$ such that $A\triangleleft G$ and $G/A\cong B$.
    \begin{itemize}
        \item We can build back up to $G$ with $A\times B$ or other ways.
        \item We'll talk about this one less than problem I.
    \end{itemize}
    \item Examples:
    \begin{itemize}
        \item Let $p$ be prime. Then $G=\Z/p\Z$ is a simple group.
        \begin{itemize}
            \item Follows directly form Lagrange's theorem.
            \item It's even stronger than simple; the only \emph{subgroups} (let alone normal subgroups) of $\Z/p\Z$ are $\Z/p\Z$ and $\{e\}$.
        \end{itemize}
        \item Let $n\geq 5$ and let $G=A_n$. Then $A_n$ is a simple group.
        \begin{itemize}
            \item More interesting and intricate. Has many subgroups but the only \emph{normal} ones are itself and the trivial one.
            \item Note that $A_3$ is also simple, but cyclic and abelian as well, so it got classified with the above.
        \end{itemize}
    \end{itemize}
    \item What does it mean to classify simple groups?
    \begin{itemize}
        \item Start by asking what are the simple groups of some particular order.
        \item Start with groups of a certain factorization or those with small order.
        \item In this series of lectures, we'll focus on groups of small order. Can we understand for order below 100, 200, or 300?
        \item What's important: Less the classification, more the application of techniques we've used. Fancier techniques needed for bigger $n$.
    \end{itemize}
    \item Things in math aren't always hard because the technique is hard; they're hard because knowing what technique to use is hard. This is the challenge here.
    \item The prime factorization of the order says a lot about the group and allows us to make various conclusions.
    \item Theorem: Let $p$ be prime. Suppose that $|G|=p^n$. Then if $G$ is simple, we have $|G|=p$ and $G\cong\Z/p\Z$.
    \begin{proof}
        If $G$ is a $p$-group, then $Z(G)\neq\{e\}$.\par
        Case 1: $G=Z(G)$, so $G$ is abelian. Therefore, let $g\in G$ have order $p$ and let $H=\gen{g}\neq\{e\}$. If $G$ is simple, then $H=G$ and therefore $|G|=p$.\par
        Case 2: $G$ is not abelian. Take $H=Z(G)\neq G$. We know that $Z(G)\triangleleft G$, so $G$ is not simple, a contradition.
    \end{proof}
    \item Takeaway: $|G|=2$ is simple, but $|G|=4,8,16,\dots$ are all not simple.
    \item The general $p^iq^j$ case is very sophisticated, so we'll start simple.
    \item Lemma 1: Let $|G|=pq$ where $p,q$ are distinct primes. Then $G$ is not simple.
    \begin{proof}
        Suppose for the sake of contradiction that $G$ is simple with $|G|=pq$. WLOG, let $p>q$.
        WTS: One of the Sylow subgroups will be normal. The normal one is the one with greater order (motivation: $D_{2n}$; it's often useful to consider the $p$-Sylow subgroups for the largest $p$).
        What do we know? From the Sylow theorems, $n_p\cong 1\mod p$ and $n_p\mid q$ (we know that $|N|=|G|/n_p=pq/n_p$, but since $n_p\equiv 1\mod p$, $n_p\nmid p$, so it must be that $n_p\mid q$). $p>q$ implies $q\not\equiv 1\pmod p$. Thus, $n_p=1$. This is a contradiction: If there's only 1 $p$-Sylow subgroup, then that $p$-Sylow is normal (because all $p$-Sylows are conjugate, so one $p$-Sylow means its in its own conjugacy class).
    \end{proof}
    \item We use a contradiction argument every time.
    \item Lemma 2: Let $|G|=pqr$. Then $G$ is not simple.
    \begin{proof}
        Strategy (again): Apply Sylow theorems and get information.\par
        WLOG, let $p>q>r$. We have that $n_p\equiv 1\pmod p$ and $n_p\mid qr$. $n_p\in\{1,q,r,qr\}$. If $n_p\equiv 1\pmod p$, the $p$-Sylow is normal in $G$, a contradiction. $q,r\not\equiv 1\mod p$, so we eliminate those cases, too. One case left: $qr$. We thus deduce that $n_p=qr$.\par
        New technique: Because of these congruences, the number of $p$-Sylows cannot be really small (congruence obstructions). But we also know that it can't be too big. If there are that many elements of order $p$, we will crowd out the elements of other orders. We know that $n_q\equiv 1\mod q$, and $n_q\mid pr$. $n_q=1$ gives a contradiction. $n_q\not\equiv r$ because $n>r$. Thus, $n_q\in\{p,pr\}$. Doing the same thing for $n_r$, we get three possibilities: $p,q,pr$. Next step: Count elements. How many elements of order $p$ are in $G$?\par
        Proposition: If $p\mid |G|$ exactly, then any two distinct $p$-Sylows have only trivial intersection. The number of $g\in G$ of order $p$ is equal to $n_p(p-1)$.\par
        Because $p$ exactly divides $p$, each $p$-Sylow is a subgroup of order $p$, but their intersection is a subgroup and thus has to divide the order (Lagrange's theorem). Thus, the order of the intersection is either 1 or $p$. Thus, all elements of order $p$ lie in trivially intersecting $p$-Sylows. We count $p-1$ elements of order $p$ for each $p$-Sylow ($p$ minus the identity).\par
        Thus, since $p\,||\,|G|$ in this case, we know that the number of $g\in G$ with $|g|=p$ is $n_p(p-1)=qr(p-1)$. The number of $g\in G$ with $|g|=q$ is $n_q(q-1)\geq p(q-1)$. The number of $g\in G$ with $|g|=r$ is $n_r(r-1)\geq q(r-1)$. Counting the number of elements and the identity, we get
        \begin{equation*}
            qr(p-1)+p(q-1)+q(r-1)+1 = qrp+pq-p-q+1
            = pqr+(p-1)(q-1)
            > pqr
            = |G|
        \end{equation*}
        a contradiction.
    \end{proof}
    \item This has to fail eventually, though --- we know $A_5$ is simple for instance, and it has prime factorization $2^2\cdot 3\cdot 5$, so $pqr^2$ can be simple.
    \item Thus, we now turn to other types of factorizations.
    \item Thus, consider variations of the two primes case.
    \item First, new technique.
    \item Lemma 3: Let $G\subset S_4$ is simple. Then $|G|=2,3$.
    \begin{proof}
        If we have a homomorphism from a simple group to any other group, it is either trivial or injective (our group doesn't break up; it either injects fully or disappears completely). We know that $\ker\phi\triangleleft G$, so if $G$ is simple, either $\ker\phi=\{e\}$ (injective) or $\ker\phi=G$ (trivial).\par
        We know that $A_4\triangleleft S_4\twoheadrightarrow S_4/A_4\cong\Z/2\Z$. Now let $G\triangleleft S_4$. We can apply a homomorphism to get a map from $G\to S_4/A_4$. It follows by the above claim that the homomorphism is either trivial or injective.\par
        Let $\Gamma$ be a group with $A\triangleleft\Gamma$. Let $\Gamma/A=B$. If $G\hookrightarrow\Gamma$ is simple, then either $G\hookrightarrow A$ or $G\hookrightarrow\Gamma/A=B$. Proof: We have $G\to\Gamma\to\Gamma/A=B$. Case 1: $G$ injects into $B$, so we get the latter claim. Case 2: the map is trivial, so everything in $G$ maps to the identity in $B=\Gamma/A$. Then $G\leq A$. So if we know how to divide our group up, we can make something of the pieces.\par
        Returning to our example, we have $A_r\triangleleft S_4$, $S_4/A_4=\Z/2\Z$, so $G\hookrightarrow A_4$, $G\hookrightarrow\Z/2\Z$. In the latter case, it has order 2. We have $K\triangleleft A_4$ and $A_4/K\equiv\Z/3\Z$. So either $G\leq K$ or $G\leq\Z/3\Z$. The first one implies since $K=(\Z/2\Z)^2$ that $G\triangleleft\Z/2\Z$.
    \end{proof}
    \item Groups of order 2,3 are trivially simple, so it's kind of meaningless, but doesn't matter; the lemma still holds.
    \item We narrowed in on the case of $S_4$ in order to prove our next theorem.
    \item Lemma 4 (No small actions): Let $G$ be a simple group, and suppose $G\acts X$ transitively, where $|X|=2,3,4$. Then $|G|=2,3$.
    \begin{proof}
        Given a transitive action, we get a homomorphism $G\to S_X$. Transitivity and $|X|\geq 2$ implies the homomorphism is nontrivial. But since $G$ is simple, $G\hookrightarrow S_X$. But since $|X|\leq 4$, this means that $G\hookrightarrow S_4$. We now use Lemma 3. This means that $|G|=2,3$.\par
        See lemma 6 for the kind of group action we are talking about??
    \end{proof}
    \item Corollary: If $p\mid |G|$, $p$ is prime, $G$ is simple, $|G|\neq 2,3,p$, then $n_p\neq 1,2,3,4$.
    \item Next time: At the same time these videos release, there will be a blog post with the statements of these lemmas and maybe some words on them.
\end{itemize}



\section{Office Hours (Abhijit)}
\begin{itemize}
    \item What do we need to know about the affine group of order $p$, as discussed in Lecture 8.1?
    \item Friday's office hours will be the last one unless Frank changes something. If the Twitch stream actually happens, Abhijit isn't sure what day it would be. Frank is currently traveling.
    \item HW8 2c requires 2d.
    \item Abhijit will email me more info on the $A_5$ question that he "beat a tactical retreat from."
\end{itemize}



\section{Simple Groups II}
\begin{itemize}
    \item \marginnote{11/30:}Lemma 5: Assume $|G|=2p^n,3p^n,4p^n$. $p$ is a prime, and for $mp^n$, $m\neq p$. Then $G$ is not simple.
    \begin{proof}
        We again look at $p$-Sylows and how many are there. $n_p$ must divide the order of the group and not divide $p$ in these cases. Therefore, $n_p=1,2,3,4$, so as in Lemma 4, we have a very small set for $G$ to act on. Thus, by Lemma 4, $G$ is not simple.
    \end{proof}
    \item Beefing this up a bit.
    \item Lemma 6: Assume $|G|=5p^n$. Then $G$ is not simple.
    \begin{proof}
        Only interesting case: $n_p=5$. In this case, we get an action of $G$ on 5 points, namely the transitive action of $G$ on the 5 $p$-Sylows by conjugation. Thus, we get an injective map $G\to S_5$, so $G\leq 5$ and has order $5p^n$.
        Additionally, since $n_p=5\equiv 1\pmod p$, we know that $p=2$.
        What else can we say? We now look at $n_5$. We know from Sylow III that $n_5\equiv 1\mod 5$ and $n_5\mid(p^n=2^n)$, so $2^n\geq 16$ (since $16\equiv 1\mod 5$ and $16$ divides a power of 2). Thus, $|G|$ divides 16, but $|S_5|=120$ which is not divisible by 16, a contradiction.
    \end{proof}
    \item See again the procedure of "assume it's simple; derive a contradiction."
    \item Lemma 7: Assume $|G|=6p^n$. Then $G$ is not simple.
    \begin{proof}
        We know that $n_p\mid 6$, so $n_p=1,2,3,6$. Lemma 4: $n_p=6$. Sylow III: $n_p\equiv 1\mod p$. Thus, $p=5$. $G$ acts on 6 $p$-Sylows, so we get an injective map from $G\hookrightarrow S_6$. How many powers of 5 can divide $|S_6|$? Only $5^1$, so we must have $n=1$. Thus, $|G|=6\cdot 5=30=5\cdot 3\cdot 2$. Applying Lemma 2 (3 distinct primes) finishes us off.
    \end{proof}
    \item Lemma 8: Assume $|G|=8p,9p$. Then $G$ is not simple.
    \begin{proof}
        Look at $n_p$. $n_p=1,2,4,8$ in the first case; $n_p=1,3,9$ in the second case. Lemma 4: Rule out $1,2,4$ and $1,3$. Thus, $n_p=8$ in the first case and $n_p=9$ in the second case.\par
        First case: $n_p=8$ and $n_p\equiv 1\mod p$. We must have $p=7$. Thus $|G|=8\cdot 7=56$. That's all we can get from the $p$-Sylow; now let's look at the 2-Sylow ($2^3=8$). $n_2=1,7$. We apply the $pqr$ too-many-elements style again. Number of elements of order 7 is $n_7(7-1)=8\cdot 6=48$. We have a group of 56 elements and 48 of them have order 7. So what's left? There are 8 elements left. But we know that the 2-Sylow has order 8, so let $P=G\setminus\{g\in G\mid |g|=7\}$. Then $|P|=8$. This implies that there is only one 2-Sylow, so $n_2=1$, meaning that the 2-Sylow is normal and giving us a contradiction.
    \end{proof}
    \item Again, we only have a contradiction because we are assuming $G$ is simple. If we don't assume $G$ is simple, we have a perfectly valid mathematical derivation of the properties of a group of order $8p$.
    \item Example: Let $G=A_4$. Then $|G|=12=3\cdot 2^2$, so $n_3=1,2,4$ and $n_3\equiv 1\mod 3$. The 3-Sylow in $A_4$ is not normal, so $n_p\neq 1$. $n_p\neq 2$ because $2\not\equiv 1\mod 3$. Thus, $n_3=4$ and therefore the number of elements of order 3 is $n_3(3-1)=4\cdot 2=8$. Thus, there are 12 elements $A_4$ minus 8 elements of order 3, so there are 4 elements left. These 4 elements compose the 2-Sylow, meaning that the 2-Sylow is normal. And here (where we're not assuming $G$ is simple), that's fine!
    \item So far: Continuing to build up and rule out groups of certain (mostly prime) factorizations.
    \item This will not classify all simple groups, but if we start taking $n$ to be small, we know the factorizations of small numbers tend to have a small number of prime factors. So can we classify all groups of small order? That's our task now.
    \item Lemma 9: If $|G|=84,126,140,156,175,189,198,200$, then $G$ is not simple.
    \begin{proof}
        $84=7\cdot 3\cdot 2^2$. Sylow III: $n_7\equiv 1\mod 7$ and $n_7\mid 12$, so $n_7=1$.\par
        $126=7\cdot 18$. Sylow III: $n_7\equiv 1\mod 7$, $n_7\mid 18$ implies $n_7=1$.\par
        $140=7\cdot 20$. Sylow III: $n_7\equiv 1\mod 7$, $n_7\mid 20$ implies $n_7=1$.\par
        $189=7\cdot 27$. Sylow III: $n_7\equiv 1\mod 7$, $n_7\mid 27$ implies $n_7=1$.\par\smallskip
        $176=11\cdot 16$. Sylow III: $n_{11}\equiv 1\mod 11$, $n_{11}\mid 16$ implies $n_{11}=1$.\par
        $176=11\cdot 18$. Sylow III: $n_{11}\equiv 1\mod 11$, $n_{11}\mid 18$ implies $n_{11}=1$.\par\smallskip
        $175=5^2\cdot 7$. Sylow III: $n_5\equiv 1\mod 5$, $n_5\mid 7$ implies $n_5=1$.\par
        $200=5^2\cdot 8$. Sylow III: $n_5\equiv 1\mod 5$, $n_5\mid 8$ implies $n_5=1$.\par\smallskip
        $156=13\cdot 12$. Sylow III: $n_{13}\equiv 1\mod 13$, $n_{13}\mid 12$ implies $n_{13}=1$.
    \end{proof}
    \item A bit messy and \emph{ad hoc}, but this covers the simple groups of certain orders we haven't covered so far.
    \begin{itemize}
        \item If you do a random case for a small number, often it will be a bit like this.
    \end{itemize}
    \item For a bunch of small numbers, we immediately get without any work from the Sylow theorems a normal $p$-Sylow.
    \begin{itemize}
        \item We actually get these conclusions for any groups of these orders??
    \end{itemize}
    \item This way of thinking lends itself to you generating your own problem; you basically choose a random number and look for a prime with $n_p=1$.
    \item Two more exceptional cases.
    \item Lemma 10: If $|G|=132$, then $G$ is not simple.
    \begin{proof}
        $132=11\cdot 12$, so $n_{11}=1,12$. If 1, we're done. If very large, we get that contradiction. The number of elements of order 11 is $n_{11}(11-1)=12\cdot 10=120$, so only 12 elements left. We now consider other primes. $11\cdot 12=11\cdot 3\cdot 2^2$. $n_3\equiv 1\mod 3$, $n_3\mid 44$. Thus, $n_3=1,2,4,11,22,44$. If 1, we're done. 2,11,44 can't happen because of the congruence law. If $n_3=4$, apply Lemma 4. Thus, $n_3=22$, so the number of elements of order 3 is $n_3(3-1)=22\cdot 2=44$. $120+44>132$, so we win.
    \end{proof}
    \item At this point, we've considered a bunch of easy general and special cases. At this point, let's look at the numbers under 200 we can rule out.
    \begin{itemize}
        \item Calegari writes out all numbers from 1-200.
        \item For the prime numbers, there is a simple group of that order.
        \item Powers of primes, there is no simple group, so we can cross those off ($4=2^2,8=2^3,9=3^2,16=2^4,25=5^2,27=3^3,32=2^5,\dots$).
        \item Products of two primes, there is no simple group ($6=2\cdot 3,10=2\cdot 5,15=3\cdot 5,\dots$).
        \item Products of three distinct primes, there is no simple group ($30=2\cdot 3\cdot 5,42=2\cdot 3\cdot 7,\dots$).
        \item Everything that's $2p^n,3p^n,4p^n$, there is no simple group ($12=3\cdot 2^2,18=2\cdot 3^2,\dots$).
        \item Everything that's $5p^n$, there is no simple group ($20=5\cdot 2^2,40=5\cdot 2^3,\dots$).
        \item Everything that's $6p^n$, there is no simple group (just $150=6\cdot 5^2$). Only got one number with Lemma 7 :(
        \item Everything that's $8p,9p$, there is no simple group ($56=8\cdot 7,63=9\cdot 7,88=8\cdot 11,99=9\cdot 11,\dots$).
        \item Exceptional numbers done by hand: $84,126,132,140,156,175,189,198,200$.
    \end{itemize}
    \item There are no simple groups of order 1 by definition, so the trivial group is not a simple group much the same way 1 is not a prime number.
    \item At this point, we have to acknowledge that this list left out numbers, so we have to see what's left and then work with it.
    \item First numbers up: 60 (there does happen to be a simple group of this order here --- $A_5$), 72, 90.
    \begin{itemize}
        \item Thus, if we have a simple group of order less than 100, it is of prime order or of order 60, 72, or 90 (note that we can still eliminate 72 and 90, but we haven't investigated them yet, so it's fair to include them here; not the most restrictive theorem, but a valid one all the same).
    \end{itemize}
    \item Next numbers up: 112, 120, 144, 168, 180.
    \begin{itemize}
        \item Calegari really does have quite a fast mind as he's doing prime factorizations by memory.
    \end{itemize}
    \item Thus, using the lemmas, we've proven the following: A simple group of order at most 200 either has prime order, or order 60, 72, 90, 112, 120, 144, 168, or 180.
\end{itemize}



\section{Simple Groups III}
\begin{itemize}
    \item \marginnote{12/2:}Proposition: Let $G$ be a simple group of order $|G|\leq 200$. Then either $|G|$ is prime (in which case we know there is a unique simple group of said order) or $|G|\in\{72,112,60,90,120,144,180,168\}$.
    \begin{itemize}
        \item These numbers are not in increasing order; they are more or less in order of increasing difficulty.
    \end{itemize}
    \item Case: $|G|=72=3^2\cdot 2^3$.
    \begin{proof}
        $n_3\equiv 1\mod 3$ and $n_3\mid 8$, so either $n_3=1,4$. $n_3=1$ implies normal 3-Sylow; $n_3=4$ implies invoke Lemma 4. Therefore, $G$ is not simple.
    \end{proof}
    \item Reminder: If $G$ is simple, then any homomorphism from it to another group must be either trivial or injective.
    \item Lemma 11: Let $G$ be a simple group with a transitive action on a set of $n\geq 2$ points. Then $G\hookrightarrow A_n$ or $|G|=2$.
    \begin{proof}
        The action induces a homomorphism $G\to S_n$ which is nontrivial. Therefore, $G\hookrightarrow S_n$. Now we want to upgrade this map and restrict the range. Now suppose that the image is not in $A_n$. Then the composite map $G\to S_n\to S_n/A_n\cong\Z/2\Z$ is nontrivial. If $g\in A_n$, then $g\mapsto eA_n=A_n$. If $g\notin A_n$, then $g\mapsto gA_n$. This nontrivial map into $\Z/2\Z$ implies, since $G$ is simple, that $G\hookrightarrow\Z/2\Z$, so $|G|=2$.
    \end{proof}
    \item Lemma 12: If $|G|=112$, then $G$ is not simple.
    \begin{proof}
        $112=7\cdot 2^4$. We can deduce that $n_7\equiv 1\mod 7$ and $n_7\mid 16$, so $n_7=1,8$. If $n_7=1$, we win, so let's consider $n_7=8$. 8 is pretty big, but not too big. It's in this annoying intermediate range. We deduce from this that there are $n_7(7-1)=8\cdot 6=48$ elements of order 7, which doesn't help much. Here, it's better to look at $n_2=1,7$. If $n_2=1$, we win, so consider $n_2=7$. Then by Lemma 11, $G\hookrightarrow A_7$. $|G|=112$ and $|A_7|=2520$. But $|G|\nmid |A_7|$, contradicting Lagrange's Theorem.
    \end{proof}
    \item Next up: 60, 90, 120, which we will consider all at once because they're all a bit similar. One preliminary lemma first, though.
    \item Lemma 13: Let $n\geq 5$ and $H\leq A_n$ of index $d>1$. Then $d\geq n$ and if $d=n$, then $H\cong A_{n-1}$. In other words, there are no small-index subgroups and the smallest one is isomorphic to $A_{n-1}$.
    \begin{proof}
        $A_n$ acts transitively on the cosets $A_n/H$ (by left multiplication??), inducing a nontrivial map from $A_n\to S_d=S_{\{\text{cosets}\}}$. Since $A_n$ is simple for $n\geq 5$, we get that $A_n\hookrightarrow A_d=A_{\{\text{cosets}\}}$ (by Lemma 11). The injection implies that $d\geq n$. Now assume that $d=n$. Then we have an isomorphism from $A_n$ to $A_{\{\text{cosets}\}}$. What is $H\leq A_n$? How does $H$ act on $A_{\{\text{cosets}\}}$? Well $H$ stabilizes the coset $H$, so $H\hookrightarrow\Stab(H)$. But the stabilizer of a point in the symmetric group is isomorphic to the symmetric group of one smaller size, and the same holds true in the alternating group, so $\Stab(H)\cong A_{n-1}$. $[A_n:H]=n$ by hypothesis, so $|H|=(n!/2)/n=(n-1)!/2$. But since this is also $|A_{n-1}|$, we have an injection from $H$ into a group of the same order, meaning that we have a bijection. In particular, $H\cong A_{n-1}$, as desired.
    \end{proof}
    \item Comments on Lemma 13: ...
    \item Lemma 14: If $G$ is simple and $|G|\in\{60,90,120\}$, then $G\cong A_5$. In particular, we rule out 90,120 and know that 60 must be the last number standing.
    \begin{proof}
        $60=5\cdot 12$, $90=5\cdot 18$, $120=5\cdot 24$. $n_5\equiv 1\mod 5$ and $n_5\mid 12,18,24$ implies $n_5=1,6$ (we can confirm that 11,16,21 are not factors). If $n_5=1$, we're done, so assume $n_5=6$. $G$ acts on the 6 5-Sylow subgroups transitively, giving us $G\hookrightarrow A_6$ by Lemma 13. $[A_6:G]=6,4,3$ for 60,90,120. But by Lemma 13, $d\geq 6$, so contradiction for 90,120, and the latter part of the Lemma proves that $G\cong A_5$.
    \end{proof}
    \item We have our first genuine simple group of nonprime order at this point, and we know that that group is unique and equal to $A_5$.
    \item Lemma 15: There is no simple group of order 144.
    \begin{proof}
        Many of the arguments we've used thus far will play in, but in new and unexpected ways. We have $144=2^43^2$. $n_3\equiv 1\mod 3$ and $n_3\mid 16$, so $n_3=1,4,16$. If $n_3=1$, we're done (normal 3-Sylow). If $n_3=4$, we apply Lemma 4. Thus, $n_3=16$. Number of elements of order 3: Issue --- we no longer know that the intersections of the $p$-Sylows are trivial since $p^2=9$; we could have a subgroup of order 3 as the intersection.\par
        Case 1: $P\neq Q$ are 3-Sylows, WTS: $P\cap Q=\{e\}$. Number of elements of order 3 or 9 is $n_3\cdot 8=16\cdot 8=128$. This leaves $144-128=16$ elements. But since $G$ has a 2-Sylow $P_2$ of order 16, so $G=P_2\cup\{|g|=3,9\}$. Thus, $n_2=1$, so we have a normal subgroup, which is a contradiction.\par
        Case 2: There exist 3-Sylows $P,Q$ such that $C=P\cap Q$ has $|C|=3$. We can still gain some advantage since it's $3^2$, not $3^n$. Remark: If $|P|=9$, then $P$ is abelian. Recall that $p$-groups of order $p^2$ are abelian. What other constructions do we have for subgroups besides Sylows? Consider $N=N_G(C)$, where $C=P\cap Q$. $C\leq P$ and $C\leq Q$. But since $P$ is abelian, then $P\subset N_G(C)$. Lagrange: $9\mid N\mid 144$, so $N=18,36,72,144$. So we need 4 contradictions to finish this off. If $|N|=144$, then $N=G$. But since $C\triangleleft N$, this means that $C\triangleleft G$, i.e., $G$ has a normal subgroup and is not simple, a contradiction. If $N=72,36$, then $[G:N]=2,4$. Thus, $G$ acts transitively on a set of size 2,4, so Lemma 4 eliminates these. Only possibility: $|N|=18$. We know that $Q\subset N$ and $P\subset N$. By applying the Sylow theorems to $N$, we get that $n_3\equiv 1\mod 3$, $n_3\mid 2$, so $n_3=1$, but $N$ contains 2 distinct $p$-Sylows, a contradiction.
    \end{proof}
    \item The case of 180 is quite similar.
    \item Lemma 16: There is no simple group of order 180.
    \begin{proof}
        $180=5\cdot 3^2\cdot 2^2$. $n_5\equiv 1\mod 5$, so $n_5=1,6,36$. $n_5\neq 1$. If $n_5=6$, then $G\hookrightarrow A_6$ and we get index 2, which is a contradiction by Lemma 13 (same contradiction as with 90,120). If $n_5=36$, then we get a number of elements of order 5 equal to $n_5(5-1)=144$, leaving $180-144=36$ elements left to contain the 2-Sylows and 3-Sylows. But we get the same annoyance about whether or not they overlap.\par
        $n_3\mid 20$ and $n_3\equiv 1\mod 3$ yields $n_3=1,4,20$. $n_3=1$ is normal, 4 invokes lemma 4, so $n_3=20$.\par
        Case 1: $P\cap Q=\{e\}$. Then the number of elements of order 3 or 9 is $n_3(9-1)=160$. Contradiction (too many elements).\par
        Case 2: $P\cap Q=C$, $|C|=3$. Take $N=N_G(C)$. So once again, we have $9\mid|N|\mid 180$ and $|N|>9$. Thus, $|N|=18,36,45,90,180$. $|N|=180$ gives us $C\triangleleft G$ again. $|N|=90,45$ yields $[G:N]=2,4$ again. This leaves us with 18,36. We can't have 2 3-Sylows, so $|N|=36$. Thus, $[G:N]=5$. This induces $G\hookrightarrow A_5$, but $G$ is too big; $G\nleq A_5$, a contradiction.
    \end{proof}
    \item Summary of what we've proven so far.
    \item Theorem: Let $G$ be a simple group of order at most 200. Then either\dots
    \begin{enumerate}
        \item $|G|=\text{prime}$;
        \item $G\cong A_5$;
        \item $|G|=168$ and $G\cong\text{GL}_3(\F^2)\cong\text{PSL}_2(\F_7)$.
    \end{enumerate}
    \item This is the limit of what we'll do; Calegari doesn't think it's worth the hour it'd take to do a full analysis of 168.
    \item A word on 168, though.
    \begin{itemize}
        \item $168=2^3\cdot 3\cdot 7$.
        \item Deduce that $n_7=8$. $n_2=1,3,7,21$ so $n_2=7,21$. $n_3=1,2,4,7,8,14,28,56$ so $n_3=7,28$ by Lemma 4 and the $1\mod 3$ rule. But none of the remaining numbers are super easy to work with; we need normalizers and such.
        \item Start with $n_7$ to learn that $G\hookrightarrow A_8$. We also know that $n_7=[G:N]$, so $|N|=21$. Let $P$ be a 7-Sylow. Up to conjugation, we may as well take $P=\gen{(1,2,3,4,5,6,7)}\subset A_6$. Additionally, $N\subset N_{A_8}(\gen{(1,2,3,4,5,6,7)})$ which has order 21.
        \item More and more elaborate facts lead you to write down parts of the group in terms of elements of $A_8$.
        \item The contradictions and computations we eventually get do get messy eventually.
        \item The reason for this is because there \emph{does} exist a simple group of order 168: We know that $\text{SL}_2(\F_7)$ has an action on 8 points. The quotient $G=\text{PSL}_2(\F_7)$ is a simple group of order 168.
        \item There's another group $|\text{GL}_3(\F^2)|=(2^3-1)(2^3-2)(2^3-2^2)=168$. These two groups are isomorphic (this is completely opaque from the description, but it is true; there is an exotic isomorphism; it's like the story between the cube group and $S_4$?? This is a recurring theme in finite group theory).
        \item Indeed, these two groups are the only one of 168.
    \end{itemize}
    \item On the blog, Calegari will continue from 200 all the way up to 300.
    \item Particularly bad cases include more and more powers of 2 and 3.
    \item Theorem (Burnside's $pq$ theorem): If $|G|=p^nq^m$, then $G$ is not simple.
    \begin{itemize}
        \item In other words, once you remove the cyclic group, any simple group has at least 3 prime factors dividing its factorization.
        \item This theorem is beyond the scope of this course, but it could come up in an intro grad course on representations of groups.
    \end{itemize}
    \item In our course, we've really been studying actions of $G$ on finite sets, i.e., maps to symmetric groups.
    \begin{itemize}
        \item Another natural thing we can do is consider the actions on vector spaces (specifically the basis of a vector space). This corresponds to a map from $G\to\text{GL}_n(V)$. This is a rich theory and allows us to write out all the possible representations. Proving Burnside's $pq$ theorem involves all of this plus algebraic number theory.
    \end{itemize}
    \item Burnside's $pq$ theorem came to prominence around 1900, so about 120 years back.
    \item Theorem (Feit-Thompson): If $G$ is a simple group and $|G|$ is odd, then $|G|=\text{prime}$.
    \begin{itemize}
        \item From the 1960s.
        \item This is the start of modern group theory.
        \item It follows by the Sylow theorems that our group has an element of order 2. From here, we can start to think about the classification of all finite simple groups.
    \end{itemize}
    \item Thus, not only do simple groups have to have at least 3 prime factors, they also have to have even order.
    \item The classification of finite simple groups was achieved 20-30 years after Feit-Thompson.
    \begin{itemize}
        \item Thus, we now really understand all finite simple order.
        \item This means that we can describe them by various lists that we know how to construct.
        \item Examples: Cyclic groups of prime order, $A_n$ ($n\geq 5$), $\text{PSL}_n(\F_p)$ ($n\geq 3$ or $n=2$ and $p\geq 5$).
        \item Calegari goes over several more groups of Lie type. There are also 26 sporadic group types that don't really fit in any other category. One class called the Mathieu group. Simplest weird group: $M_{11}$ of order $11\cdot 10\cdot 9\cdot 8=7920$. There's also $M_{12},M_{23},M_{24}$. There are 26 of these leading up to a group $M$ called the \textbf{monster group} (of order $\approx\num{8e53}$).
        \item We have to be careful in saying that the monster group is scary just because it's so big; indeed, $S_{52}$ is bigger. But the latter is understandable by its action on a relatively small set. $S_{52}$ acts on a vector space of dimension 52. If you try to let $M$ act on a vector space, the dimension is 196883.
        \item Something better than this type of description is infinite groups. The most interesting infinite groups arise when you have topological considerations as well (e.g., continuity). This is why $S_\infty$ is not the most interesting. We have orthogonal and unitary groups (these arise in physics often as infinite symmetry groups). Pure topology and spaces and symmetries are potential applications, too. We don't consider these in this course because these topics are so intertwined.
    \end{itemize}
    \item Next quarter is ring theory, and last quarter is Galois theory.
    \item In group theory, you want to understand all finite groups. In ring theory, you don't want to classify rings; rather, you want to consider a bunch of simple rings that come up all the time, and you want to understand those very well.
\end{itemize}




\end{document}