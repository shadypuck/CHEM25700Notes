\documentclass[../notes.tex]{subfiles}

\pagestyle{main}
\renewcommand{\chaptermark}[1]{\markboth{\chaptername\ \thechapter\ (#1)}{}}
\setcounter{chapter}{8}

\begin{document}




\chapter{Simple Groups}
\section{Simple Groups I}
\begin{itemize}
    \item \marginnote{11/28:}\textbf{Simple} (group): A group $G$ for which the only normal subgroups of $G$ are $G$ and itself, i.e., $H\triangleleft G$ implies $H=G$ or $H=\{e\}$.
    \begin{itemize}
        \item Simple does not mean "easy" but means "cannot be broken up into pieces."
        \item By analogy, think of atoms as indivisible.
        \item If you have $G$ and $H\triangleleft G$, you get $H$ and $G/H$, and you can think of $G$ as being made up of $H,G/H$. Together, these two groups convey quite a bit of information about $G$.
        \item Warning: $H,G/H$ do \emph{not} determine $G$; just a lot of information about it.
        \begin{itemize}
            \item Example: Let $H=\Z/2\Z$ and $G/H=\Z/2\Z$. Then we could have $G=(\Z/2\Z)^2$ or $G=\Z/4\Z$.
        \end{itemize}
    \end{itemize}
    \item Idea: If you want to classify all finite groups, you might start with all finite simple groups, knowing that finite nonsimple groups can in some way be described by its simple quotients and subgroups.
    \item Problem I (Classification): Classify all finite simple groups.
    \begin{itemize}
        \item A bit like understanding all prime numbers first in order to understand all composite numbers.
    \end{itemize}
    \item Problem II (Extension problem): Given $A,B$, understand all $G$ such that $A\triangleleft G$ and $G/A\cong B$.
    \begin{itemize}
        \item We can build back up to $G$ with $A\times B$ or other ways.
        \item We'll talk about this one less than problem I.
    \end{itemize}
    \item Examples:
    \begin{itemize}
        \item Let $p$ be prime. Then $G=\Z/p\Z$ is a simple group.
        \begin{itemize}
            \item Follows directly form Lagrange's theorem.
            \item It's even stronger than simple; the only \emph{subgroups} (let alone normal subgroups) of $\Z/p\Z$ are $\Z/p\Z$ and $\{e\}$.
        \end{itemize}
        \item Let $n\geq 5$ and let $G=A_n$. Then $A_n$ is a simple group.
        \begin{itemize}
            \item More interesting and intricate. Has many subgroups but the only \emph{normal} ones are itself and the trivial one.
            \item Note that $A_3$ is also simple, but cyclic and abelian as well, so it got classified with the above.
        \end{itemize}
    \end{itemize}
    \item What does it mean to classify simple groups?
    \begin{itemize}
        \item Start by asking what are the simple groups of some particular order.
        \item Start with groups of a certain factorization or those with small order.
        \item In this series of lectures, we'll focus on groups of small order. Can we understand for order below 100, 200, or 300?
        \item What's important: Less the classification, more the application of techniques we've used. Fancier techniques needed for bigger $n$.
    \end{itemize}
    \item Things in math aren't always hard because the technique is hard; they're hard because knowing what technique to use is hard. This is the challenge here.
    \item The prime factorization of the order says a lot about the group and allows us to make various conclusions.
    \item Theorem: Let $p$ be prime. Suppose that $|G|=p^n$. Then if $G$ is simple, we have $|G|=p$ and $G\cong\Z/p\Z$.
    \begin{proof}
        If $G$ is a $p$-group, then $Z(G)\neq\{e\}$.\par
        Case 1: $G=Z(G)$, so $G$ is abelian. Therefore, let $g\in G$ have order $p$ and let $H=\gen{g}\neq\{e\}$. If $G$ is simple, then $H=G$ and therefore $|G|=p$.\par
        Case 2: $G$ is not abelian. Take $H=Z(G)\neq G$. We know that $Z(G)\triangleleft G$, so $G$ is not simple, a contradition.
    \end{proof}
    \item Takeaway: $|G|=2$ is simple, but $|G|=4,8,16,\dots$ are all not simple.
    \item The general $p^iq^j$ case is very sophisticated, so we'll start simple.
    \item Lemma 1: Let $|G|=pq$ where $p,q$ are distinct primes. Then $G$ is not simple.
    \begin{proof}
        Suppose for the sake of contradiction that $G$ is simple with $|G|=pq$. WLOG, let $p>q$.
        WTS: One of the Sylow subgroups will be normal. The normal one is the one with greater order (motivation: $D_{2n}$; it's often useful to consider the $p$-Sylow subgroups for the largest $p$).
        What do we know? From the Sylow theorems, $n_p\cong 1\mod p$ and $n_p\mid q$ (we know that $|N|=|G|/n_p=pq/n_p$, but since $n_p\equiv 1\mod p$, $n_p\nmid p$, so it must be that $n_p\mid q$). $p>q$ implies $q\not\equiv 1\pmod p$. Thus, $n_p=1$. This is a contradiction: If there's only 1 $p$-Sylow subgroup, then that $p$-Sylow is normal (because all $p$-Sylows are conjugate, so one $p$-Sylow means its in its own conjugacy class).
    \end{proof}
    \item We use a contradiction argument every time.
    \item Lemma 2: Let $|G|=pqr$. Then $G$ is not simple.
    \begin{proof}
        Strategy (again): Apply Sylow theorems and get information.\par
        WLOG, let $p>q>r$. We have that $n_p\equiv 1\pmod p$ and $n_p\mid qr$. $n_p\in\{1,q,r,qr\}$. If $n_p\equiv 1\pmod p$, the $p$-Sylow is normal in $G$, a contradiction. $q,r\not\equiv 1\mod p$, so we eliminate those cases, too. One case left: $qr$. We thus deduce that $n_p=qr$.\par
        New technique: Because of these congruences, the number of $p$-Sylows cannot be really small (congruence obstructions). But we also know that it can't be too big. If there are that many elements of order $p$, we will crowd out the elements of other orders. We know that $n_q\equiv 1\mod q$, and $n_q\mid pr$. $n_q=1$ gives a contradiction. $n_q\not\equiv r$ because $n>r$. Thus, $n_q\in\{p,pr\}$. Doing the same thing for $n_r$, we get three possibilities: $p,q,pr$. Next step: Count elements. How many elements of order $p$ are in $G$?\par
        Proposition: If $p\mid |G|$ exactly, then any two distinct $p$-Sylows have only trivial intersection. The number of $g\in G$ of order $p$ is equal to $n_p(p-1)$.\par
        Because $p$ exactly divides $p$, each $p$-Sylow is a subgroup of order $p$, but their intersection is a subgroup and thus has to divide the order (Lagrange's theorem). Thus, the order of the intersection is either 1 or $p$. Thus, all elements of order $p$ lie in trivially intersecting $p$-Sylows. We count $p-1$ elements of order $p$ for each $p$-Sylow ($p$ minus the identity).\par
        Thus, since $p\,||\,|G|$ in this case, we know that the number of $g\in G$ with $|g|=p$ is $n_p(p-1)=qr(p-1)$. The number of $g\in G$ with $|g|=q$ is $n_q(q-1)\geq p(q-1)$. The number of $g\in G$ with $|g|=r$ is $n_r(r-1)\geq q(r-1)$. Counting the number of elements and the identity, we get
        \begin{equation*}
            qr(p-1)+p(q-1)+q(r-1)+1 = qrp+pq-p-q+1
            = pqr+(p-1)(q-1)
            > pqr
            = |G|
        \end{equation*}
        a contradiction.
    \end{proof}
    \item This has to fail eventually, though --- we know $A_5$ is simple for instance, and it has prime factorization $2^2\cdot 3\cdot 5$, so $pqr^2$ can be simple.
    \item Thus, we now turn to other types of factorizations.
    \item Thus, consider variations of the two primes case.
    \item First, new technique.
    \item Lemma 3: Let $G\subset S_4$ is simple. Then $|G|=2,3$.
    \begin{proof}
        If we have a homomorphism from a simple group to any other group, it is either trivial or injective (our group doesn't break up; it either injects fully or disappears completely). We know that $\ker\phi\triangleleft G$, so if $G$ is simple, either $\ker\phi=\{e\}$ (injective) or $\ker\phi=G$ (trivial).\par
        We know that $A_4\triangleleft S_4\twoheadrightarrow S_4/A_4\cong\Z/2\Z$. Now let $G\triangleleft S_4$. We can apply a homomorphism to get a map from $G\to S_4/A_4$. It follows by the above claim that the homomorphism is either trivial or injective.\par
        Let $\Gamma$ be a group with $A\triangleleft\Gamma$. Let $\Gamma/A=B$. If $G\hookrightarrow\Gamma$ is simple, then either $G\hookrightarrow A$ or $G\hookrightarrow\Gamma/A=B$. Proof: We have $G\to\Gamma\to\Gamma/A=B$. Case 1: $G$ injects into $B$, so we get the latter claim. Case 2: the map is trivial, so everything in $G$ maps to the identity in $B=\Gamma/A$. Then $G\leq A$. So if we know how to divide our group up, we can make something of the pieces.\par
        Returning to our example, we have $A_r\triangleleft S_4$, $S_4/A_4=\Z/2\Z$, so $G\hookrightarrow A_4$, $G\hookrightarrow\Z/2\Z$. In the latter case, it has order 2. We have $K\triangleleft A_4$ and $A_4/K\equiv\Z/3\Z$. So either $G\leq K$ or $G\leq\Z/3\Z$. The first one implies since $K=(\Z/2\Z)^2$ that $G\triangleleft\Z/2\Z$.
    \end{proof}
    \item Groups of order 2,3 are trivially simple, so it's kind of meaningless, but doesn't matter; the lemma still holds.
    \item We narrowed in on the case of $S_4$ in order to prove our next theorem.
    \item Lemma 4 (No small actions): Let $G$ be a simple group, and suppose $G\acts X$ transitively, where $|X|=2,3,4$. Then $|G|=2,3$.
    \begin{proof}
        Given a transitive action, we get a homomorphism $G\to S_X$. Transitivity and $|X|\geq 2$ implies the homomorphism is nontrivial. But since $G$ is simple, $G\hookrightarrow S_X$. But since $|X|\leq 4$, this means that $G\hookrightarrow S_4$. We now use Lemma 3. This means that $|G|=2,3$.\par
        See lemma 6 for the kind of group action we are talking about??
    \end{proof}
    \item Corollary: If $p\mid |G|$, $p$ is prime, $G$ is simple, $|G|\neq 2,3,p$, then $n_p\neq 1,2,3,4$.
    \item Next time: At the same time these videos release, there will be a blog post with the statements of these lemmas and maybe some words on them.
\end{itemize}




\end{document}