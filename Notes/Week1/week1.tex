\documentclass[../notes.tex]{subfiles}

\pagestyle{main}
\renewcommand{\chaptermark}[1]{\markboth{\chaptername\ \thechapter\ (#1)}{}}

\begin{document}




\chapter{Motivating Group Theory}
\section{Groups as Shuffles}
\begin{itemize}
    \item \marginnote{9/28:}Office hours will be pooled between the two sections.
    \begin{itemize}
        \item Our section's TA is Abhijit Mudigonda (\href{mailto:abjihitm@uchicago.edu}{abjihitm@uchicago.edu}). His office hours will always be in JCL 267\footnote{JCL is John Crerar Library.}. The times are\dots
        \begin{itemize}
            \item Monday: 12:30-2:00 (OH).
            \item Wednesday: 1:30-2:30 (PS).
            \item Thursday: 12:30-2:00 (OH).
        \end{itemize}
        \item The other section's TA is Ray Li (\href{mailto:rayli@uchicago.edu}{rayli@uchicago.edu}). His office hours will always be in Eck 17\footnote{Eckhart basement.}. The times are\dots
        \begin{itemize}
            \item Tuesday: 5:00-7:00 (OH).
            \item Thursday: 4:00-5:00 (OH).
            \item Thursday: 5:00-6:00 (PS).
        \end{itemize}
    \end{itemize}
    \item Textbook: Abstract Algebra. Download the PDF from LibGen.
    \item Weekly HW due on Monday at the beginning of class. Submit online or in person. There is a webpage w/ all the homeworks, but don't do them all at once because they're subject to change.
    \item Notes on math and math pedagogy.
    \begin{itemize}
        \item There's a tendency to say here's an object, here's its properties, etc.
        \item But this is not historically accurate or motivated. Calegari really gets it! Math is motivated by abstracting examples.
        \item Let's not just define a group, but start with an example. This week, we will give examples of groups. In later weeks, we will establish the axiomatic framework that is really only there to understand these examples.
        \item Don't stare at the page blankly waiting for inspiration when doing homework; think of examples first and test out your intuition on them to actually understand what the question means.
        \item There are some hard problems; work with each other, but acknowledge our collaborators.
        \item In-class midterm; final will be take-home. Calegari doesn't like timed exams.
    \end{itemize}
    \item Today's example: Shuffling.
    \begin{itemize}
        \item 52 cards; can be shuffled.
        \item Number of shuffles:
        \begin{equation*}
            |\text{shuffles}| = 52! \approx \num{8e67}
        \end{equation*}
        \item Properties of shuffles.
        \begin{itemize}
            \item \textbf{Distinguished shuffle}: $e$, the identity shuffle, where you do nothing.
            \item Shuffle once; shuffle again. The composition of two shuffles is another shuffle.
            \item If you repeat the \emph{same} shuffle enough times, the cards will come back to the same order.
            \begin{itemize}
                \item Let $\sigma$ be a shuffle, and $n\in\N$. Does there exist $n$ such that
                \begin{equation*}
                    \sigma^n = \underbrace{\sigma\circ\cdots\circ\sigma}_{n\text{ times}} = e
                \end{equation*}
                \item Proving this: By the piegeonhole principle, if you have $\sigma^1,\dots,\sigma^{52!+1}$, then we have repeats $a,b$ with $52!+1\geq a>b\geq 1$ such that $\sigma^a=\sigma^b$. This statement is weaker than we want, though.
                \item We need more tools. A shuffle is a bijection/permutation. Thus, for every $\sigma$, there exists $\sigma^{-1}$. This allows us to do this:
                \begin{align*}
                    \sigma^a &= \sigma^b\\
                    \sigma^{-b}\circ\sigma^a &= \sigma^{-b}\circ\sigma^b\\
                    \sigma^{a-b} &= e
                \end{align*}
                \item This implies a bound! We get that $n\leq 52!$, so $a-b\leq 52!$.
            \end{itemize}
        \end{itemize}
        \item Define two shuffles: $A$ and $B$.
        \begin{itemize}
            \item $A$ splits the deck into two halves (cards 1-26 and 27-52) and stacks (from the top down) the first card off of the 1-26 pile, then the first card off of the 27-52 pile, then the second card off of the 1-26 pile, then the second card off of the 27-52 pile, etc. The final order is $1,27,2,28,\dots,26,52$.
            \item $B$ does the same thing as $A$ but with the first card off of the 27-52 pile. The final order is $27,1,28,2,\dots,52,26$.
        \end{itemize}
        \item Computation shows that $A^8=e$ and $B^{52}=e$.
        \begin{itemize}
            \item For $A$, $2\to 3\to 5\to 9\to 17\to 33\to 14\to 27\to 2$.
            \item For $B$, we can do the same thing but obviously the cycle is much longer.
        \end{itemize}
        \item We shouldn't necessarily have an intuition for this right now, but in doing more examples, Calegari certainly believes we can develop it.
        \item First HW problem (due Friday). Can, just by using combinations of $A$ and $B$, we generate any possible shuffle? Hint: Develop your intuition on a smaller value of 52.
    \end{itemize}
    \item I really like Calegari. Very nice, relatable, not demeaning.
    \item \textbf{Binary operation} (on $G$): A map from $G\times G\to G$.
    \item \textbf{Group}: A mathematical object consisting of a set $G$ and a binary operation $*$ on $G$ satisfying the following properties.
    \begin{enumerate}
        \item There exists an identity element $e\in G$ such that $e\times g=g\times e=g$ for all $g\in G$.
        \item For any $g\in G$, there exists $h\in G$ such that $h*g=g*h=e$.
        \item (Associativity) For any $g_1,g_2,g_3\in G$, $g_1*(g_2*g_3)=(g_1*g_2)*g_3$.
    \end{enumerate}
    \emph{Denoted by} $\bm{(G,*)}$.
    \item In the cards example, the elements of $G$ are the shuffles and $*$ is the composition operation between two shuffles.
    \item Aside on shuffles: For bijections, $h(g(x))=x$ implies $g(h(y))=y$.
    \begin{itemize}
        \item Proof: Let $x=h(y)$ --- we can do this since $h$ is a bijection. Then since $h(g(h(y)))=h(y)$ and $h$ is injective, $g(h(y))=y$. This works for all $y$.
    \end{itemize}
    \item The set of shuffles, together with composition, does form a group.
    \item Theorem: If $G$ is a group such that $|G|<\infty$, then any $g\in G$ has finite \textbf{order}, i.e., there exists $n$ such that $g^n=e$.
    \item Lemma:
    \begin{enumerate}
        \item The identity $e$ is unique.
        \begin{itemize}
            \item Let $e_1,e_2$ be identities. Then
            \begin{equation*}
                e_1 = e_1*e_2 = e_2
            \end{equation*}
        \end{itemize}
        \item Inverses are unique.
        \begin{itemize}
            \item Let $h,h'$ be inverses of $g$. Then
            \begin{equation*}
                h = e*h
                = (h'*g)*h
                = h'*(g*h)
                = h'*e
                = h'
            \end{equation*}
        \end{itemize}
    \end{enumerate}
    \item Proving examples is easier, but these aren't that hard.
    \item If you understand everything about $S_5$, you'll understand everything about this course.
\end{itemize}



\section{Blog Post: What is Group Theory About?}
\emph{From \textcite{bib:Calegari}.}
\begin{itemize}
    \item \marginnote{10/24:}Many great ideas on how mathematics should be taught.
    \begin{itemize}
        \item Example: "A natural mathematical question is: How do we quantify this symmetry? This is unlike mathematical questions you might be used to, like `what is $2^{10}$' or `what is $\int_{-1}^1\sqrt{1-x^2}$,' but it is actually reflective of what real mathematicians do."
    \end{itemize}
    \item A terrific intuitive motivation for group theory.
    \item Using symmetry to put constraints on physical laws.
    \begin{itemize}
        \item Suppose we want to understand the gravitational attraction between two particles $\mathbf{x},\mathbf{y}$ in $\R^3$.
        \item The gravitation pull $F$ has a certain magnitude which depends on the positions of the two particles, i.e.,
        \begin{equation*}
            F(\mathbf{x},\mathbf{y}) = F(x_1,x_2,x_3,y_1,y_2,y_3)
        \end{equation*}
        \item However, "our conception of this force is that it shouldn't depend on how we are labeling the coordinates," i.e., the force should be invariant under translation. Thus,
        \begin{equation*}
            F(\mathbf{x},\mathbf{y}) = F(\mathbf{x}-\mathbf{y},\bm{0})
            = F(x_1-y_1,x_2-y_2,x_3-y_3,0,0,0)
        \end{equation*}
        \item Going further, the force should not depend on the direction, but only the distance between the two particles. Thus,
        \begin{equation*}
            F(\mathbf{x},\mathbf{y}) = H(|\mathbf{x}-\mathbf{y}|)
        \end{equation*}
        \item Thus, we see that through only consideration of symmetry, we have put strong constraints on how the force of gravity may behave.
    \end{itemize}
    \item Review of the riffle shuffle stuff from class.
\end{itemize}



\section{Blog Post: The Axioms of a Group}
\emph{From \textcite{bib:Calegari}.}
\begin{itemize}
    \item Relevant section from \textcite{bib:DummitFoote}: 1.1.
    \item Review of the content covered in class, plus the cancellation lemma (from the 10/3 lecture).
    \item Note that the cancellation lemma for groups is stronger than the one for the real numbers.
    \begin{itemize}
        \item In $\R$, we have $xy=xz$ implies $y=z$ or $x=0$, but the latter case doesn't happen in groups.
        \item One consequence of this observation is that $\R$ under numerical multiplication does not form a group.
    \end{itemize}
\end{itemize}



\section{The Cube Group}
\begin{itemize}
    \item \marginnote{9/30:}Can't download \verb|.tex| file for homework?
    \begin{itemize}
        \item Calegari will check it.
    \end{itemize}
    \item Detail on the homework?
    \begin{itemize}
        \item Up to your level of confidence in what you think is clear to be true.
        \item The problem is not about doing linear algebra; it's about finding some facts about linearly algebraic objects.
        \item Concentrate on the new geometry of the situation.
        \item Project confidence to the grader that you know what you're doing.
    \end{itemize}
    \item The symmetries of the cube.
    \begin{itemize}
        \item Rotational symmetries.
        \item Rigid transformation.
        \item Preserves lengths, angles, and lines.
        \item A map from the cube to itself, i.e., $\phi:\text{cube}\to\text{cube}$.
        \item No scaling allowed.
        \item Reflectional symmetries are \emph{not} going to be allowed for today; we will insist that the orientation is also preserved for now.
        \item We want the set of all rotations and compositions of rotations. (Are compositions of rotations also rotations? We'll answer later. Yes they are.)
    \end{itemize}
    \item Symmetries should be composable: If you compose two symmetries, you should get a third one.
    \begin{itemize}
        \item In other words, we want the symmetries to form a group.
    \end{itemize}
    \item We want to fix the center of the cube at the origin. Thus, a symmetry can be a linear map $M:\R^3\to\R^3$.
    \begin{itemize}
        \item We want it to preserve angles, i.e., orthogonality. Thus, we should assert $MM^T=I$.
        \item We also want it to preserve orientation. Then we should have $\det(M)=1$.
    \end{itemize}
    \item \textbf{Cu}: The cube group.
    \item Does the permutation of faces determine $M$?
    \begin{itemize}
        \item Yes.
        \item Furthermore, if we know where $e_1,e_2$ go, then the fact that orientation and orthogonality are preserved implies that we know where $e_3$ goes. Thus, $M$ is determined by two (adjacent) faces.
    \end{itemize}
    \item An upper bound on $|\text{Cu}|$.
    \begin{itemize}
        \item Send $e_1$ to one of 6 faces and send $e_2$ to one of the 5 remaining faces (so $|\text{Cu}|\leq 6\cdot 5=30$).
        \item Send $e_1$ to one of 6 faces and send $e_2$ to one of the four remaining \emph{adjacent} faces (so $|\text{Cu}|\leq 6\cdot 4=24$).
        \item And, in fact, $|\text{Cu}|=24$.
    \end{itemize}
    \item Moreover, since the rotations of the cube are determined by permutations of the faces, we can map $\text{Cu}\hookrightarrow S_6$. Additionally, composing any permutations of the faces is the same as composing any permutations of $S_6$, i.e., $\phi$ is an \textbf{injective homomorphism} to a \textbf{subgroup} of $S_6$.
    \item We can also think about permuting the vertices.
    \begin{itemize}
        \item 3 vertices (chosen correctly) form a basis of $\R^3$.
        \item Thus, since there are 8 vertices, we have another map from $\text{Cu}\hookrightarrow S_8$.
        \item Since we can map the first vertex to any of eight and the second to only one of three adjacent vertices, the order is $8\cdot 3=24$\footnote{We have gotten the order a different way. Deep connection to prime factorization? Edges would be $2\cdot 12$!}.
    \end{itemize}
    \item We now have both Cu and $S_4$ with order 24. Are they isomorphic?
    \begin{itemize}
        \item One characteristic of a cube that numbers four are its four diagonals. This induces a function from $\text{Cu}\to S_4$. We now just need to prove it's bijective.
        \item Let $v_1,v_2,v_3,v_4$ be the vertexes of one face. Then $-v_1,\dots,-v_4$ are the vertexes of the opposite face, and the line from each $v_i$ to $-v_i$ is a diagonal of the cube. To prove that the function is bijective, we will show that different elements of Cu map to different elements of $S_4$.
        \item Let $A$ and $B$ be actions on the cube group such that
        \begin{align*}
            Bv_1 &= \pm Av_1\\
            Bv_2 &= \pm Av_2\\
            Bv_3 &= \pm Av_3\\
            Bv_4 &= \pm Av_4
        \end{align*}
        \item Taking $C=A^{-1}B$ means that
        \begin{align*}
            Cv_1 &= \pm v_1\\
            Cv_2 &= \pm v_2\\
            Cv_3 &= \pm v_3\\
            Cv_4 &= \pm v_4
        \end{align*}
        \item If $Cv_1=v_1$, it implies that $Cv_i=v_i$ for $i=2,3,4$.
        \item Thus, $A$ and $B$ are distinct?
    \end{itemize}
\end{itemize}



\section{Blog Post: Symmetries of the Cube}
\emph{From \textcite{bib:Calegari}.}
\begin{itemize}
    \item \marginnote{10/24:}Motivating the definition of symmetries of the cube.
    \item From our intuition, symmetries can be of the form\dots
    \begin{enumerate}
        \item Rotations in lines passing through the origin.
        \item Linear maps which preserve distances, angles, and orientation.
    \end{enumerate}
    \item Claim: These two sets are the same.
    \begin{proof}
        From HW1, the first set is $\text{SO}(3)$. Thus, we need only prove that the second set is exactly $\text{SO}(3)$. To begin, we will concentrate only on distances and angles.\par
        Let $\inp{x}{y}$ denote the Euclidean inner product of $x,y\in\R^3$. Since $\inp{x}{y}=|x||y|\cos(\theta)$, the set of all linear maps that preserve distances and angles is equal to the set of all linear maps $M$ satisfying
        \begin{equation*}
            \inp{Mx}{My} = \inp{x}{y}
        \end{equation*}
        Since $\inp{x}{y}=x^Ty$, this means we want
        \begin{align*}
            (Mx)^T(My) &= x^Ty\\
            x^TM^TMy &= x^Ty\\
            M^TM &= I
        \end{align*}
        It follows that we want all $M\in\text{O}(3)$.\par
        Without going into detail, the orientation issue further restricts us to $\text{SO}(3)$.
    \end{proof}
    \item Calegari subtly gives away the $f(n)+f(53-n)=53$ from the homework in this post!
\end{itemize}



\section{Chapter 0: Preliminaries}
\emph{From \textcite{bib:DummitFoote}.}
\subsection*{Basics}
\begin{itemize}
    \item \marginnote{12/4:}Frequently used notation defined at the beginning of the textbook.
    \item Know the basics of set theory.
    \begin{itemize}
        \item Definitions given for: \textbf{Subset}, \textbf{Cartesian product}, $\pmb{\Z}$\footnote{The German word for numbers is "Zahlen."}, $\pmb{\Q}$, $\pmb{\R}$, and $\pmb{\C}$.
        \item Additional important terms are defined below.
    \end{itemize}
    \item \textbf{Order} (of a set $A$): The number of elements of $A$ (provided $A$ is a finite set). \emph{Also known as} \textbf{cardinality}. \emph{Denoted by} $\bm{|A|}$.
    \item $\Z^+,\Q^+,\R^+$ denote the positive nonzero elements of $\Z,\Q,\R$, respectively.
    \item \textbf{Function} (from $A$ to $B$). \emph{Also known as} \textbf{map}. \emph{Denoted by} $\bm{f:A\to B}$, $\bm{A\xrightarrow{f}{B}}$.
    \begin{itemize}
        \item Additional function-adjacent definitions not defined below: \textbf{Domain}, \textbf{codomain}, \textbf{composition}, \textbf{injection}, \textbf{surjection}, \textbf{bijection}, \textbf{bijective correspondence}, and \textbf{restriction}.
    \end{itemize}
    \item \textbf{Value} (of $f$ at $a$): \emph{Denoted by} $\bm{f(a)}$.
    \begin{itemize}
        \item Implication: We apply functions on the left throughout the book.
    \end{itemize}
    \item $\bm{f:a\mapsto b}$, $\bm{a\mapsto b}$, $\bm{f(a)=b}$ are all used interchangeably to describe the action of $f$ on \textbf{elements}.
    \begin{itemize}
        \item The middle one is used only when $f$ is understood from the context.
    \end{itemize}
    \item If $f$ is not specified on elements but defined by a rule, we must check that it is \textbf{well-defined} for each element in its domain.
    \item \textbf{Image} (of $A$ under $f$): The following subset of $B$ (the codomain of $f$). \emph{Also known as} \textbf{range}. \emph{Denoted by} $\bm{f(A)}$. \emph{Given by}
    \begin{equation*}
        f(A) = \{b\in B\mid b=f(a),\ a\in A\}
    \end{equation*}
    \item \textbf{Preimage} (of $C\subset B$ under $f$): The following subset of $A$ (the domain of $f$). \emph{Also known as} \textbf{inverse image}. \emph{Denoted by} $\bm{f^{-1}(C)}$. \emph{Given by}
    \begin{equation*}
        f^{-1}(C) = \{a\in A\mid f(a)\in C\}
    \end{equation*}
    \item \textbf{Fiber} (of $f$ over $b$): The preimage of $\{b\}$ under $f$.
    \begin{itemize}
        \item $f^{-1}$ is not, in general, a function since the fibers of $f$ generally contain many elements, i.e., many elements of $A$ in general map to the same $B$.
    \end{itemize}
    \item \textbf{Left inverse} (of $f$): A function $g:B\to A$ such that $g\circ f:A\to A$ is the identity map on $A$.
    \item \textbf{Right inverse} (of $f$): A function $h:B\to A$ such that $f\circ h:B\to B$ is the identity map on $B$.
    \item \textbf{2-sided inverse} (of $f$): A function $g:B\to A$ such that $f\circ g$ is the identity map on $B$ and $g\circ f$ is the identity map on $A$. \emph{Also known as} \textbf{inverse}.
    \begin{itemize}
        \item Implied to be unique by part (3) of Proposition \ref{prp:0.1}.
    \end{itemize}
    \item Relating properties of functions.
    \begin{proposition}\label{prp:0.1}
        Let $f:A\to B$.
        \begin{enumerate}
            \item $f$ is injective iff $f$ has a left inverse.
            \item $f$ is surjective iff $f$ has a right inverse.
            \item $f$ is a bijection iff $f$ has a 2-sided inverse.
            \item If $A,B$ satisfy $|A|=|B|$, then $f:A\to B$ is bijective iff $f$ is injective iff $f$ is surjective.
        \end{enumerate}
    \end{proposition}
    \item \textbf{Permutation} (of a set $A$): A bijection from $A$ to itself.
    \item \textbf{Extension} (of $g$ to $B$): The function $f:B\to C$ where $A\subset B$, $g:A\to C$, and $f|_A=g$.
    \begin{itemize}
        \item Extensions need not exist nor be unique.
    \end{itemize}
    \item \textbf{Representative} (of an equivalence class): Any element of the equivalence class.
    \begin{itemize}
        \item Additional relation-adjacent terms: \textbf{Binary relation}, \textbf{reflexive}, \textbf{symmetric}, \textbf{transitive}, \textbf{equivalence relation}, \textbf{equivalence class}, \textbf{equivalent} (elements), and \textbf{partition}.
    \end{itemize}
    \item The notions of an equivalence relation and a partition of $A$ are the same.
    \begin{proposition}\label{prp:0.2}
        Let $A$ be a nonempty set.
        \begin{enumerate}
            \item If $\sim$ defines an equivalence relation on $A$, then the set of equivalence classes of $\sim$ form a partition of $A$.
            \item If $\{A_i\mid i\in I\}$ is a partition of $A$, then there is an equivalence relation on $A$ whose equivalence classes are precisely the sets $A_i$, $i\in I$.
        \end{enumerate}
    \end{proposition}
    \item Assumed familiarity with induction proofs.
\end{itemize}


\subsection*{Properties of the Integers}
\begin{itemize}
    \item Many of the properties stated herein will be familiar from elementary arithmetic.
    \begin{itemize}
        \item We state them now because we will need them in Part I (Group Theory).
        \item However, we delay proofs until Chapter 8, when we prove them in the more general context of ring theory.
        \item To avoid circular reasoning, the proofs in Chapter 8 will not rely on any result from Part I, so the full logical structure of this book is Ring Theory and then Group Theory, but it is presented the other way around for pedagogical purposes.
    \end{itemize}
    \item \textbf{Well Ordering} (of $\Z$): Any nonempty subset $A\subset\Z^+$ contains a \textbf{minimal element} $m$ satisfying $m\leq a$ for all $a\in A$.
    \item \textbf{$\bm{a}$ divides $\bm{b}$}: Two numbers $a,b\in\Z$ with $a\neq 0$ such that $b=ac$ for some $c\in\Z$. \emph{Denoted by} $\bm{a\mid b}$.
    \begin{itemize}
        \item If $a$ doesn't divide $b$, then we write $\bm{a\nmid b}$.
    \end{itemize}
    \item \textbf{Greatest common divisor} (of $a,b\in\Z\setminus\{0\}$\footnote{\textcite{bib:DummitFoote} prefers the notation $\Z-\{0\}$ for set differences, but I will stick with what I know.}): The unique positive integer $d$ satisfying
    \begin{enumerate}
        \item $d\mid a$ and $d\mid b$. ($d$ is a \emph{common} divisor of $a,b$.)
        \item If $e\mid a$ and $e\mid b$, then $e\mid d$. ($d$ is the \emph{greatest} common divisor of $a,b$.)
    \end{enumerate}
    \emph{Also known as} \textbf{g.c.d.} \emph{Denoted by} $\bm{(a,b)}$.
    \item \textbf{Relatively prime} (numbers): Two numbers $a,b\in\Z\setminus\{0\}$ for which $(a,b)=1$.
    \item \textbf{Least common multiple} (of $a,b\in\Z\setminus\{0\}$): The unique positive integer $l$ satisfying
    \begin{enumerate}
        \item $a\mid l$ and $b\mid l$. ($l$ is a \emph{common} multiple of $a,b$.)
        \item If $a\mid m$ and $b\mid m$, then $l\mid m$. ($l$ is the \emph{least} common multiple of $a,b$.)
    \end{enumerate}
    \emph{Also known as} \textbf{l.c.m.}
    \item "The connection between the greatest common divisor $d$ and the least common multiple $l$ of two integers $a$ and $b$ is given by $dl=ab$" \parencite[4]{bib:DummitFoote}.
    \item \textbf{Division Algorithm}: If $a,b\in\Z\setminus 0$, then there exist unique $q,r\in\Z$ such that
    \begin{align*}
        a &= qb+r&
        0 &\leq r < |b|
    \end{align*}
    \item \textbf{Quotient}: The number $q$ in the above definition.
    \item \textbf{Remainder}: The number $r$ in the above definition.
    \item \textbf{Euclidean Algorithm}: A procedure for finding the greatest common divisor of two integers $a$ and $b$ by iterating the Division Algorithm. \emph{Given by}
    \begin{align*}
        a &= q_0b+r_0\\
        b &= q_1r_0+r_1\\
        r_0 &= q_2r_1+r_2\\
        r_1 &= q_3r_2+r_3\\
        &\vdots\\
        r_{n-2} &= q_nr_{n-1}+r_n\\
        r_{n-1} &= q_{n+1}r_n
    \end{align*}
    This yields $(a,b)=r_n$.
    \begin{proof}
        Existence of $r_n$: $|b|>|r_0|>|r_1|>\cdots|r_n|$ is a decreasing sequence of strictly positive integers and such a sequence cannot continue indefinitely.\par
        The rest of the proof comes later.
    \end{proof}
    \item Example of applying the Euclidean Algorithm given.
    \item $(a,b)$ is a $\Z$-linear combination of $a,b$: In particular, there exist $x,y\in\Z$ such that
    \begin{equation*}
        (a,b) = ax+by
    \end{equation*}
    \begin{proof}
        Exploit the Euclidean Algorithm. Use the second-to-last line to write $(a,b)$ in terms of $r_{n-1},r_{n-2}$:
        \begin{equation*}
            r_n = r_{n-2}-q_nr_{n-1}
        \end{equation*}
        Then use $r_{n-1}=r_{n-3}-q_{n-1}r_{n-2}$ to express $r_n$ in terms of $r_{n-2},r_{n-3}$. Go back and back until we express $r_n$ in terms of $a,b$, and then combine terms.
    \end{proof}
    \item Notes on the above result.
    \begin{itemize}
        \item Previous example expanded to apply here.
        \item Either $x$ or $y$ will be negative.
        \item $x$ and $y$ are not unique. The general solution is known, though (see Exercise \ref{exr:0.2.4} and Chapter 8).
    \end{itemize}
    \item \textbf{Prime} (number $p\in\Z^+$): A number $p\in\Z^+$ for which $p>1$ and the only positive divisors of $p$ are 1 and $p$.
    \item \textbf{Composite} (number $n\in\Z^+$): A number $n\in\Z^+$ for which $n>1$ and $n$ is not prime.
    \item Examples given.
    \item If $p$ is a prime and $p\mid ab$ for some $a,b\in\Z$, then $p\mid a$ or $p\mid b$.
    \begin{itemize}
        \item This property can be used to define the primes (see Exercise \ref{exr:0.2.3}).
    \end{itemize}
    \item \textbf{Fundamental Theorem of Arithmetic}: If $n\in\Z$ and $n>1$, then $n$ can be factored uniquely into the product of primes, i.e., there are distinct primes $p_1,p_2,\dots,p_s$ and positive integers $\alpha_1,\alpha_2,\dots,\alpha_s$ such that
    \begin{equation*}
        n = p_1^{\alpha_1}p_2^{\alpha_2}\cdots p_s^{\alpha_s}
    \end{equation*}
    \begin{itemize}
        \item This decomposition is unique.
    \end{itemize}
    \item Let $a,b$ be positive integers such that
    \begin{align*}
        a &= p_1^{\alpha_1}p_2^{\alpha_2}\cdots p_s^{\alpha_s}&
        b &= p_1^{\beta_1}p_2^{\beta_2}\cdots p_s^{\beta_s}
    \end{align*}
    are their prime factorizations (we let $\alpha_i,\beta_j\geq 0$ so that we can express both as the product of the same primes). Then
    \begin{align*}
        \gcd(a,b) &= p_1^{\min(\alpha_1,\beta_1)}p_2^{\min(\alpha_2,\beta_2)}\cdots p_s^{\min(\alpha_s,\beta_s)}\\
        \lcm(a,b) &= p_1^{\max(\alpha_1,\beta_1)}p_2^{\max(\alpha_2,\beta_2)}\cdots p_s^{\max(\alpha_s,\beta_s)}
    \end{align*}
    \item \textbf{Euler $\bm{\varphi}$-function}: The function $\varphi:\Z^+\to\Z^+$ where $\varphi(n)$ is defined to be the number of positive integers $a\leq n$ such that $(a,n)=1$.
    \begin{itemize}
        \item If $p$ prime, then $\varphi(p)=p-1$.
        \item If $p$ prime and $a\geq 1$, then $\varphi(p^a)=p^a-p^{a-1}=p^{a-1}(p-1)$.
        \item If $(a,b)=1$, then $\varphi(ab)=\varphi(a)\varphi(b)$.
        \item If $n=p_1^{\alpha_1}p_2^{\alpha_2}\cdots p_s^{\alpha_s}$, then
        \begin{align*}
            \varphi(n) &= \varphi(p_1^{\alpha_1})\varphi(p_2^{\alpha_2})\cdots\varphi(p_s^{\alpha_s})\\
            &= p_1^{\alpha_1-1}(p_1-1)p_2^{\alpha_2-1}(p_2-1)\cdots p_s^{\alpha_s-1}(p_s-1)
        \end{align*}
        \item $\varphi$ is used for many functions throughout the text, so when we wish to indicate the Euler $\varphi$-function, we do so explicitly.
    \end{itemize}
\end{itemize}

\subsubsection*{Exercises}
\begin{enumerate}[label={\textbf{\arabic*.}},ref={0.2.\arabic*}]
    \setcounter{enumi}{2}
    \item \label{exr:0.2.3}Prove that if $n$ is composite, then there are integers $a,b$ such that $n\mid ab$ but $n\nmid a$ and $n\nmid b$.
    \item \label{exr:0.2.4}Let $a,b,N$ be fixed integers with $a,b$ nonzero, and let $d=(a,b)$. Suppose $x_0,y_0$ are particular solutions to $ax+by=N$ (i.e., $ax_0+by_0=N$). Prove that for any integer $t$, the integers
    \begin{align*}
        x &= x_0+\frac{b}{d}t&
        y &= y_0-\frac{a}{d}t
    \end{align*}
    are also solutions to $ax+by=N$ (this is, in fact, the general solution).
\end{enumerate}


\subsection*{\texorpdfstring{$\bm{\pmb{\Z}/n\pmb{\Z}}$}{TEXT}: The Integers Modulo \texorpdfstring{$\bm{n}$}{TEXT}}
\begin{itemize}
    \item Fix $n\in\Z^+$.
    \item Define $\sim$ on $\Z$ by $a\sim b\Longleftrightarrow n\mid(b-a)$.
    \begin{itemize}
        \item We can prove that $\sim$ is an equivalence relation.
        \item If $a\sim b$, we write $a\equiv b\pmod n$\footnote{"$a$ is congruent to $b\bmod n$."}.
    \end{itemize}
    \item \textbf{Congruence class} (of $a\bmod n$): The equivalence class of $a\bmod n$. \emph{Also known as} \textbf{residue class}. \emph{Denoted by} $\bm{\bar{a}}$. \emph{Given by}
    \begin{align*}
        \bar{a} &= \{a+kn\mid k\in\Z\}\\
        &= \{a,a\pm n,a\pm 2n,a\pm 3n,\dots\}
    \end{align*}
    \begin{itemize}
        \item There are $n$ distinct equivalence classes mod $n$, namely $\bar{0},\bar{1},\dots,\overline{n-1}$, and collectively referred to as the \textbf{integers modulo $\bm{n}$}.
        \item The congruence classes differ for different $n$, so always be sure to fix $n$ before discussing them.
    \end{itemize}
    \item \textbf{Integers modulo $\bm{n}$}: The set of equivalence classes under this equivalence relation. \emph{Also known as} \textbf{integers mod $\bm{n}$}. \emph{Denoted by} $\bm{\pmb{\Z}/n\pmb{\Z}}$\footnote{The motivation for this notation will become clear in the discussion of quotient groups and quotient rings.}
    \item \textbf{Reducing $\bm{a\bmod\, n}$}: The process of finding the equivalence class mod $n$ of some integer $a$.
    \begin{itemize}
        \item Also frequently refers to finding the \textbf{least residue} of $a\bmod n$.
    \end{itemize}
    \item \textbf{Least residue} (of $a\bmod n$): The smallest nonnegative number congruent to $a\bmod n$.
    \item \textbf{Modular arithmetic} (on $\Z/n\Z$): The addition and multiplication operations defined by
    \begin{align*}
        \bar{a}+\bar{b} &= \overline{a+b}&
        \bar{a}\cdot\bar{b} &= \overline{ab}
    \end{align*}
    for all $\bar{a},\bar{b}\in\Z/n\Z$.
    \begin{itemize}
        \item In other words, take a representative element of both residue classes, add or multiply them, and then take the class containing the product to be the sum (resp. product).
    \end{itemize}
    \item Example given to hint at the well-definedness of modular arithmetic.
    \item Proof that modular arithmetic is well-defined.
    \begin{theorem}\label{trm:0.3}
        The operations of addition and multiplication on $\Z/n\Z$ defined above are both well-defined, that is, they do not depend on the choices of representatives for the classes involved. More precisely, if $a_1,a_2\in\Z$ and $b_1,b_2\in\Z$ with $\overline{a_1}=\overline{b_1}$ and $\overline{a_2}=\overline{b_2}$, then $\overline{a_1+a_2}=\overline{b_1+b_2}$ and $\overline{a_1a_2}=\overline{b_1b_2}$, i.e., if
        \begin{align*}
            a_1 &\equiv b_1\pmod n&
            a_2 &\equiv b_2\pmod n
        \end{align*}
        then
        \begin{align*}
            a_1+a_2 &\equiv b_1+b_2\pmod n&
            a_1a_2 &\equiv b_1b_2\pmod n
        \end{align*}
        \begin{proof}
            Given.
        \end{proof}
    \end{theorem}
    \item Further comments on equivalence classes and the integers mod $n$.
    \begin{itemize}
        \item Preview: Adding equivalence classes by their representatives is a special case of a more general construction (that of a \textbf{quotient}).
        \item We should be familiar with manipulating equivalence classes from studying $\Q$ rigorously.
        \item We should be familiar with modular arithmetic from timekeeping: 8 hours after 5:00 AM? Must be 13h00, but $13\equiv 1\pmod{12}$ so 1:00 PM.
        \item We do need to be able to think of equivalence classes as elements that can be manipulated in their own right. But it is important to remember that these \emph{are} still equivalence classes at the end of the day.
        \item Useful application of modular arithmetic: Computing the last two digits of $2^{1000}$ using the integers modulo 100.
    \end{itemize}
    \item $\bm{(\pmb{\Z}/n\pmb{\Z})^\times}$: The collection of residue classes which have a multiplicative inverse in $\Z/n\Z$. \emph{Given by}
    \begin{equation*}
        (\Z/n\Z)^\times = \{\bar{a}\in\Z/n\Z\mid \exists\ \bar{c}\in\Z/n\Z:\bar{a}\cdot\bar{c}=\bar{1}\}
    \end{equation*}
    \item An alternate form for $(\Z/n\Z)^\times$.
    \begin{proposition}\label{prp:0.4}
        $(\Z/n\Z)^\times=\{\bar{a}\in\Z/n\Z\mid(a,n)=1\}$.
        \begin{proof}
            See Exercises \ref{exr:0.3.10}-\ref{exr:0.3.14}.
        \end{proof}
    \end{proposition}
    \item Further comments on $(\Z/n\Z)^\times$.
    \begin{itemize}
        \item The set given in Proposition \ref{prp:0.4} is well-defined since if $(a,n)=1$, then we clearly have $(a+qn,n)=1$ as well.
        \item Explicit example given: $(\Z/9\Z)^\times$.
        \item Computing the multiplicative inverse of $\bar{a}$: Let $(a,n)=1$. Then the Euclidean algorithm generates integers $x,y$ such that $ax+ny=1$. But this implies that $ax=1+(-y)n$, i.e., $ax\equiv 1\bmod n$. Therefore, $\bar{a}\cdot\bar{x}=\bar{1}$, so $\bar{x}$ is the multiplicative inverse of $\bar{a}$.
    \end{itemize}
\end{itemize}

\subsubsection*{Exercises}
\begin{enumerate}[label={\textbf{\arabic*.}},ref={0.3.\arabic*}]
    \setcounter{enumi}{9}
    \item \label{exr:0.3.10}Prove that the number of elements of $(\Z/n\Z)^\times$ is $\varphi(n)$ where $\varphi$ denotes the Euler $\varphi$-function.
    \item \label{exr:0.3.11}Prove that if $\bar{a},\bar{b}\in(\Z/n\Z)^\times$, then $\bar{a}\cdot\bar{b}\in(\Z/n\Z)^\times$.
    \item \label{exr:0.3.12}Let $n\in\Z$, $n>1$, and let $a\in\Z$, $1\leq a\leq n$. Prove that if $a,n$ are not relatively prime, then there exists an integer $b$ with $1\leq b<n$ such that $ab\equiv 0\pmod n$ and deduce that there cannot be an integer $c$ such that $ac\equiv 1\pmod n$.
    \item \label{exr:0.3.13}Let $n\in\Z$, $n>1$, and let $a\in\Z$, $1\leq a\leq n$. Prove that if $a,n$ are relatively prime, then there exists an integer $c$ with such that $ac\equiv 1\pmod n$ [use the fact that the g.c.d. of two integers is a $\Z$-linear combination of the integers].
    \item \label{exr:0.3.14}Conclude from the previous two exercises that $(\Z/n\Z)^\times$ is the set of elements $\bar{a}$ of $\Z/n\Z$ with $(a,n)=1$ and hence probe Proposition \ref{prp:0.4}. Verify this directly in the case $n=12$.
\end{enumerate}



\section{Chapter 1: Introduction to Groups}
\emph{From \textcite{bib:DummitFoote}.}
\subsection*{A Word on Group Theory}
\begin{itemize}
    \item \marginnote{12/5:}History of and motivation for group theory and abstract algebra in general.
    \item Power of the abstract approach:
    \begin{itemize}
        \item Results for a number of examples are obtained from a single result for the abstraction.
        \item General theorems allow specific theorems of interest to be recovered as a special case, while more broadly illustrating the connections between related results.
    \end{itemize}
    \item \textbf{Groups} were abstracted from extremely old problems in algebraic equations, number theory, and geometry, all of which were found to have related solutions.
    \begin{itemize}
        \item Examples given.
    \end{itemize}
    \item "One of the essential characteristics of mathematics is that after applying a certain algorithm or method of proof, one then considers the scope and limits of the method. As a result, properties possessed by a number of interesting objects are frequently abstracted and the question raised; can one determine \emph{all} the objects possessing these properties? Attempting to answer such a question also frequently adds considerable understanding of the original objects under consideration" \parencite[13]{bib:DummitFoote}.
    \item "It is important to realize, with or without the historical context, that the reason the abstract definitions are made is because it is useful to isolate specific characteristics and consider what structure is imposed on an object having these characteristics" \parencite[15]{bib:DummitFoote}.
    \begin{itemize}
        \item Structure of algebraic objects is a major and recurring theme throughout the text.
    \end{itemize}
\end{itemize}


\subsection*{Basic Axioms and Examples}
\begin{itemize}
    \item Goal: Introduce the algebraic structure studied in Part I and give some examples.
    \item \textbf{Binary operation} (on a set $G$): A function from $G\times G$ to $G$. \emph{Denoted by} $\bm{\star}$.
    \begin{itemize}
        \item Additional binary operation-adjacent definitions: \textbf{Associative}, \textbf{commutative} (binary operation).
    \end{itemize}
    \item Examples of commutative/noncommutative and associative/nonassociative operations given.
    \item \textbf{Closed} (subset $H\subset G$ under $\star$): A subset $H\subset G$ such that $a\star b\in H$ for all $a,b\in H$, where $\star$ is a binary operation on $G$.
    \begin{itemize}
        \item Alternatively, we can require that $\star|_H$ be a binary operation on $H$.
        \item $\star$ associative/commutative on $G$ implies $\star|_H$ associative/commutative on $H$.
    \end{itemize}
    \item If $\star$ is an associative (respectively, commutative) binary operation on $G$ and $\star|_H$ is a binary operation on $H\subset G$, then $\star$ is associative (respectively, commutative) on $H$ as well.
    \item \textbf{Group}: An ordered pair $(G,\star)$ where $G$ is a set and $\star$ is a binary operation on $G$ satisfying the following axioms:
    \begin{enumerate}[label={(\roman*)}]
        \item \emph{Associativity}: $(a\star b)\star c=a\star(b\star c)$ for all $a,b,c\in G$.
        \item \emph{Identity}: There exists an element $e\in G$ such that for all $a\in G$, $a\star e=e\star a=a$.
        \item \emph{Inverses}: For all $a\in G$, there exists an element $a^{-1}\in G$ such that $a\star a^{-1}=a^{-1}\star a=e$.
    \end{enumerate}
    \item \textbf{Abelian} (group): A group $(G,\star)$ such that for all $a,b\in G$, $a\star b=b\star a$. \emph{Also known as} \textbf{commutative}.
    \item Informally, we may call $G$ a group under $\star$ or, if $\star$ is clear from context, $G$ alone a group.
    \item \textbf{Finite group}: A group $G$ for which $G$ is a finite set.
    \item Axiom (ii) implies that $G$ is nonempty.
    \item Examples of groups given (and justified).
    \begin{enumerate}
        \item $\Z$, $\Q$, $\R$, $\C$.
        \item $(\Q\setminus\{0\})^\times$, $(\R\setminus\{0\})^\times$, $(\C\setminus\{0\})^\times$, $(\Q^+)^\times$, $(\R^+)^\times$.
        \item A vector space under vector addition.
        \item $\Z/n\Z$.
        \item $(\Z/n\Z)^\times$.
        \item \textbf{Direct product} of groups $(A,\star)$ and $(B,\diamond)$.
    \end{enumerate}
    \item \textbf{Direct product} (of $(A,\star)$ and $(B,\diamond)$): The group $A\times B$ whose elements are those in the Cartesian product
    \begin{equation*}
        A\times B = \{(a,b)\mid a\in A,\ b\in B\}
    \end{equation*}
    and whose operation is defined component-wise by
    \begin{equation*}
        (a_1,b_1)(a_2,b_2) = (a_1\star a_2,b_1\diamond b_2)
    \end{equation*}
    \item Basic properties of groups.
    \begin{proposition}\label{prp:1.1}
        Let $G$ be a group under the operation $\star$.
        \begin{enumerate}
            \item The identity of $G$ is unique.
            \item For each $a\in G$, $a^{-1}$ is uniquely determined.
            \item $(a^{-1})^{-1}=a$ for all $a\in G$.
            \item $(a\star b)^{-1}=(b^{-1})\star(a^{-1})$.
            \item \textbf{Generalized associative law}: For any $a_1,\dots,a_n\in G$, the value of $a_1\star\cdots\star a_n$ is independent of how the expression is bracketed.
        \end{enumerate}
        \begin{proof}
            Given.
        \end{proof}
    \end{proposition}
    \item Further notational simplification:
    \begin{align*}
        \star &\mapsto \cdot&
        a\cdot b &\mapsto ab&
        (ab)c &\mapsto abc&
        e &\mapsto 1&
        \underbrace{xx\cdot x}_{n\text{ times}} &\mapsto x^n&
        (x^{-1})^n &\mapsto x^{-n}&
        x^0 &\mapsto 1
    \end{align*}
    \begin{itemize}
        \item Note that the operation we're using for a given group can alter the above.
        \item For example, when the operation is $+$, we let $e\mapsto 0$, $x+\cdots+x$ ($n$ times) be written as $nx$, $-a-a-\cdots-a$ ($n$ times) be written as $-nx$, and $0x=0$.
    \end{itemize}
    \item Cancellation lemma.
    \begin{proposition}
        Let $G$ be a group and let $a,b\in G$. The equations $ax=b$ and $ya=b$ have unique solutions for $x,y\in G$. In particular, the left and right cancellation laws hold in $G$, i.e.,
        \begin{enumerate}
            \item If $au=av$, then $u=v$;
            \item If $ub=vb$, then $u=v$.
        \end{enumerate}
        \begin{proof}
            Given.
        \end{proof}
    \end{proposition}
    \item Corollary: Either $ab=e$ or $ba=e$ implies $b=a^{-1}$ directly; we don't have to verify the other if we only know one.
    \item Corollary: Either $ab=a$ or $ba=a$ implies $b=e$.
    \begin{itemize}
        \item Related to the previous one.
    \end{itemize}
    \item \textbf{Order} (of $x$): The smallest positive integer $n$ such that $x^n=1$. \emph{Denoted by} $\bm{|x|}$.
    \begin{itemize}
        \item We say $x$ is of order $n$.
        \item If no such $n$ exists, the order of $x$ is defined to be infinity and $x$ is said to be of infinite order.
        \item Note that $|\cdot|$ means different things in the contexts $|g|$ and $|G|$ (so be careful), but the uses are naturally related since the order of $g$ is equal to the cardinality of the set of all its (distinct) powers (see Proposition \ref{prp:2.2}).
    \end{itemize}
    \item Examples.
    \begin{itemize}
        \item $|g|=1$ iff $g=e$.
        \item Others given.
    \end{itemize}
    \item \textbf{Multiplication table} (of a finite group): The $n\times n$ matrix whose $i,j$ entry is the group element $g_ig_j$, where $G=\{g_1,\dots,g_n\}$ and $g_1=e$. \emph{Also known as} \textbf{group table}.
    \begin{itemize}
        \item Contains all group information, but is a computationally and visually unwieldy object. Does not easily reveal deeper relations.
        \item Analogy: A multiplication table is like having a list of the distances between every US city; what would be far more useful is a map with such distances labeled on it.
    \end{itemize}
    \item A goal going forward: Develop a better visualization of the internal structure of groups.
\end{itemize}




\end{document}