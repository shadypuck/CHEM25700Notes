\documentclass[../notes.tex]{subfiles}

\pagestyle{main}
\renewcommand{\chaptermark}[1]{\markboth{\chaptername\ \thechapter\ (#1)}{}}
\setcounter{chapter}{7}

\begin{document}




\chapter{Applications of the Sylow Theorems}
\section{Sylow III and Examples}
\begin{itemize}
    \item \marginnote{11/14:}Last time:
    \begin{itemize}
        \item Sylow I: $p$-Sylow subgroups exist.
        \item Sylow II: $p$-Sylow subgroups are unique up to conjugation. Moreover, if $Q\subset G$ is a $p$-group, then $Q\subset gPg^{-1}$ with the same $g$.
        \item We proved Sylow II by taking $H\subset G$, and separately taking $P\subset G$ to be $p$-Sylow. In this case, there exists $g\in G$ such that $H\cap gPg^{-1}$ is a $p$-Sylow of $H$. If $H=Q$, then $Q\cap gPg^{-1}=Q$.
        \begin{itemize}
            \item More on this??
        \end{itemize}
    \end{itemize}
    \item Alternate proof of Sylow II.
    \begin{proof}
        % Consider the action $G\acts G/P$ by left multiplication. Restrict it to $Q\hookrightarrow G$. So why do we want to consider this action? $P$ is a $p$-group. Fact: For any $P,Q$, $Q$ fixes a point in $G/P$. This is equivalent to $Q\subset G/P$. $Q$ stabilizes a point $gP$, so $Q\subset gPg^{-1}=\Stab(gP)$. One group is contained in the other iff the action of one on the other has a fixed point??\par
        % Say $Q$ has a fixed point. We have proven that the number of fixed points of $Q$ is congruent to $|G/P|\mod p$; this is also equal to $|G|/|P|$. The order of $P$ is the largest power of $p$ dividing $G$. Thus, $|G|/|P|\neq 0\mod p$, so there (doesn't?? which one is it) exists a fixed point.


        We attack the first claim (equality for $p$-Sylows) in three steps; we will not prove the second claim (containment for $p$-groups) herein. Step 1 defines a useful group action, allowing us to apply relevant theorems from that domain later on. Step 2 proves the existence of a fixed point of said group action, which will be intimately related to the final element $g$ by which we conjugate $P$ to make it equal $Q$. Step 3 relates this element $g$ to the desired result. Let's begin.\par\smallskip
        Let $X$ denote the set of all $p$-Sylows of $G$. By Sylow I, $X$ is nonempty. Thus, we may choose $P,Q\in X$ (note that $P,Q$ are not necessarily distinct). Define $G\acts G/P$ by left multiplication. Restrict the group action to $Q$ (i.e., restrict the function $\cdot:G\times G/P\to G/P$ to $Q\times G/P$).\par
        Since $|G|=p^nk$ and $|P|=p^n$, we have that $\gcd(|G/P|,p)=1$. Thus, $|G/P|$ is not divisible by $p$, so $|G/P|\mod p\not\equiv 0\mod p$. Additionally, since $Q$ is a $p$-group (by definition as a $p$-Sylow), we have from the proposition in Lecture 7.2 that $\Fixed(G/P)\equiv|G/P|\mod p$. This combined with the previous result reveals that $\Fixed(G/P)$ is nonempty. As such, we may choose $gP\in\Fixed(G/P)$.\par
        By definition, $Q$ stabilizes $gP$, i.e.,
        \begin{align*}
            QgP &= gP\\
            g^{-1}QgP &= P
        \end{align*}
        where the latter equation above is a simple rearrangement of the first, but can be interpreted to mean that $g^{-1}Qg$ stabilizes $P$. Thus, if $g^{-1}qg\in g^{-1}Qg$, we have $(g^{-1}qg)p_1=p_i$ for some $i=1,\dots,p^n$, and hence $q=g(p_ip_1^{-1})g^{-1}\in gPg^{-1}$. Therefore, $Q\subset gPg^{-1}$. Since $|P|=|Q|$, we additionally have that $Q=gPg^{-1}$, as desired.
    \end{proof}
    \item Sylow III. The first is existence, the second is uniqueness, and then there's this one (divisibility and congruence).
    \item Theorem (Sylow III --- divisibility and congruence): Let $P$ be a $p$-Sylow, and let $n_p$ denote the number of $p$-Sylows of $G$. Then
    \begin{enumerate}
        \item Let $N=N_G(P)$. Then $n_p=|G|/|N|=[G:N]$. In particular, $n_p$ divides $|G|$.
        \begin{proof}
            % What group action is relevant to 1? Should be something to do with conjugation, because we have a normalizer involved. The normalizer is associated with the stabilizer. Let $X=\{p\text{-Silow subgroups}\}$. $G\acts X$ by conjugation. Apply O-S. Then $|G|=|\Stab_G(P)|\cdot|\Orb(P)|=|N|\cdot|X|$. The fact that $\Orb(P)=X$ follows from Sylow II. And $|X|=n_p$, implying the desired result.

            To prove a claim which expresses $|G|$ in terms of the product of two other numbers, we should think about using the Orbit-Stabilizer theorem. To do so, we need a group action. In particular, a group action by conjugation could be useful because we have a normalizer involved. With this motivation mentioned, let's begin.\par\smallskip
            Let $X$ be the set of $p$-Sylows of $G$. Define $G\acts X$ by conjugation. By the Orbit-Stabilizer theorem,
            \begin{equation*}
                |\Stab_G(P)|\cdot|\Orb(P)| = |G|
            \end{equation*}
            Since the group action is by conjugation, we have by the definition of the stabilizer and the normalizer that
            \begin{equation*}
                \Stab_G(P) = \{g\in G\mid gPg^{-1}=P\}
                = N_G(P)
                = N
            \end{equation*}
            According to Sylow II, every $p$-Sylow (every element of $X$) is conjugate to every other via some element of $G$. Thus, since our group action is conjugation, the group action is transitive and $\Orb(P)=X$. Thus,
            \begin{equation*}
                |\Orb(P)| = |X| = n_p
            \end{equation*}
            Therefore, substituting the previous two results into the preceding one, we have that
            \begin{align*}
                |N|\cdot n_p &= |G|\\
                n_p &= |G|/|N| = [G:N]
            \end{align*}
            as desired.
        \end{proof}
        \item $n_p\equiv 1\mod p$.
        \begin{proof}
            % Second part. Congruence should make us think, "fixed points." Let $|Y|=n_p$. We have a $p$-group acting on $Y$ with 1 fixed point. Take $X$ since $|X|=n_p$. Restrict action by conjugation to $P\acts X$. $P$ acting on the $p$-Sylows. Doesn't have to be transitive like the action of $G$. We want to compute the number of fixed points.\par
            % $P$ is a fixed point of $P$. Assume $Q\neq P$ is a fixed point of $P$. How do we translate this statement about the group action into a statement about the groups $P,Q$. This translates to $P\subset N=N_G(Q)$. What do we know about $N$? Fact 1: $P\subset N$. Fact 2: $Q\subset N$. Fact 3: $Q\triangleleft N$. $|P|$ is the largest power of $p$ that divides $G$. $G$ has order $p^n\cdot k$, and each of $P,Q$ has order $p^n$. Thus, $P,Q$ are $p$-Sylow subgroups of $N$. Just what's on this board is now enough to achieve a contradiction. Therefore, either by II or III(1), we can deduce that $P=Q$. Now we apply this to $N$.


            Congruence should make us think, "fixed points." In this argument, we will pick up where we left off, using the same group action defined in the proof of part 1 to express the claim in the language of fixed points. We will then deduce that this latter claim is true, proving the original claim. Let's begin.\par\smallskip
            Restrict the action from part 1 to $P$. This may mean that $P\acts X$ is no longer transitive, but this will not cause any issues. Moving on, we know by the closure of subgroups that $gPg^{-1}=P$ for any $g\in P$; thus, $P$ is a fixed point of $P\acts X$. It follows by the proposition from Lecture 7.2 that $\Fixed_P(X)\equiv|X|\mod p$, and hence $n_p=|X|\equiv\Fixed_P(X)\mod p$. Thus, we are done if we can show that $\Fixed_P(X)=1$, i.e., that $P$ is the only fixed point of $X$ under $P\acts X$.\par
            Let $Q\in\Fixed_P(X)$ be arbitrary; we seek to prove that $Q=P$. Define $N:=N_G(Q)$. By definition, $Q\subset N$. Additionally, $P\subset N$: Since $Q\in\Fixed_P(X)$, $gQg^{-1}=g\cdot Q=Q$ for all $g\in P$. Hence $P,Q$ are both $p$-Sylows of $N$ (the order of $p$ dividing $|N|$ certainly [by Lagrange's Theorem] divides the order of $p$ dividing $|G|$). By Sylow II, any two $p$-Sylows are conjugate, so there exists $n\in N$ such that $nQn^{-1}=P$. Additionally, since $Q\triangleleft N$ by HW4 Q3c, we have that $nQn^{-1}=Q$. Therefore, by transitivity, $P=Q$, as desired.
        \end{proof}
    \end{enumerate}
    \item We are now done with proving the Sylow theorems. Make sure you have nice copies written out!
    \begin{itemize}
        \item Perhaps before the final, I should take all important proofs from the quarter and make "proof outlines" in my review sheet, giving the tricks and motivation in as concise a format as possible but still allowing me to deduce the rest of the proof for myself. This could be a great exercise!
    \end{itemize}
    \item The arguments that we've used thus far in this class are mostly combinatorical with a bit of number theory sprinkled in.
    \item Before going into applications of the Sylow theorems, we present an example that's good to keep in mind.
    \item Let $G=S_p$ for some $p\in\N$ prime.
    \begin{itemize}
        \item S I: Yes, $G$ has a $p$-Sylow, namely $P=\gen{(1,2,\dots,p)}$.
        \item S II: Any $p$-cycles are conjugate to one another.
        \item Intuitive derivation of the value of $n_p$: $n_p$ is the number of elements of order $p$\footnote{Recall that this is $p!/p$, since there are $p$ options for the first entry, $p-1$ for the second, on and on down to 1, but there are also $p$ ways to write said element.} divided by $p-1$\footnote{Each $p$-Sylow $P$ contains $p-1$ distinct $p$-cycles.}. Thus,
        \begin{equation*}
            n_p = \frac{p!}{p(p-1)}
            = (p-2)!
        \end{equation*}
        \item S III: $(p-2)!\equiv 1\mod p$.
        \begin{itemize}
            \item We obtain a related statement from \textbf{Wilson's theorem}: $(p-1)!\equiv -1\mod p$.
        \end{itemize}
        \item S III: $|N|=|N_G(P)|=p(p-1)$.
        \item This result combined with $P\triangleleft N$: $|N/P|=p-1$.
    \end{itemize}
    \item Theorem (Wilson's theorem): A natural number $p>1$ is prime iff
    \begin{equation*}
        (p-1)! \equiv -1\mod p
    \end{equation*}
    \item \textbf{Affine group} (of order $p$): The following group, which consists of permutations given by affine maps. \emph{Denoted by} $\bm{\Aff_p}$. \emph{Given by}
    \begin{equation*}
        \Aff_p = S_{\Z/p\Z}
    \end{equation*}
    \begin{itemize}
        \item We send $x\in\Z/p\Z$ to $ax+b\in\Z/p\Z$.
        \item Injective:
        \begin{align*}
            ax+b &= ay+b\\
            a(x-y) &\equiv 0\mod p\\
            x &= y
        \end{align*}
        \item We also need to check that $\Aff_p$ is actually a subgroup. The group operation...
        \item An affine map is the sum of a linear transformation and a translation. Thus,
        \begin{equation*}
            A(ax+b)+B = Aax+Ab+B
        \end{equation*}
        so
        \begin{equation*}
            (a,b)(A,B) = (aA,Ab+B)
        \end{equation*}
        \item We claim that $P=\gen{X\to X+1}$ is a subgroup??
        \item In particular, $P\triangleleft\Aff_p\leq N$.
        \item Thus, $\Aff_p=N_{S_p}(\gen{(1,2,\dots,p)})$. This is a nice new group to have.
        \item We have $P:\Aff_p\to(\Z/p\Z)^*$ defined by $\gen{x\mapsto x+b}$. $x\mapsto ax+b$ goes to $a$ in the codomain, $Ax+B$ maps to $A$, and $aAx+\cdots$ maps to $aA$.
        \item Remark: If $q|p-1$ is prime, then $(\Z/p\Z)^*$ has an element of order $q$ (Sylow). Call it $\sigma$. Then $\gen{\sigma}\leq(\Z/p\Z)^*$.
    \end{itemize}
    \item Theorem: Let $p,q$ be primes such that $p>q$. Then either\dots
    \begin{enumerate}
        \item $p\equiv 1\mod q$ and there exists a nonabelian group of order $pq$ that is a subset of $\Aff_p$.
        \item $p\not\equiv 1\mod q$ and all groups of order $pq$ are isomorphic to $\Z/p\Z\times\Z/q\Z\cong\Z/pq\Z$.
    \end{enumerate}
    \begin{proof}
        ...
    \end{proof}
    \item Misc notes: According to S III\dots
    \begin{itemize}
        \item $|G|=pq$ and $n_p\equiv 1\mod p$. Either $n_p=1$ or $n_p=q\equiv 1\mod p$, implying $q>p$, a contradiction.
        \item Alternatively, $G\cong P_p\times P_q$. $n_q=1$ or $n_q=p$. If $p\not\equiv 1\mod q$, then $n_q=1$. We end up with $P_p\trianglelefteq G$ and $P_q\trianglelefteq G$, which implies that $P_p\cap P_q=\{e\}$. Therefore, $P_p$ and $P_q$ commute.
    \end{itemize}
    \item First example: 15; the first composite number for which $p,q>2$ (and thus the structure is not covered by our previous analysis).
    \item We still haven't completely classified groups of order $pq$; sometimes there's one, sometimes there's more. We will look at these groups in greater detail next lecture.
\end{itemize}



\section[Groups of Order \texorpdfstring{$pq$}{TEXT}]{Groups of Order \texorpdfstring{$\bm{pq}$}{TEXT}}
\begin{itemize}
    \item \marginnote{11/16:}Classifying groups of order $|G|=2p$ for $p>2$ prime.
    \item By Sylow I, there exists a $p$-Sylow $P_p$ and a 2-Sylow $P_2$.
    \begin{itemize}
        \item Since $[G:P_p]=2$, HW4 Q6 implies that $P_p$ is normal.
        \begin{itemize}
            \item Alternate strategy: By SyIII, $n_p\equiv 1\mod p$ and $n_p=|G|/|N|=|G|/|P|=2p/p=2$. Thus, $n_p=1$ or $n_p=2$. These facts combine to say that $n_p=1$ and $P_p\trianglelefteq G$.
        \end{itemize}
        \item By Lagrange's Theorem, we must have $P_p=\gen{x}$ and $P_2=\gen{y}$ for some $x,y\in G$.
        \item $x^p=e=y^2$.
        \item $G=\gen{x,y}$.
    \end{itemize}
    \item The elements have order 1, 2, $p$ or $2p$ by Lagrange.
    \item Since $\gen{x}$ is normal, it follows that
    \begin{gather*}
        y\gen{x}y^{-1} = \gen{x}\\
        yxy^{-1}\in\gen{x}\\
        yxy^{-1}=x^k
    \end{gather*}
    where the $x,y$ used throughout are the previously referenced generators (not any sort of arbitrary variable).
    \item Goal: Put constraints on $k$.
    \item $k\equiv 0\mod p$ iff $x=e$.
    \begin{itemize}
        \item If $k\equiv 0\mod p$, then $yxy^{-1}=x^k=e$, so $x=y^{-1}y=e$.
        \item If $x=e$, then $x^k=yey^{-1}=e$, so we must have $k\equiv 0\mod p$.
    \end{itemize}
    \item A preview of something we will shortly prove.
    \begin{itemize}
        \item There are two groups of order $2p$: $D_{2p}$ and $\Z/2p\Z$.
        \item In the latter, $k=1$.
        \begin{itemize}
            \item Since $\Z/2p\Z$ is abelian, the conjugate of any element is itself. Thus, $yxy^{-1}=x^1$.
        \end{itemize}
        \item In the former, $k=-1$ (if conjugating by a reflection??).
        \begin{itemize}
            \item Recall the multiplication rule $rs=sr^{-1}$, from which we can deduce that $srs^{-1}=r^{-1}$.
            \item Note that it is proper to use $s$ analogously to $y$ and $r$ analogously to $x$ since reflections ($s$) have order 2 like $y$ and rotations ($r$) can have much higher orders (e.g., $p$).
        \end{itemize}
    \end{itemize}
    \item Another (redundant??) possibility: $yx^iy^{-1}=yx^{ik}y^{-1}$.
    \item We now prove that there are only two groups of order $2p$.
    \item Conjugating $x$ by $y$ twice gives us
    \begin{equation*}
        x = exe
        = y^2xy^{-2}
        = y(yxy^{-1})y^{-1}
        = yx^ky^{-1}
        = (yxy^{-1})^k
        = (x^k)^k
        = x^{k^2}
    \end{equation*}
    \begin{itemize}
        \item Comparing exponents, we have $k^2\equiv 1\mod p$.
        \item This is equivalent to $(k^2-1)\equiv 0\mod p$, which in turn is equivalent to $(k+1)(k-1)\equiv 0\mod p$.
        \item It follows that $k\equiv\pm 1\mod p$.
    \end{itemize}
    \item Now we must consider each case in turn.
    \item If $k=1$, then $G$ is abelian, i.e., $G=P_p\times P_2$.
    \begin{itemize}
        \item Example: $\Z/2p\Z\cong\Z/2\Z\times\Z/p\Z$.
        \item We'll see a lot of this breaking up of groups next quarter.
        \item Calegari alludes to the \textbf{Chinese remainder theorem}.
    \end{itemize}
    \item Theorem (Chinese remainder theorem): Let $m,n$ be relatively prime positive integers. For all integers $a,b$, the pair of congruences
    \begin{align*}
        x &\equiv a\mod m\\
        y &\equiv b\mod m
    \end{align*}
    has a solution, and this solution is uniquely determined modulo $mn$.
    \item If $k=-1$, then $yx=x^{-1}y$.
    \begin{table}[h!]
        \centering
        \small
        \renewcommand{\arraystretch}{1.2}
        \begin{tabular}{c|c|c}
             & $x^i$ & $x^iy$\\
            \hline
            $x^j$ & $x^{i+j}$ & $x^{i+j}y$\\
            $x^jy$ & $x^{j-i}y$ & $x^{j-i}$\\
        \end{tabular}
        \caption{Multiplication table for $|G|=2p$ and $k=-1$.}
        \label{tab:2pmult}
    \end{table}
    \begin{itemize}
        \item We still have that $x^p=1$.
        \item We want to show based on this multiplication rule that we really have the dihedral group. Once we have this, there's at most one group it could possibly be. Since $D_{2p}$ is such a group, then they must be isomorphic.
        \item To do so, we show that the rule determines the multiplication table (see Table \ref{tab:2pmult} above).
        \item Thus, there is at \emph{most} one group.
        \item But since $D_{2p}$ exists, there is also at \emph{least} one group.
        \item Therefore, if $k=-1$, we must have $G\cong D_{2p}$.
    \end{itemize}
    \item Proposition: Let $|G|=2n$, $n>2$. If $x\in G$ and $|x|=n$, $|y|=2$, $yx=x^{-1}y$ implies $G\cong D_{2n}$.
    \begin{proof}
        The multiplication table is uniquely determined (analogous to the above argument).
    \end{proof}
    \item Remark about $D_4=K$, where $K$ is the Klein 4-group??
    \item We now move on to $|G|=pq$, where $p>q$ are both prime.
    \item Applying S III, we get $n_p$ equals 1 or $q$ and is congruent to 1 mod p, and $n_q$ equals 1 or $p$ and is congruent to 1 mod q.
    \begin{itemize}
        \item Thus, $n_p=1$ always and $n_q=1$ unless $p=1\mod q$.
    \end{itemize}
    \item If $|G|=pq$ and$p>2$, $p\not\equiv 1\mod q$, then $G\cong\Z/p\Z\times\Z/q\Z$.
    \item Case where $|G|=q$ and $p\equiv 1\mod q$. Then $P_p=\gen{x}$ and $P_q=\gen{y}$, so $P_p\trianglelefteq G$. This is another (strange??) application of S III.
    \begin{itemize}
        \item Using what we have here, we know that $yxy^{-1}=x^k$, $k\not\equiv 0\mod p$. $k=1$ implies $G$ is abelian and $G\cong\Z/p\Z\times\Z/q\Z$.
        \item Now we just need to conjugate $x$ by $y$, $q$ times over: $x=y^qxy^{-q}=x^{k^q}$. Thus, $k^q\equiv 1\mod p$.
        \item Unlike when $q=2$, we could factor then. Now we've got a more difficult problem; can't factor it.
        \item Does there exist $q$ satisfying the above property? If so, how many are there?
        \item Think about this as an identity in the multiplicative group $(\Z/p\Z)^\times$ which has order $p-1$. We can thus deduce by Lagrange that $q|p-1$.
        \item Sylow I: There exists $\eta$ of order $q$ such that $\eta,\eta^2,\eta^3,\dots,\eta^{q-1}$ all have order $p$.
        \item We could argue that $(\Z/p\Z)^\times$ is cyclic (and in fact it is), but here's something else: We have that $k^q-1=(k-1)(k-\eta)\cdots(k-\eta^{q-1})$. This is factoring polynomials mod $p$ (weird for now, but very commonplace next quarter).
        \item Fix $\eta$. Then $yxy^{-1}=x^{\eta^2}$.
    \end{itemize}
    \item Claim I: This determines the multiplication table; $\gen{x}\subset G$. The right cosets $\gen{x},\gen{x}y,\dots,\gen{x}y^{q-1}$. $G/P_p\cong\Z/q\Z$. If we have all of the elements of the form $x^iy^j$, do we know how to multiply these together? In particular, can we determine how to write
    \begin{equation*}
        x^iy^jx^ay^b = x^ry^s
    \end{equation*}
    We have that $yx=x^{\eta^i}y$, so the multiplication table is determined. This implies that there is at most $q-1$ nonabelian groups.
    \item Now we have
    \begin{gather*}
        yxy^{-1} = x^{\eta^i}\\
        y^2xy^{-1} = x^{\eta^{2i}}\\
        \vdots\\
        y^rxy^{-r} = x^{\eta^{ri}}
    \end{gather*}
    Thus, $\eta^{ri}=\eta$. Therefore, $y_i=y^r$ so $yxy^{-1}=x^\eta$, so there is at most 1 non abelian group.
    \item But, $P$ a $p$-Sylow of $S_p$ and $N=N_{S_p}(P)$ and $C=C_{S_p}(P)$ gives us $|N|=p(p-1)$ and $|C|=p$ so that $N/C=(\Z/p\Z)^\times$. We now take the preimage in $N$ so that $\gen{y,x}=G$. $|G|=pq$. Then $P,G$ abelian would imply $G\subset C$, but this is not possible since $G$ ahs $pq$ and $C$ has $p$, so $G$ is not abelian.
    \item Example $21=7\cdot 3$. $2^3\equiv 1\mod 7$. Then we take $\Z/7\Z\to\Z/7\Z$ so we take $x\mapsto x+a$, $x\mapsto 2x+a$, $x\mapsto 4x+a$, on and on where $a$ is a constant. There are 21 such maps.
    \item If $\eta^1=1\mod p$, then the affine maps from $\Z/p\Z$ to $\Z/p\Z$ send $x\mapsto \eta^ix+b$.
    \item If we call $\sigma=x+1$ and and $\tau=x\to x\eta$, then $x\mapsto x+\eta=\sigma\eta$.
    \item The set of affine maps has both $\Z/p\Z$ and $(\Z/p\Z)^\times$ as subsets.
    \item If we think about the groups we've classified, we've classified $1,p,p^2,p^3,pq$. $p^3$ just a bit, though. Limit to this strategy: The prime factorizations are so simple that we get immediate and very restrictive information about the $p$-Sylow subgroups (e.g., the biggest one is normal). This can't occur indefinitely because we will eventually get to cases like $A_5$ of order 60, for example, which has no normal subgroups.
    \item If we think about our progress (classifying groups of low order up to 4), then going upwards, the first group we can't do is of order $12=2\cdot 2\cdot 3$. This is like $A_4$, which is not too bad but all the same, $n_3=1,4$, $n_2=1,3$. If $n_3=4$, then we have an action of $G$ on the 3 Sylow's, giving a transitive map from $G$ to $S_4$. Thus, the stabilizer has size 3.
    \item $n_3=1$, so $G=P_3\times P_2$. $n_3=1$ and $n_2=3$, so $G\times S_3$. Since there is such an explosion of groups, this is not the optimal strategy. Thus, ...
    \item We may do a review session of the 25 practice problems over Twitch with him playing speedtest.
    \item At this point, we have the tools to do every outgoing homework problem, save the last one of the last psets on symmetry groups.
\end{itemize}



\section{Symmetries in Three-Space}
\begin{itemize}
    \item \marginnote{11/18:}Classify the finite subgroups of $\text{SO}(3)$.
    \item We can take any regular $n$-gon and think of $D_{2n}\subset\text{O}(2)\subset\text{SO}(2)$.
    \item Five platonic solids: Te, Cu, Oc, Do, and Ic.
    \item Cu and Oc are paired and Do and Ic are paired. $\text{Te}\cong A_4$, $\text{Cu}\cong\text{Oc}\cong S_4$, and $\text{Do}\cong\text{Ic}\cong A_5$.
    \item Theorem: Let $G\subset\text{SO}(3)$ be a finite group. Then $G$ is conjugate to one of these groups.
    \item Let $g\in\text{SO}(3)$, $g\neq e$. The only fixed points of $g$ lie on a line $\ell$ which contains the origin 0.
    \item We have a group action $\text{SO}(3)\acts S^2=\{v\mid\norm{v}=1\}$. Consider $G\acts S^2$. Any $g\neq e$ has exactly 2 fixed points which we may call $\{\pm u\}$ for some $u$.
    \item Thus, $|\Stab(x)|=1$ for all but finitely many points $x\in S^2$.
    \item Claim:
    \begin{equation*}
        \sum_{x\in S^2}|\Stab(x)-1|
    \end{equation*}
\end{itemize}




\end{document}