\documentclass[../notes.tex]{subfiles}

\pagestyle{main}
\renewcommand{\chaptermark}[1]{\markboth{\chaptername\ \thechapter\ (#1)}{}}
\setcounter{chapter}{6}

\begin{document}




\chapter[Group Action Applications: \texorpdfstring{$A_5$}{TEXT} and the Sylow Theorems]{Group Action Applications: \texorpdfstring{$\bm{A_5}$}{TEXT} and the Sylow Theorems}
\section[Actions of \texorpdfstring{$A_5$}{TEXT}]{Actions of \texorpdfstring{$\bm{A_5}$}{TEXT}}
\begin{itemize}
    \item \marginnote{11/7:}Classifying subgroups of $G=A_5\cong\text{Do}$.
    \item Let $H\leq G$. We must have $|H|\big||G|$ by Lagrange's theorem.
    \begin{itemize}
        \item Thus, if $H\leq A_5$, we must have
        \begin{equation*}
            |H| \in \{1,2,3,4,5,6,10,12,15,20,30,60\}
        \end{equation*}
    \end{itemize}
    \item A good place to start is with orders of $H$ that correspond to cyclic subsets.
    \item In particular, let's start with subgroups of the form $\gen{(**)(**)}$, which all have order 2.
    \begin{itemize}
        \item Are such groups conjugate?
        \item To prove that two groups of the form $\gen{(**)(**)}$ are conjugate, it will suffice to show that their generators are conjugate (since the only other element --- the identity --- will naturally be conjugate to itself).
        \item Let $x,y\in A_5$ be arbitrary elements of the form $(**)(**)$. Then there exists $g\in S_5$ such that $gxg^{-1}=y$.
        \item But is $g\in A_5$? If $g\in A_5$, then we are done. If $g\notin A_5$, then can we find an element $g'\in A_5$ such that $g'xg'^{-1}=y$?
        \item First, note that if $gxg^{-1}=y=g'xg'^{-1}$, then
        \begin{align*}
            g^{-1}(gxg^{-1})g' &= g^{-1}(g'xg'^{-1})g'\\
            x(g^{-1}g') &= (g^{-1}g')x
        \end{align*}
        Thus, $g^{-1}g'\in C_{S_5}(x)$, or $g'=gh$ for some $h\in C_{S_5}(x)$.
        \begin{itemize}
            \item If $g\notin A_5$ and we want $g'\in A_5$, then we must have $h\notin A_5$.
            \begin{itemize}
                \item Intuitively, this means that if $g$ is the product of an odd number of permutations and we want $g'=gh$ to be the product of an even number of permutations, $h$ had better be a product of an odd number of permutations as well.
                \item More formally, consider $G/A_5$. If $g\in gA_5\neq A_5$ and we want $g'\in g'A_5=A_5$, then by homomorphically mapping $gA_5$ to $1\in\Z/2\Z$ and $A_5$ to $0\in\Z/2\Z$, we must have $h\in gA_5$ to get $gh\in A_5$.
            \end{itemize}
            \item Regardless, this example motivates the following two propositions, which we can use to resolve the original conjugacy question.
        \end{itemize}
        \item By Proposition 1, since $x\sim y$ in $S_5$ and $C_{S_5}(x)\not\subset A_5$ (take the first transposition in $(**)(**)$; for example, know that $(12)$ commutes with $(12)(34)$), we know that $x\sim y$ in $A_5$.
        \item Therefore, there are 15 subgroups of the form $\gen{(**)(**)}$, all of which are conjugate in $A_5$.
    \end{itemize}
    \item Proposition 1: Let $x\sim y$ in $S_n$. Then if $C_{S_n}(x)\not\subset A_n$, then $x\sim y$ in $A_n$.
    \begin{proof}
        Since $x\sim y$ in $S_n$, there exists $g\in S_n$ such that $gxg^{-1}=y$. If $g\in A_n$, then we are done. Now suppose $g\notin A_n$. Since $C_{S_n}(x)\not\subset A_n$, there exists $h\in C_{S_n}(x)$ such that $hxh^{-1}=x$ and $h\notin A_n$. Since $g,h\notin A_n$, we have that $gh\in A_n$. Additionally, we have that
        \begin{equation*}
            (gh)x(gh)^{-1} = g(hxh^{-1})g^{-1}
            = gxg^{-1}
            = y
        \end{equation*}
        Therefore, $x\sim y$ in $A_n$, as desired.
    \end{proof}
    \item Proposition 2: If $C_{S_n}(x)\subset A_n$ and $\sigma x\sigma^{-1}=y$, then $x\sim y$ in $A_n$ iff $\sigma\in A_n$.
    \begin{proof}
        Suppose first that $x\sim y$ in $A_n$. Then $gxg^{-1}=y$ for some $g\in A_n$. Then as per the above, $gxg^{-1}=\sigma x\sigma^{-1}$ implies that $g^{-1}\sigma\in C_{S_n}(x)$. Thus, $\sigma=gh$ for some $h\in C_{S_n}(x)\subset A_n$. But since $g,h\in A_n$, we must have $\sigma\in A_n$, too.\par
        Now suppose that $\sigma\in A_n$. Then since $\sigma x\sigma^{-1}=y$, $x\sim y$ in $A_n$ as desired.
    \end{proof}
    \item Now we discuss subgroups of the form $\gen{(***)}$.
    \begin{itemize}
        \item Let $x$ be an arbitrary element of $A_5$ of the form $(***)$. In particular, suppose $x=(abc)$ for $a,b,c\in[5]$.
        \item Then $(de)\in C_{S_5}(x)$, where $d,e\in[5]$ are the other two elements that are not already represented by $a,b,c$.
        \item Moreover, $(de)$ will be in the centralizers of both $x$ and $x^2$.
        \item There are $\binom{5}{2}=10$ subgroups of the form we're discussing (20 generators/elements of the form $(***)$, though).
        \item Suppose we have two subgroups $\gen{x},\gen{y}$ of the form being discussed. We know that $\gen{x},\gen{y}$ are conjugate in $S_5$. But since $C_{S_5}(x)\not\subset A_5$ again as per the above, we know the groups are conjugate in $A_5$.
        \item Therefore, there are 10 subgroups of the form $\gen{(***)}$, all of which are conjugate in $A_5$.
    \end{itemize}
    \item Now we discuss subgroups of the form $\gen{(*****)}$.
    \begin{itemize}
        \item We know that $|C_{S_5}((12345))|\cdot|\{(12345)\}|=120$. Additionally, only a power of $(12345)$ commutes with it in this case, so the first term is 5. Thus, the second must be 24.
        \begin{itemize}
            \item In sum, we have showed that there are 24 elements conjugate to $(12345)$ in $S_5$.
            \item Another way we could show this is by counting all of the 5-cycles and knowing that they are all conjugate as 5-cycles. Indeed, there are $4!=24$ 5-cycles.
        \end{itemize}
        \item Claim: In $A_5$, $|x|=5$ implies $x\sim x$, $x\nsim x^2$, $x\nsim x^3$, and $x\sim x^4=x^{-1}$.
        \begin{proof}
            We know that $|x|=5$. Thus, let $x=(abcde)$.\par
            By the above statements on $C_{S_5}((12345))$, we know that $C_{S_5}(x)\subset A_5$. Thus, by proposition 2, $gxg^{-1}=x'$ iff $g\in A_n$. Thus,
            \begin{align*}
                exe^{-1} = x &\quad\Longrightarrow\quad x\sim x\\
                [(bc)(cd)(de)]x[(bc)(cd)(de)]^{-1} = (bced)(abcde)(bced)^{-1} = (acebd) &\quad\Longrightarrow\quad x\nsim x^2\\
                (bdec)(abcde)(bdec)^{-1} = (adbec) &\quad\Longrightarrow\quad x\nsim x^3\\
                [(be)(cd)](abcde)[(be)(cd)]^{-1} = (aedcb) &\quad\Longrightarrow\quad x\sim x^4=x^{-1}
            \end{align*}
            as desired.
        \end{proof}
        \item $x^2\sim x^3$ in $A_5$ as well.
        \item $(abced)$ and $(acebd)$ are conjugate by $(bce)\in A_5$.
        \item Six subgroups, all conjugate.
        \item All of the subgroups are conjugate, but not all of the elements are conjugate?
    \end{itemize}
    \item Consider $K=\{e,(12)(34),(13)(24),(14)(23)\}\triangleleft A_4\subset A_5$.
    \item Consider a transitive group action from $A_5$ to $X=\{\text{cong of }K\}$.
    \item $\Stab(K)=N_{A_5}(K)\supset A_4$.
    \item By O.S. trm, $X=|A_5|/|A_4|=5$.
    \item Let $H\subset A_5$ have $|H|=4$.
    \item We want to show that $H$ fixes a point. Equivalently, we want to find $x\in\{1,2,3,4,5\}$ such that $|\Orb(x)|=1$.
    \item Since $4=|H|=|\Orb(x)|\cdot|\Stab(x)|$ and $5\equiv 1\mod 2$. Thus, there is a fixed point.
    \item Thus, there are 15 cyclic subgroups of order 4 like $K$, and they are all conjugate.
    \item $H\leq A_5$ has index $d$ iff there is a transitive action and puts $A_5/H$. Induces a map from $A_5\to S_d$?? As $A_5$ has no normal subgroups. If $d=2,3,4$, ...?? If $d=5$, then $A_5\to S_5\to S_5/A_5$. But really $A_5\to S_5\to S_5/A_5\cong\Z/2\Z$.
    \item The hard ones are 6, 10, or 12.
    \item Consider a subgroup of $A_5$ of order 6. Must be $\Z/6\Z$ or $S_3$. These groups have subgroups of order 3. If we have this, it must be a subgroup of $S_3\times S_2\cap A_5$. Important: $\gen{(1,2,3)}$ and $(1,2)(4,5)$.
    \item Same analysis for subgroups of order 10. Subsets of order 1,2,5,10. $(12)$ orbits include...
    \item Table with sets.
    \item If we spend a couple of hours understanding this example in complete detail, that will be very helpful for the final.
\end{itemize}



\section[\texorpdfstring{$p$}{TEXT}-Groups]{\texorpdfstring{$\bm{p}$}{TEXT}-Groups}
\begin{itemize}
    \item \marginnote{11/9:}\textbf{$\bm{p}$-group}: A finite group of order $p^m$, where $p$ is prime and $m\geq 1$. \emph{Denoted by} $\bm{P}$.
    \item Example: If $|P|=p$, then $P\cong\Z/p\Z$.
    \item \textbf{Fixed point} (of $X$ under $G\acts X$): A point $x\in X$ for which $|\Orb(x)|=1$.
    \item Proposition: Let $P\acts X$ where $P$ is a $p$-group. Then the number of fixed points is congruent to $|X|\mod p$.
    \begin{proof}
        Let $x\in X$ be arbitrary. By the Orbit-Stabilizer theorem,
        \begin{equation*}
            p^m = |P| = |\Orb(x)|\cdot|\Stab(x)|
        \end{equation*}
        If $x$ is a fixed point, then $|\Orb(x)|=1$. However, if $x$ is not a fixed point, then we have by the above that no nontrivial element has order less than $p$ and hence $|\Orb(x)|\equiv 0\mod p$.\par
        As we know,
        \begin{equation*}
            X = \bigsqcup\text{Orbits}
            = \{\text{Fixed points}\}\sqcup\{\text{Non-trivial orbits}\}
        \end{equation*}
        Therefore, $|X|$ is equal to the number of fixed points plus the sum of the magnitudes of the other orbits. But since the magnitudes of the other orbits are all multiples of $p$ as per the above, we have that $|X|$ is congruent to the number of fixed points mod $p$. The desired result readily follows.
    \end{proof}
    \item Corollary: If $|X|\not\equiv 0\mod p$, then there exists at least one fixed point.
    \item \textbf{Center} (of $G$): The set of elements in $G$ that commute with every element of $G$. \emph{Denoted by} $\bm{Z(G)}$. \emph{Given by}
    \begin{equation*}
        Z(G) = \{g\in G\mid gx=xg\ \forall\ x\in G\}
    \end{equation*}
    \item Proposition: Let $P$ be a $p$-group, and $Z:=Z(P)$ be the center of $P$. Then $Z$ is a non-trivial normal subgroup.
    \begin{proof}
        To prove that $Z$ is normal, it will suffice to show that for all $x\in Z$ and $g\in G$, $gxg^{-1}\in Z$. Let $x\in Z$ and $g\in G$ be arbitrary. Then since $x\in Z$, $gx=xg$, i.e., $gxg^{-1}=x\in Z$, as desired.\par
        To prove that $Z$ is non-trivial, we make use of the previous proposition. Let $P\acts P$ by conjugation. We first prove that $Z(P)$ is exactly the set of fixed points of $P$. If $x\in P$ is a fixed point, then $pxp^{-1}=x$ for all $p$, so $x\in Z(P)$. In the other direction, if $x\in Z(P)$ normal, then by the definition of the center, $pxp^{-1}=x$ for all $p\in P$. Thus, $|Z(P)|$ is equal to the number of fixed points of $P$, and hence $|Z(P)|\equiv|P|\mod p\equiv 0\mod p$. Thus, we could have $|Z(P)|=0$, but since $e\in Z(P)$, we must instead have $|Z(p)|\geq p$. Therefore, $Z(P)$ is nontrivial.
    \end{proof}
    \item We get from this proposition an outline for "classifying" $p$-groups. We will do this inductively on $k$. Here are the steps.
    \begin{enumerate}
        \item Understand Abelian $p$-groups.
        \item Understand all $p$-groups of order $|p^k|$.
        \item Let $|P|=p^{k+1}$. Then by the above, $Z\triangleleft P$. If $Z=P$, use 1. If $Z\neq P$, then $|Z|$ and $|P/Z|$ divide $p^k$, so we can use 2.
    \end{enumerate}
    \item Goal: Knowing $Z$ and $G/Z$, try to find all possible $G$.
    \item Classification for $k=2$.
    \begin{enumerate}
        \item Abelian groups. By Lagrange's theorem, there are two possibilities: There exists $x$ with $|x|=p^2$, and there exists $x$ with $|x|=p$.
        \begin{enumerate}
            \item $G$ has an element of order $p^2$, and hence $G\cong\Z/p^2\Z$.
            \item There exists $x\in G$ such that $|x|=p$. Let $y\in G\setminus\gen{x}$. Then $y^p=e$. Thus, $G=\gen{x,y}$. $x^p=e=y^p$ and $xy=yx$. Thus, $G\cong\Z/p\Z\times\Z/p\Z$.
        \end{enumerate}
        \item Suppose $G$ is not abelian. $Z$ still has a nontrivial center, though, and hence any proper nontrivial subgroup of $G$ is necessarily isomorphic to $\Z/p\Z$ for the $k=2$ case. Thus, the only possible pair $(Z,G/Z)$ is $(Z,G/Z)=(\Z/p\Z,\Z/p\Z)$. But then $G/Z\cong\Z/p\Z$ is cyclic, so by HW4 Q5, $G$ is abelian, a contradiction. Therefore, $G\cong\Z/p^2\Z$ or $(\Z/p\Z)^2$, hence abelian.
    \end{enumerate}
    \item (Partial) classification for $k=3$.
    \begin{enumerate}
        \item Abelian groups: $\Z/p^3\Z$, $\Z/p^2\Z\times\Z/p\Z$, and $(\Z/p\Z)^3$.
        \item Possible pairs $(Z,G/Z)$:
        \begin{align*}
            (\Z_{p^2},\Z_p)^\times&&
            (\Z_p,\Z_{p^2})^\times&&
            (\Z_p^2,\Z_p)^\times&&
            (\Z_p,\Z_p^2)^\times
        \end{align*}
        $G/Z$ cyclic implies the same contradiction, so the only possibility is $Z=\Z_p$ and $G/Z=(\Z_p)^2$.
    \end{enumerate}
    \item Does the trend of no nonabelian groups continue for higher powers? No --- for $|G|=2^3=8$, both $D_8$ and $Q$ (the Quaternion group) are nonabelian counterexamples.
    \begin{itemize}
        \item Case 1: All elements in $G$ have order 2.
        \begin{itemize}
            \item $G$ is abelian: If $x,y\in G$ are arbitrary, then
            \begin{equation*}
                xy = xey
                = x(xy)^2y
                = xxyxyy
                = x^2yxy^2
                = eyxe
                = yx
            \end{equation*}
        \end{itemize}
        \item There are, of course, the other abelian groups as well. We now focus on the other case, and specifically its nonabelian forms.
        \item Case 2: There exists $g\in G$ with $|g|=4$.
        \begin{itemize}
            \item $g^2\neq e$.
            \item We also assume that $G$ is not abelian.
            \item $[G:\gen{g}]=2$, so $\gen{g}\triangleleft G$.
            \item Let $h\in G\setminus\gen{g}$. If $|h|=8$, then $G\cong\Z/8\Z$. But $G$ is not abelian, so this cannot be the case.
            \item Hence $|h|=2$ or $|h|=4$.
            \item If $|h|=4$, then $h^2\notin\gen{g}$ implies $G/\gen{g}\cong\Z/2\Z$ (another abelian case we are not interested in). Similarly, $h^2\in\gen{g}$ implies $h^2=g^2$. Thus, either $h^2=e$ or $h^2=g^2$.
            \item Since $\gen{g}\triangleleft G$, $hgh^{-1}\in\gen{g}$. It follows since the powers of $hgh^{-1}$ are as distinct as the powers of $g$ that $\gen{g}=\gen{hgh^{-1}}$. Thus, we either have $hgh^{-1}=g$ or $hgh^{-1}=g^{-1}$. In the first case, $hg=gh$, so $G=\gen{g,h}$ is abelian, and we are not interested.
            \item If $g^4=e=h^4$, then $G=Q$ and $hg=g^{-1}h$.
            \item If $g^4=e=h^2$, then $G=D_8$ and $hg=g^{-1}h$.
        \end{itemize}
    \end{itemize}
    \item We now investigate the case where $p$ is odd and $G=p^3$. Let $Z=\Z/p\Z$ and $G/Z=(\Z/p\Z)^2$.
    \begin{itemize}
        \item Consider a surjection $G\twoheadrightarrow G/Z$. Choose $x\mapsto(1,0)$ and $y\mapsto(0,1)$.
        \item Let $x^p,y^p,xyx^{-1}y^{-1}\in Z$.
        \item If $xy=yx$, then $G=\gen{x,y,Z}$ is abelian.
        \item Suppose $xy=yxz$ for some $z\in Z$ nontrivial.
        \item Case 1: All $g\in G$ have order $p$. Then
        \begin{equation*}
            G = \{y^bx^az^c\mid 0\leq a,b,c\leq p-1\}
        \end{equation*}
        \item We have that
        \begin{equation*}
            y^bx^az^c(y^Bx^Az^C) = y^bx^ay^Bx^Az^{c+C}
            = y^{b+B}x^{a+A}z^{c+C+aB}
        \end{equation*}
        since $xy=yxz$??
        \item This gets into $\text{GL}_3(\F_p)$, the group of $3\times 3$ invertible matrices over the field of numbers 0 to $p$ under addition mod $p$. In particular,
        \begin{equation*}
            \begin{pmatrix}
                1 & a & c\\
                0 & 1 & b\\
                0 & 0 & 1\\
            \end{pmatrix}
            \begin{pmatrix}
                1 & A & C\\
                0 & 1 & B\\
                0 & 0 & 1\\
            \end{pmatrix}
            =
            \begin{pmatrix}
                1 & a+A & c+c+aB\\
                0 & 1 & b+B\\
                0 & 0 & 1\\
            \end{pmatrix}
        \end{equation*}
    \end{itemize}
    \item $p$-groups and their orders for different values of $p,m$.
    \begin{table}[h!]
        \centering
        \small
        \renewcommand{\arraystretch}{1.2}
        \begin{tabular}{c|c|c|c|c}
             & $p$ & $p^2$ & $p^3$ & $p^4$\\
            \hline
            2 & 1 & 2 & $3+2$ & 14\\
            3 & 1 & 2 & $3+2$ & 15\\
            5 & 1 & 2 & $3+2$ & 15\\
            7 & 1 & 2 & $3+2$ & 15\\
        \end{tabular}
        \caption{$|P|$ for various $p,m$ values.}
        \label{tab:pGroupOrder}
    \end{table}
    \item Another perspective.
    \begin{itemize}
        \item Consider $x^p=e=y^p$, $xy=yxz$, $z^p=e$, and $z\in Z(P)$.
        \item Then
        \begin{equation*}
            (xy)^p = y^px^pz^{1+\cdots+p}
            = z^{p(p+1)/2}
        \end{equation*}
        \item If $p$ is odd, then $z^{p(p+1)/2}=e$ implies $(xy)^p=e$ \emph{except} when $p=2$.
    \end{itemize}
\end{itemize}




\end{document}